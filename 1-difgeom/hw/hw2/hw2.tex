\documentclass[10pt, a4paper]{article} % use larger type; default would be 10pt

\usepackage{fontspec} % Font selection for XeLaTeX; see fontspec.pdf for documentation
\defaultfontfeatures{Mapping=tex-text} % to support TeX conventions like ``---''
\usepackage{xunicode} % Unicode support for LaTeX character names (accents, European chars, etc)
\usepackage{xltxtra} % Extra customizations for XeLaTeX
\usepackage{tikz}
% \usepackage{coloremoji}
\usetikzlibrary{arrows,calc,patterns}


% other LaTeX packages.....
\usepackage{fullpage}
\usepackage[top=2cm, bottom=4.5cm, left=2.5cm, right=2.5cm]{geometry}
\usepackage{amsmath,amsthm,amsfonts,amssymb,amscd,systeme}
\usepackage{unicode-math}
\usepackage{cancel}
\geometry{a4paper} 
%\usepackage[parfill]{parskip} % Activate to begin paragraphs with an empty line rather than an indent
\usepackage{fancyhdr}
\usepackage{listings}
\usepackage{graphicx}
\usepackage{hyperref}
\usepackage{multicol}

% FONTS
% \setmainfont[Ligatures=TeX]{Cambria Math} % set the main body font (\textrm), assumes Charis SIL is installed
%\setsansfont{Deja Vu Sans}
% \setmonofont[Ligatures=TeX]{Fira Code}
\setmathfont[Ligatures=TeX]{NewCMMath-Regular}

\setmainfont{Cambria}
\setmonofont[Ligatures=TeX]{Roboto Mono}

\renewcommand\lstlistingname{Algorithm}
\renewcommand\lstlistlistingname{Algorithms}
\def\lstlistingautorefname{Alg.}
\lstdefinestyle{mystyle}{
    % backgroundcolor=\color{backcolour},   
    % commentstyle=\color{codegreen},
    % keywordstyle=\color{magenta},
    % numberstyle=\tiny\color{codegray},
    % stringstyle=\color{codepurple},
    basicstyle=\ttfamily\footnotesize,
    breakatwhitespace=false,         
    breaklines=true,                 
    captionpos=b,                    
    keepspaces=true,                 
    numbers=left,                    
    numbersep=5pt,                  
    showspaces=false,                
    showstringspaces=false,
    showtabs=false,                  
    tabsize=2
}
\lstset{style=mystyle}

\newcommand\course{MS1 - Differential geometry and topology}
\newcommand\hwnumber{HW 2}                   % <-- homework number
\newcommand\idgroup{111-2023}                
\newcommand\idname{Mykhailo Koreshkov}  

\usepackage[framemethod=TikZ]{mdframed}
\mdfsetup{%
	backgroundcolor = black!5,
}
\mdfdefinestyle{ans}{%
    backgroundcolor = green!5,
    linecolor = green!50,
    linewidth = 1pt,
}

\pagestyle{fancyplain}
\headheight 35pt
\lhead{\idgroup \\ \idname}
\chead{\textbf{\Large \hwnumber}}
\rhead{\course \\ \today}
\lfoot{}
\cfoot{}
\rfoot{\small\thepage}
\headsep 1.5em

\linespread{1.2}

\newcommand{\R}{\mathbb{R}}
\newcommand{\N}{\mathbb{N}}
\newcommand{\Z}{\mathbb{Z}}
\DeclareMathOperator{\lcm}{lcm}
\DeclareMathOperator{\cd}{CD}

\renewcommand{\B}{\mathcal{B}}

\begin{document}


\section*{Ex 2.1}
\begin{mdframed}
Нехай $X$ - топологічний простір. Показати еквівалентність
\begin{enumerate}
    \item $\forall x \in X: \exists n\ge 0, U \text{ - open in } X, \varphi:U \to \R^n \text{ - hom}$
    \item $\forall x \in X: \exists n\ge 0, U \text{ - open in } X, \varphi:U \to \varphi(U) \subset \R^n \text{ - hom}$
\end{enumerate}
\end{mdframed}
Доведення 1->2 очевидне. Доводжу 2->1.
\begin{proof}
    Фіксуємо 
    $x\in X, n\ge0, U \text{ - open in } X, \varphi:U \to \varphi(U) \subset \R^n \text{ - hom}$.\\
    Хочемо довести існуваня гомеомеорфізму з (можливо іншого) околу точки $x$ на весь $\R^n$. 
    Число $n$ залишаємо те саме через теорему про інваріантність області.
\begin{itemize}
    \item 
    Нехай $\varphi(x) = y, \varphi(U) = Y$.\\
    Оскільки $\varphi^{-1}$ - неперервне, з відкритості $U$ випливає відкритість $\varphi(U)$.
    З визначення відкритої множини в $\R^n$ можемо вписати в $Y$ відкриту кулю навколо $y$:
    \[\exists r>0: B_r(y) \subset Y\]
    Де $B_r(y) = \{z\in \R^n : |z-y| < r\}$.
    
    Нехай $B = B_r(y), V = \varphi^{-1}(B)$. Зауважу, що $\varphi$ - неперервна бієкція, а тому $V$ відкрита та містить $x$.
    Тобто, $V$ - дійсно відкритий окіл $x$ в просторі $X$.

    \item
    Розглянемо 
    \[\psi(v) = \frac{\frac{v-y}{r}}{1 - \frac{|v-y|^2}{r^2}} : B_r(y) \to \R^n\]
    Відомо, що $\psi$ - гладкий гомеомеорфізм.
    Тоді 
    \[\alpha = \psi\circ\varphi : V \to \R^n, \quad \alpha \text{ - hom}\] 
\end{itemize}
Отже, знайшли $V$ - окіл $x$ та $\alpha : V \to \R^n$ - hom для довільного $x \in X$.
\end{proof}

\newpage
\section*{Ex 2.2}
\begin{mdframed}
    Перевірити наступні властивості топологічного многовиду $X$.
\end{mdframed}

\subsection*{Ex 2.2.1}
\begin{mdframed}
    $X$ є локально лінійно зв'язним, тобто у кожної точки є база, що складається з лінійно зв'язних околів.
\end{mdframed}

\begin{mdframed}[backgroundcolor=blue!10]
    $\B = \{B_i\} \subset \tau$ є базою точки $x$ якщо 
    \[\forall U\in\tau: x\in U \implies \exists I: U = \bigcup_{i\in I} B_i \]
    Або еквівалентно
    \[(\forall \alpha: x \in \B_\alpha) \land (\forall U_x \text{ - neighborhood of }x: \exists B \in \B: x \in B \subset U_x)\]
\end{mdframed}

% \begin{mdframed}[backgroundcolor=blue!10]
    Альтернативне формулювання локальної лінійної зв'язності.
    \[\forall U_x \text{ - neighborhood of }x: \exists V\in\tau: x \in V \subset U_x, \;V \text{ - path connected}\]

    Доведу еквівалентність визначень через базу (1) та через \\ 
    відкриті зв'язні підмножини (2):
    \begin{proof}[Доведення 1->2]
        Нехай $\B$ - база $x\in X$.
        Нехай $\forall B \in \B: B$ - лінійно зв'язна.
        Нехай $U_x$ - деякий окіл $x$.  
        З визначення околу $\exists U\in\tau: x \in U \subset U_x$.
        З визначення бази слідує
        \[\exists B \in \B: x \in B \subset U \;(\subset U_x)\]
        Але кожна $B\in\B$ line connected.
        Тобто знайшли $B: x\in B\subset U_x$ таку що $B$ - path connected.
    \end{proof}
    \begin{proof}[Доведення 2->1]
        \[\forall U_x \text{ - neighborhood of }x: \exists V\in\tau: x \in V \subset U_x, \;V \text{ - path connected}\]
        Візьмемо систему околів $x$:
        \[N(x) = \{U_x\} = \{U_x : \exists V\in \tau: x\in V \subset U_x\}\]
        Для кожного околу застосую визначення (2):
        \[S = \{V : U_x \in N(x), V \in \tau, x \in V \subset U_x, V \text{ - path connected}\}\]
        Кожен елемент $S$ лінійно зв'язний. $S$ - це також база точки $x$, оскільки
        $\forall V \in S: x \in V$ та \\
        $\forall U_x: \exists V \in S: x\in V \subset U_x$ за побудовою множини $S$.
    \end{proof}
% \end{mdframed}
Отже, довели еквівалентність визначень локально лінійно зв'язного простору.

Доведення властивості 2.2.1
\begin{proof}[Властивість 2.2.1]
    Візьмемо $U_x$ - довільний окіл $x$.
    З визначення околу, $\exists U \in \tau: x \in U \subset U_x$.\\
    З визначення многовиду, $\exists (V,\phi), V \in \tau$ - локальна карта для точки $x$.
    \[\phi:V \to \R^n \text{ - hom}\]
    Нехай $H = U \cap V$. $H$ відкрита як скінченний перетин відкритих множин.  
    $\psi = \phi|_H : H \to \R^n$ - обмеження $\phi$ на $H$ - все ще гомеоморфізм.  

    Візьмемо довільні $a,b \in H$. 
    $\R^n$ - лінійно зв'язний. Тобто 
    \[\exists f: [0,1] \to \R^n: f\in C^0, f(0)=\psi(a), f(1)=\psi(b)\]
    $\psi$ - hom, отже $g = \left(\psi^{-1} \circ f\right) : [0,1] \to H$ - також неперервне відображення.  
    Більше того, $g(0) = \psi^{-1}(\psi(a)) = a, g(1) = \psi^{-1}(\psi(b)) = b$.
    
    Таким чином для довільного околу $x$ знайшли відкритий лінійно зв'язний підокіл, 
    що еквівалентно локальній лінійній зв'язності простору $X$.
\end{proof}

\subsection*{Ex 2.2.2}
\begin{mdframed}
    Довести що $X$ є локально компактним. Тобто у кожної точки є база що скаладється з околів із компактними замиканнями (1)
\end{mdframed}

Альтернативне визначення локально компактного простору
\[\forall x: \exists U_x \text{ - neighborhood}: U_x \text{ - compact} \quad (2)\]

\begin{proof}[Доведення властивості 2.2.2 для визначення (2)]
    Фіксуємо $x\in X$.
    $\exists (U,\phi)$ - локальна карта.
    \[\phi : U \to \R^n \text{ - homeomorphism}\]
    Беремо $D = D_1(\phi(x)) = \{u \in \R^n: |u-\phi(x)|\le 1\}$ - замкнена куля навколо образу $x$ в $\R^n$.\\
    $D$ - компакт, оскільки обмежена та замкнена. Компактність змерігається під дією гомеоморфізмів.
    Тоді $\phi^{-1}(D)$ також компакт в $X$.\\
    $B = B_1(\phi(x))$ - відкрита куля в $\R^n$, $B \subset D$.
    $\phi^{-1}(B)$ - відкрита в $X$. При чому $\phi^{-1}(B) \subset \phi^{-1}(D)$.
    А тоді $V = \phi^{-1}(D)$ - компактний окіл $x$ в $X$.
    
    Таким чином ми показали що для кожної точки $x\in X$ існує компактний окіл.
\end{proof}

\begin{proof}[Доведення властивості 2.2.2 для визначення (1)]
    Фіксуємо точку $x\in X$.
    Хочу показати що для довільного околу $x$ знайдеться локальна карта та відповідний 
    прообраз $\R^n$ що лежить в цьому околі. 
    
    Розгляну довільний окіл $V_x$ та деяку локальну карту $(U,\phi)$.
    $\exists V\in\tau: x\in V \subset V_x$\\
    Let $W_o = V \cap U \subset V_x$. $W_o$ відкрита як перетин двох відкритих множин.\\
    Let $W = \overline W$ - замикання $W$. Тоді $W \subset V_x$ (це неочевидне твердження).\\
    $A = \phi(W)$ - замкнена в $\R^n$ бо $\phi^{-1} \in C^0$.\\
    $A$ - компакт в $\R^n$ як замкнена та обмежена множина.
    А тоді й $W$ є компактом.
    Тобто для довільного окола $x$ знайшли компактний підокіл.

    Повторюємо цей процес для всіх можливих околів $x$.
    $S = \{W(U_x) : U_x \in \mathcal N(x)\}$.

    $S$ - це база точки $x$ за побудовою, у якої кожен елемент є компактом.

    Таким чином довели локальну компактність $X$.
\end{proof}

\section*{Ex 2.2.3}
\begin{mdframed}
    $X$ має зліченну базу в кожній точці.
\end{mdframed}

\begin{proof}
    Фіксуємо $x\in X$. У точки є локальна карта $(U,\phi)$.\\
    Let $y = \phi(x) \in \R^n$.
    В $\R^n$ у кожної точки є зліченна база. Бізьмемо таку для точки $y$ та назвемо $A = \{A_i\}$.

    Нехай $B = \{\phi^{-1}(A_i) : A_i \in A\}$.\\
    $|B| \le |A|$, $B$ не більше ніж зліченна множина.
    Далі, оскільки $\forall i: y\in A_i$, маємо $\forall B_i: x \in B_i = \phi^{-1}(A_i)$.
    З іншого боку, оскільки $A$ - база в $\R^n$
    \[\forall U_x \in \mathcal{N}(x): \exists A_i \in A: \phi(x) \in A_i \subset \phi(U_x)\]
    А тому
    \[\forall U_x \in \mathcal{N}(x): \exists B_i=\phi^{-1}(A_i) \in B: x \in B_i \subset U_x\]
    
    Отже, для довільної точки знайшлли зліченну множину околів, яка є базою.
\end{proof}

\section*{Ex 2.2.4}
\begin{mdframed}
    Нехай $X$ - многовид, $Y \subset X$.
    Довести що $Y$ - також многовид.
\end{mdframed}

Розглянемо індуковану топологію на $Y$:
\[(Y,\chi), \quad \chi = \{U\cap Y : U \in \tau\}\]
Доведу, що $(Y,\chi)$ - також многовид.

\begin{proof}
    Візмемо довільну точку $x\in Y\subset X$.
    $X$ - многовид, отож $\exists (U,\phi)$ - локальна карта з $X$, при чому\\
    $\phi: U \to \R^n$ - homeomorphism.
    Нехай $W = U\cap Y \subset U$. Розгляну обмеження $\phi|_W : W \to \phi(W)$. 
    Покажемо, що $\phi|_W$ - також гомеомеорфізм.
    
    \[V \in \chi \implies V \in \tau\]
    \[\forall V \subset \phi(W), V \in \tau_{\R^n}: \phi^{-1}(V) \subset W \land \phi^{-1}(V) \in \tau\]
    \[\forall V \subset W, V \in \chi: \phi(V) \subset \phi(W), \phi(V) \in \tau_{\r^n}\]

    Отже, для кожної точки знайшли локальну карту. Тому $Y$ - многовид.
\end{proof}

% \section*{Ex 2.2.7}
% \begin{mdframed}
%     $(U,\phi)$ - локальна карта.
%     $\phi: U \to \R^n$ - homeomorphism.
%     $V \subset U$ - відкрита.
%     $\alpha:\R^n \to \R^n$ - довільний гомеомеорфізм.

%     Довести, що $\alpha \circ \phi|_V : V \to \R^n$ також є локальною картою
% \end{mdframed}
% \begin{proof}
    
% \end{proof}



\end{document}

