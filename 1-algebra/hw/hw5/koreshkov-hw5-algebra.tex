\documentclass[11pt, a4paper]{article} % use larger type; default would be 10pt

\usepackage{fontspec} % Font selection for XeLaTeX; see fontspec.pdf for documentation
\defaultfontfeatures{Mapping=tex-text} % to support TeX conventions like ``---''
\usepackage{xunicode} % Unicode support for LaTeX character names (accents, European chars, etc)
\usepackage{xltxtra} % Extra customizations for XeLaTeX
\usepackage{tikz}
\usetikzlibrary{arrows,calc,patterns}


% other LaTeX packages.....
\usepackage{fullpage}
\usepackage[top=2cm, bottom=4.5cm, left=2.5cm, right=2.5cm]{geometry}
\usepackage{amsmath,amsthm,amsfonts,amssymb,amscd,systeme}
\usepackage{unicode-math}
\usepackage{cancel}
\geometry{a4paper} 
%\usepackage[parfill]{parskip} % Activate to begin paragraphs with an empty line rather than an indent
\usepackage{fancyhdr}
\usepackage{listings}
\usepackage{graphicx}
\usepackage{hyperref}
\usepackage{multicol}

% FONTS
% \setmainfont[Ligatures=TeX]{Cambria Math} % set the main body font (\textrm), assumes Charis SIL is installed
%\setsansfont{Deja Vu Sans}
% \setmonofont[Ligatures=TeX]{Fira Code}
\setmathfont[Ligatures=TeX]{NewCMMath-Regular}

\setmainfont{Cambria}
\setmonofont[Ligatures=TeX]{Roboto Mono}

\renewcommand\lstlistingname{Algorithm}
\renewcommand\lstlistlistingname{Algorithms}
\def\lstlistingautorefname{Alg.}
\lstdefinestyle{mystyle}{
    % backgroundcolor=\color{backcolour},   
    % commentstyle=\color{codegreen},
    % keywordstyle=\color{magenta},
    % numberstyle=\tiny\color{codegray},
    % stringstyle=\color{codepurple},
    basicstyle=\ttfamily\footnotesize,
    breakatwhitespace=false,         
    breaklines=true,                 
    captionpos=b,                    
    keepspaces=true,                 
    numbers=left,                    
    numbersep=5pt,                  
    showspaces=false,                
    showstringspaces=false,
    showtabs=false,                  
    tabsize=2
}
\lstset{style=mystyle}

\hypersetup{
  colorlinks   = true, %Colours links instead of ugly boxes
  urlcolor     = blue, %Colour for external hyperlinks
  linkcolor    = blue, %Colour of internal links
  citecolor   = red %Colour of citations
}


\newcommand\course{MS1 - Algebra}
\newcommand\hwnumber{HW 5}                   % <-- homework number
\newcommand\idgroup{111-2023}                
\newcommand\idname{Mykhailo Koreshkov}  

\usepackage[framemethod=TikZ]{mdframed}
\mdfsetup{%
	backgroundcolor = black!5,
}
\mdfdefinestyle{ans}{%
    backgroundcolor = green!5,
    linecolor = green!50,
    linewidth = 1pt,
}

\pagestyle{fancyplain}
\headheight 35pt
\lhead{\idgroup \\ \idname}
\chead{\textbf{\Large \hwnumber}}
\rhead{\course \\ \today}
\lfoot{}
\cfoot{}
\rfoot{\small\thepage}
\headsep 1.5em

\linespread{1.1}

\newcommand{\R}{\mathbb{R}}
\newcommand{\N}{\mathbb{N}}
\newcommand{\Z}{\mathbb{Z}}
\newcommand{\F}{\mathbb{F}}
\newcommand{\Q}{\mathbb{Q}}
\DeclareMathOperator{\lcm}{lcm}
\DeclareMathOperator{\cd}{CD}
\DeclareMathOperator{\ch}{char}
\DeclareMathOperator{\ob}{Ob}
\DeclareMathOperator{\mor}{Mor}
\DeclareMathOperator{\catring}{Ring}

\newtheorem*{proposition}{Твердження}
\newtheorem*{definition}{Визначення}

\begin{document}


\section*{Ex 5.1}
\begin{mdframed}
Нехай $R,S$ - кільця. Довести, що $(R\times S)^* = R^* \times S^*$
\end{mdframed}

\begin{proof}
    Let $P = R\times S$.
    Розгляну $P^*$.
    \[P^* = \{(r,s): r\in R, s\in S \& \exists (r',s'): (r,s)\cdot (r',s') = (1_R, 1_S)\}\]
    \[(r,s)\cdot (r',s') = (rr',ss') = (1_R, 1_S) \iff r'=r^{-1}, s'=s^{-1}\]
    Таким чином $p \in P^* \iff p \in R^*\times S^*$
\end{proof}

\section*{Ex 5.2}
\begin{mdframed}
    $R$ - кільце, $S \subset R, 0_R \in S, 1_R \in S$.
    Нехай також
    \[\forall a,b \in S: ab \in S\]
    \[\forall a,b \in S: a-b = a+(-b) \in S\]
    Довести, що $S$ -  підкільце $R$.
\end{mdframed}

\begin{mdframed}[backgroundcolor=blue!10]
    \begin{proposition}[5.11]
        $S$ - підкільце $R$ тоді і лише тоді коли
        \begin{itemize}
            \item $S$ замкнена відносно операцій на $R$
            \item $\forall a\in S: -a \in S$
        \end{itemize}
        Замість другої умови існує еквівалентна: $-1_R\in S$
    \end{proposition}
\end{mdframed}

\begin{proof}
    Дано замкненість відносно множення. Доведу замкненість відносно додавання.\\
    Дано
    \[\forall a,b \in S: a-b = a+(-b) \in S\]
    Покладу $a=0, b$ - довільне. З цього маємо
    \[\forall b\in S: 0-b = 0+(-b) = -b \in S\]
    Тобто $\forall b\in S: \exists -b \in S$ - існування зворотнього елемента в $S$.
    Далі, для довільних $a,b\in S$:
    \[\forall a,b\in S: a+b = a-(-b) \in S\]
    Тобто, довели замкненість відносно додавання.

    За твердженням 5.11, цього достатньо щоб $S$ було підкільцем $R$.
\end{proof}

\section*{Ex 5.3}
\begin{mdframed}
    $S = \{[0]_6, [3]_6\} \subset \Z/6\Z$\\
    Відомо, що $S$ замкнена відносно операцій.
    \begin{enumerate}
        \item Чи є $S$ кільцем?
        \item Чи є $S$ підкільцем $R$?
    \end{enumerate}
\end{mdframed}

\subsection*{а) Чи є $S$ кільцем?}
\begin{enumerate}
    \item Коректність операцій - ОК
    \item Асоціативність - ОК
    \item Нуль: $O_S = O_R = [0]_6$
    \item Обернені відносно додавання: $[3]_6+[3]_6=[6]_6=[0]_6$ - ОК
    \item Комутативність додавання - ОК
    \item Асоціативність множення - ОК
    \item Одиниця: $[0]_6\cdot [3]_6 = 0, [3]_6\cdot [3]_6 = [9]_6 = [3]_6$
    \[1_S = [3]_6\]
    \item Дистрибутивність - ОК
\end{enumerate}
Отже, $S$ є кільцем

\subsection*{б) Чи є $S$ підкільцем $R$?}
Ні, $S$ не є підкільцем, бо $1_R = [1]_6 \notin S$, $1_R \ne 1_S$.

\section*{Ex 5.4}
\begin{mdframed}
    Нехай $R$ - область цілісності.
    Нехай $S$ - підкільце $R$.
    Довести, що $S$ - також область цілісності. 
\end{mdframed}

\begin{proof}
    Зауважу, що $0_R = 0_S = 0, 1_R = 1_S = 1$.\\
    Припустимо, що $S$ - не область цілісності.
    Тобто
    \[\exists a,b\in S\subset R : a\ne 0_S, b\ne 0_S, ab = 0_S\]
    Але це означає, що 
    \[\exists a,b\in R: a\ne 0_R, b\ne 0_R, ab=0_R\]
    що суперечить тому, що $R$ - область цілісності.
    
    Таким чином довели, що $S$ - також область цілісності.
\end{proof}

\section*{Ex 5.5}
\begin{mdframed}
    \[S = \{a+b\sqrt{2} : a,b\in \Q\}\]
    Довести, що $S$ - підполе $\R$.

    Тобто, $S$ - підкільце $\R$, що є полем.
\end{mdframed}

\begin{proof}
    Щоб довести, що $S$ - підкільце, достатньо довести замкненість відносно операцій та існування зворотніх елементів.
    Нехай
    \[x,y\in S, \qquad x=a_1 + b_1\sqrt{2}, \qquad y = a_2+b_2\sqrt{2}\]
    Тоді
    \[\forall x,y\in S: x+y = (a_1+a_2) + (b_1+b_2)\sqrt{2}, \quad (a_1+a_2)\in \Q, (b_1+b_2) \in \Q\]
    \[\forall x,y\in S: xy = (a_1a_2 + 2b_1b_2) + (a_1b_2+b_1a_2)\sqrt{2}, \quad (a_1a_2 + 2b_1b_2)\in \Q, (a_1b_2+b_1a_2) \in \Q\]
    Тобто, $S$ - замкнена відносно додавання та множення.

    Також зауважу, що 
    \[\forall x\in S: x+0_R = x = 0_R+x\]
    \[\forall x\in S: x\cdot 1_R = x = 1_R\cdot x\]
    Тобто, $1_R = 1_S, 0_R = 0_S$

    Існування обернених:
    \[-1_R = -1 + 0\sqrt{2} \in S\]
    або інакше
    \[\forall x\in S, x=a+b\sqrt{2}: \exists y = (-a)+(-b)\sqrt{2} \in S: x+y=0_R=0_S\]

    Отже, $S$ - підкільце $\R$. Залишається довести, що $S$ це поле. \\
    Тобто, довести комутативність множення та існування оборотних елементів.
    \[\forall x,y\in S: xy = (a_1+b_1\sqrt{2})(a_2+b_2\sqrt{2}) = (a_2+b_2\sqrt{2})(a_1+b_1\sqrt{2}) = yx\]

    Let $a_1\ne 0, b_1 \ne 0$. Let
    \[xy = (a_1+b_1\sqrt{2})(a_2+b_2\sqrt{2}) = (a_1a_2 + 2b_1b_2) + (a_1b_2+b_1a_2)\sqrt{2} = 1\]
    тоді
    \[\begin{cases}
        a_1a_2 + 2b_1b_2 = 1\\
        a_1b_2 + b_1a_2 = 0
    \end{cases} \qquad \begin{cases}
        \frac{a_2}{a_1} = -\frac{b_2}{b_1} = t\\
        a_2 = ta_1\\
        b_2 = -tb_1\\
        ta_1^2 + 2tb_1^2 = t(a_1^2+2b_1^2) = 1
    \end{cases}\]
    \[\begin{cases}
        a_2 = ta_1\\
        b_2 = -tb_1\\
        t = (a_1^2 + 2b_1^2)^{-1}
    \end{cases}\]
    Тобто
    \[\forall x\in S, x\ne 0: x^{-1} = \frac{a_1-b_1\sqrt{2}}{a_1^2 + 2b_1^2}\]
    Перевірка
    \[(a+b\sqrt{2})\frac{a-b\sqrt{2}}{a^2 + 2b^2} = \frac{a^2 + 2b^2}{a^2 + 2b^2} = 1\]
    Зауважу, що $(a_1^2 + 2b_1^2)^{-1} \in \Q$.

    Отже, $S^* = S \setminus \{0_S\}$, тобто $S$ - поле.

    Таким чином довели, що $S$ - підполе $\R$.
\end{proof}

\section*{Ex 5.6}
\begin{mdframed}
    Нехай $f: R\to S$ - гомоморфізм кілець.
    Довести:
    \begin{itemize}
        \item $f(b-a) = f(b)-f(a)$
        \item $f(-a) = -f(a)$
    \end{itemize}
\end{mdframed}

\begin{proof}
    $f$ - гомоморфізм, тобто
    \begin{gather*}
        f(a+b) = f(a)+f(b)\\
        f(ab) = f(a)f(b)
    \end{gather*}

    \begin{enumerate}
        \item Доведу $f(-a) = -f(a)$ довівши, що $f(a) + f(-a) = 0_S$.
        \[f(a) + f(-a) = f(a+(-a)) = f(0_R) = 0_S\]
        Тобто дійсно $-f(a) = f(-a)$
        
        \item Доведу $f(b-a) = f(b)-f(a)$\\
        \[f(b)-f(a) = f(b)+(-f(a)) = f(b) + f(-a) = f(b+(-a)) = f(b-a)\]
    \end{enumerate}
 
\end{proof}


\section*{Ex 5.7}\label{ex5.7}
\begin{mdframed}
    Нехай $\sigma: \Z \to R$ - гомоморфізм, $R$ - кільце.
    \[\sigma(k) = k\cdot 1_R\]
    Довести, що $\sigma$ - єдиний гомоморфізм.
\end{mdframed}

\begin{proof}
    Нехай $\phi: \Z \to R$ - довільний гомеоморфізм.
    Тоді
    \[\forall k\in\Z, k>0: \phi(k) = \phi(\underbrace{1+1+\cdots+1}_k) = \underbrace{\phi(1) + \phi(1) + \cdots + \phi(1)}_k 
    = k \cdot \phi(1) = k \cdot 1_R\]
    \[\phi(0) = 0_R = 0\cdot 1_R\]
    \begin{align*}
        \forall k\in\Z, k<0: \phi(k) &= \phi(\underbrace{(-1)+(-1)+\cdots+(-1)}_{|k|}) = \\
        &= \underbrace{\phi(-1) + \phi(-1) + \cdots + \phi(-1)}_{|k|} = |k| \cdot \phi(-1) = \\
        &= |k| \cdot (-1_R) = k \cdot 1_R
    \end{align*}
    Тобто, $\forall k\in\Z: \phi(k) = k\cdot 1_R$, а отже, $\phi = \sigma$
\end{proof}

\section*{Ex 5.8}

\begin{definition}[Ініціальний об'єкт]
    Ініціальний об'єкт категорії $\mathcal A$:
    \[A \in \ob \mathcal A: \forall B \in \ob \mathcal A: \exists! f \in \mor(A,B)\]
\end{definition}
\begin{definition}[Термінальний об'єкт]
    Термінальний об'єкт категорії $\mathcal A$:
    \[D \in \ob \mathcal A: \forall C \in \ob \mathcal A: \exists! g \in \mor(C,D)\]
\end{definition}
\begin{definition}[Категорія $\catring$]
    Категорія кілець та гомоморфізмів кілець
\end{definition}
\begin{proposition}[Про ініціальний об'єкт категорії $\catring$]
    З \hyperref[ex5.7]{Ex 5.7} випливає, що $\Z$ - ініціальний об'єкт.
\end{proposition}

\begin{mdframed}
    Вказати термінальний об'єкт категорії $\catring$ або довести що його не існує.
\end{mdframed}

Я припускаю, що термінальним об'єктом буде $T = \{0_T\}$.
Перевіримо.

Нехай $R$ - довільне кільце. Існує лише єдине відображення $\alpha: R\to T$: $\alpha(a) = 0_T$.
\begin{itemize}
    \item $\alpha(0_R) = 0_T, \alpha(1_R) = 0_T = 1_T$
    \item $\alpha(a+b) = 0_T = \alpha(a) + \alpha(b) = 0_T + 0_T = 0_T$
    \item $\alpha(ab) = 0_T = \alpha(a)\alpha(b) = 0_T\cdot 0_T = 0_T$
\end{itemize}
Тобто, $\alpha$ - дійсно гомоморфізм кілець і це єдиний гомоморфізм на $T$.

Отже, тривіальне кільце $T$ - дійсно термінальний об'єкт категорії $\catring$

\end{document}

