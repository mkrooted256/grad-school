\documentclass[11pt, a4paper]{article} % use larger type; default would be 10pt

\usepackage{fontspec} % Font selection for XeLaTeX; see fontspec.pdf for documentation
\defaultfontfeatures{Mapping=tex-text} % to support TeX conventions like ``---''
\usepackage{xunicode} % Unicode support for LaTeX character names (accents, European chars, etc)
\usepackage{xltxtra} % Extra customizations for XeLaTeX
\usepackage{tikz}
\usetikzlibrary{arrows,calc,patterns}


% other LaTeX packages.....
\usepackage{fullpage}
\usepackage[top=2cm, bottom=4.5cm, left=2.5cm, right=2.5cm]{geometry}
\usepackage{amsmath,amsthm,amsfonts,amssymb,amscd,systeme}
\usepackage{unicode-math}
\usepackage{cancel}
\geometry{a4paper} 
\usepackage[parfill]{parskip} % Activate to begin paragraphs with an empty line rather than an indent
\usepackage{fancyhdr}
\usepackage{listings}
\usepackage{graphicx}
\usepackage{hyperref}
\usepackage{multicol}


\usepackage{faktor}


% FONTS
% \setmainfont[Ligatures=TeX]{Cambria Math} % set the main body font (\textrm), assumes Charis SIL is installed
%\setsansfont{Deja Vu Sans}
% \setmonofont[Ligatures=TeX]{Fira Code}
\setmathfont[Ligatures=TeX]{NewCMMath-Regular}

\setmainfont{Cambria}
\setmonofont[Ligatures=TeX]{Roboto Mono}

\renewcommand\lstlistingname{Algorithm}
\renewcommand\lstlistlistingname{Algorithms}
\def\lstlistingautorefname{Alg.}
\lstdefinestyle{mystyle}{
    % backgroundcolor=\color{backcolour},   
    % commentstyle=\color{codegreen},
    % keywordstyle=\color{magenta},
    % numberstyle=\tiny\color{codegray},
    % stringstyle=\color{codepurple},
    basicstyle=\ttfamily\footnotesize,
    breakatwhitespace=false,         
    breaklines=true,                 
    captionpos=b,                    
    keepspaces=true,                 
    numbers=left,                    
    numbersep=5pt,                  
    showspaces=false,                
    showstringspaces=false,
    showtabs=false,                  
    tabsize=2
}
\lstset{style=mystyle}

\hypersetup{
  colorlinks   = true, %Colours links instead of ugly boxes
  urlcolor     = blue, %Colour for external hyperlinks
  linkcolor    = blue, %Colour of internal links
  citecolor   = red %Colour of citations
}


\newcommand\course{MS1 - Algebra}
\newcommand\hwnumber{HW 8}                   % <-- homework number
\newcommand\idgroup{111-2023}                
\newcommand\idname{Mykhailo Koreshkov}  

\usepackage[framemethod=TikZ]{mdframed}
\mdfsetup{%
	backgroundcolor = black!5,
}
\mdfdefinestyle{ans}{%
    backgroundcolor = green!5,
    linecolor = green!50,
    linewidth = 1pt,
}

\pagestyle{fancyplain}
\headheight 35pt
\lhead{\idgroup \\ \idname}
\chead{\textbf{\Large \hwnumber}}
\rhead{\course \\ \today}
\lfoot{}
\cfoot{}
\rfoot{\small\thepage}
\headsep 1.5em

\linespread{1.1}

\newcommand{\R}{\mathbb{R}}
\newcommand{\N}{\mathbb{N}}
\newcommand{\Z}{\mathbb{Z}}
\newcommand{\F}{\mathbb{F}}
\newcommand{\Q}{\mathbb{Q}}
\DeclareMathOperator{\lcm}{lcm}
\DeclareMathOperator{\cd}{CD}
\DeclareMathOperator{\ch}{char}
\DeclareMathOperator{\ob}{Ob}
\DeclareMathOperator{\mor}{Mor}
\DeclareMathOperator{\catring}{Ring}

\newtheorem*{proposition}{Твердження}
\newtheorem*{definition}{Визначення}

\begin{document}


\section*{Ex 8.1}
\begin{mdframed}
    Нехай $f: R \to S$ - гомоморфізм комутативних кілець.
    Нехай $J$ - простий ідеал в $S$. \\
    Довести, що $I = f^{-1}(J)$ - простий ідеал в $R$.
\end{mdframed}

\begin{mdframed}[backgroundcolor=purple!20]
    Ідеал $J$ є простим якщо $R/J$ - область цілісності.\\
    Еквівалентно, $\forall a,b \in R: (ab\in J) \iff (a\in J)\vee(b\in J)$.

    Також знаємо, що повний прообраз $f^{-1}(J)$ ідеала буде ідеалом.
% \begin{proof}[Твердження про прообраз ідеала]
%     Нехай $f: R \to S$ - гомоморфізм комутативних кілець.
%     Нехай $J$ - ідеал в $S$. Нехай $I = f^{-1}(J)$.

%     Візьмемо довільні $a,b\in I$ та $c \in R$.\\
%     Відображення є гомоморфізмом, отже $f(a+b) = f(a)+f(b) \in J$ з замкненості $J$.\\
%     Оскільки $f(a)\in J$, за властивістю поглинання маємо $f(ca) = f(c)\cdot f(a) \in J$.

%     Тобто, $I = f^{-1}(J)$ - ідеал кільця $R$.
% \end{proof}
\end{mdframed}

\begin{proof}
    Доведення від супротивного.\\
    Нехай $f: R \to S$ - гомоморфізм комутативних кілець.
    Нехай $J$ - простий ідеал в $S$.
    Знаємо, що $I = f^{-1}(J)$ буде ідеалом. 
    Припустимо, що $I = f^{-1}(J)$ не є простим ідеалом, тобто
    $\exists a,b \in R: ab \in I \land a\notin I \land b \notin I$.\\
    Іншими словами,
    $\exists a,b \in R: f(ab) \in J \land f(a) \notin J \land f(b) \notin J$.\\
    Оскільки $f$ це гомоморфізм, маємо $f(a)f(b) \in J, f(a)\notin J, f(b)\notin J$, а це суперечить умові простоти ідеалу $J$.

    Таким чином довели, що ідеал $I$ має бути простим.
\end{proof}

\section*{Ex 8.2}
\begin{mdframed}
    Нехай $f: R \to S$ - гомоморфізм комутативних кілець.
    Нехай $J$ - максимальний ідеал в $S$. \\
    Показати, що $I = f^{-1}(J)$ в загальному випадку не буде максимальним ідеалом в $R$.

    Знайти достатні умови.
\end{mdframed}

Нехай $f : \Z \to \Z[x], \; f(a) = a + 0x + ...$.

Розглянемо $(2) = \{\sum_{k=0}^n a_kx^k : 2 \mid a_k \forall k\}$.\\
$\Z[x]/(2) = $


\section*{Ex 8.3}
\begin{mdframed}
    Нехай $R$ - скінченне комутативне кільце, $I$ - його простий ідеал.\\
    Довести, що $I$ є максимальним.
\end{mdframed}

Нехай $|R| = n < \infty$. 

З того, що $I$ - простий, маємо $I \ne (1)$.\\
$R/I$ буде скінченною областю цілісності та $0_{R/I} \ne 1_{R/I}$.

% \begin{proof}[Твердження 4.31]
%     Розглянемо відображення  
%     $f : R/I \to R/I, \; f_a(x+I) = ax+I$.
    
%     Нехай $f_a(x+I) = f_a(y+I)$. 
%     Тоді $(a+I)(x+I) = ax+I = ay+I = (a+I)(y+I)$.
%     $(a+I)$ - не дільник ноля, отже можна скоротити.
%     $x+I = y+I$.

%     Таким чином, $f_a$ - ін'єктивне. 
%     А тоді зі скіченності $R/I$ маємо, що $f_a - бієкція$.
%     Тоді $\forall a+I \in R/I, a+I \ne I: \exists (a+I)^{-1}$, 
%     тобто $R/I$ - поле.
% \end{proof}

Відомо, що скінченні області цілісності є полями. 
Тоді $R/I$ буде полем.

Тобто $I$ - максимальний ідеал.






\end{document}

