\documentclass[11pt, a4paper]{article} % use larger type; default would be 10pt

\usepackage{fontspec} % Font selection for XeLaTeX; see fontspec.pdf for documentation
\defaultfontfeatures{Mapping=tex-text} % to support TeX conventions like ``---''
\usepackage{xunicode} % Unicode support for LaTeX character names (accents, European chars, etc)
\usepackage{xltxtra} % Extra customizations for XeLaTeX
\usepackage{tikz}
\usetikzlibrary{arrows,calc,patterns}


% other LaTeX packages.....
\usepackage{fullpage}
\usepackage[top=2cm, bottom=4.5cm, left=2.5cm, right=2.5cm]{geometry}
\usepackage{amsmath,amsthm,amsfonts,amssymb,amscd,systeme}
\usepackage{unicode-math}
\usepackage{cancel}
\geometry{a4paper} 
%\usepackage[parfill]{parskip} % Activate to begin paragraphs with an empty line rather than an indent
\usepackage{fancyhdr}
\usepackage{listings}
\usepackage{graphicx}
\usepackage{hyperref}
\usepackage{multicol}

% FONTS
% \setmainfont[Ligatures=TeX]{Cambria Math} % set the main body font (\textrm), assumes Charis SIL is installed
%\setsansfont{Deja Vu Sans}
% \setmonofont[Ligatures=TeX]{Fira Code}
\setmathfont[Ligatures=TeX]{NewCMMath-Regular}

\setmainfont{Cambria}
\setmonofont[Ligatures=TeX]{Roboto Mono}

\renewcommand\lstlistingname{Algorithm}
\renewcommand\lstlistlistingname{Algorithms}
\def\lstlistingautorefname{Alg.}
\lstdefinestyle{mystyle}{
    % backgroundcolor=\color{backcolour},   
    % commentstyle=\color{codegreen},
    % keywordstyle=\color{magenta},
    % numberstyle=\tiny\color{codegray},
    % stringstyle=\color{codepurple},
    basicstyle=\ttfamily\footnotesize,
    breakatwhitespace=false,         
    breaklines=true,                 
    captionpos=b,                    
    keepspaces=true,                 
    numbers=left,                    
    numbersep=5pt,                  
    showspaces=false,                
    showstringspaces=false,
    showtabs=false,                  
    tabsize=2
}
\lstset{style=mystyle}

\newcommand\course{MS1 - Algebra}
\newcommand\hwnumber{HW 4}                   % <-- homework number
\newcommand\idgroup{111-2023}                
\newcommand\idname{Mykhailo Koreshkov}  

\usepackage[framemethod=TikZ]{mdframed}
\mdfsetup{%
	backgroundcolor = black!5,
}
\mdfdefinestyle{ans}{%
    backgroundcolor = green!5,
    linecolor = green!50,
    linewidth = 1pt,
}

\pagestyle{fancyplain}
\headheight 35pt
\lhead{\idgroup \\ \idname}
\chead{\textbf{\Large \hwnumber}}
\rhead{\course \\ \today}
\lfoot{}
\cfoot{}
\rfoot{\small\thepage}
\headsep 1.5em

\linespread{1.1}

\newcommand{\R}{\mathbb{R}}
\newcommand{\N}{\mathbb{N}}
\newcommand{\Z}{\mathbb{Z}}
\newcommand{\F}{\mathbb{F}}
\DeclareMathOperator{\lcm}{lcm}
\DeclareMathOperator{\cd}{CD}
\DeclareMathOperator{\ch}{char}


\begin{document}


\section*{Ex 4.1}
\begin{mdframed}
    \[a\oplus b = a+b-1; \quad a\odot b = a+b-ab\]
    Довести, що $(\Z, \oplus, \odot)$ - кільце. Знайти нуль та одиницю.
\end{mdframed}

\begin{enumerate}
    \item Коректність операцій\\
        $a+b-1 \in \Z$, $a+b-ab \in Z$
    \item Асоціативність додавання\\
        $(a\oplus b) \oplus c = (a+b-1) \oplus c = a+b-1 + c - 1 = a+b+c-2$\\
        $a \oplus (b\oplus c) = a \oplus (b+c-1) = a+b+c-1-1 = a+b+c-2$
    \item Нуль\\
        $a \oplus (1) = a+1-1 = a$\\
        $(1) \oplus a = 1+a-1 = a$\\
        \[0_R = (1)\]
    \item Зворотні елементи відносно додавання\\
        $a\oplus (1-a) = a+1-a-1 = 0$\\
        $(1-a)\oplus a = 1-a+a-1 = 0$
    \item Комутативність додавання\\
        $a \oplus b = a+b-1 = b+a-1 = b\oplus a$
    \item Асоціативність множення\\
        $(a\odot b) \odot c = (a+b-ab)+c-(a+b-ab)c = a+b-ab+c-ac-bc+abc = a+b+c-ab-ac-bc-abc$\\
        $a\odot (b \odot c) = a + (b+c-bc) - a(b+c-bc) = a+b+c-bc-ab-ac-abc$
    \item Існування одиниці\\
        $a\odot (0) = a+0-0 = a$\\
        $(0) \odot a = 0+a-0 = a$
        \[1_R = (0)\]
    \item Дистрибутивність\\
        $(a\oplus b)\odot c = (a+b-1) + c - (a+b-1)c = a+c-ac +b+c-bc -1 = (a\odot c) \oplus (b\odot c)$\\
        $c\odot (a\oplus b) = c + (a+b-1) - c(a+b-1) = (a\oplus b)\odot c$
\end{enumerate}

Отже, $(\Z, \oplus, \odot)$ дійсно Кільце.
\[0_R = (1)\]
\[1_R = (0)\]

\newpage
\section*{Ex 4.2}
\begin{mdframed}
    $R$ is an integral domain.\\
    Довести, що $\ch R$ - просте.
\end{mdframed}

$\ch R = \begin{cases}
    \min \{k\in\N \mid k\cdot 1_R = 0_R\}, & \exists k: k\cdot 1_R = 0_R\\
    0, & \text{otherwise}
\end{cases} $

$R$ - область цілісності якщо $R$ - кільце без дільників нуля (окрім нуля).
Тобто
\[\forall a,b\in R: (ab=0) \implies (a=0) \vee (b=0)\]

\begin{proof}
    Нехай $1_R \ne 0_R$. \\
    Припустимо, що $\ch R = n$ - не просте. 
    Тобто $\exists k,l>1: n = kl$.\\
    Нехай $a = k\cdot 1_R, \quad b = l\cdot 1_R$.
    Також зверну увагу, що $k,l < n$.\\
    Оскільки $n$ - найменше число таке, що $n\cdot 1_R=0_R$, $a\ne 0_R, b\ne 0_R$.

    Розглянемо добуток $ab$
    \[ab = (k\cdot 1_R) \cdot (l \cdot 1_R) = (\underset{k}{\underbrace{1+1+\cdots+1}}) \cdot (\underset{l}{\underbrace{1+1+\cdots+1}}) = \]
    \[= \left(\underset{k}{\underbrace{(\underset{l}{\underbrace{1+1+\cdots+1}}) + (\underset{l}{\underbrace{1+1+\cdots+1}}) + \cdots + (\underset{l}{\underbrace{1+1+\cdots+1}})}}\right) = \]
    \[= \underset{kl}{\underbrace{1+1+\cdots+1}} = n\cdot 1_R = 0_R\]
    Що суперечить тому, що $R$ - область цілісності, адже $a\ne0_R, b\ne0_R$.

    Таким чином довели, що $n$ - просте.
\end{proof}

\newpage
\section*{Ex 4.3}
\begin{mdframed}
    Довести, що якщо $R$ - область цілісності, то $R[x]$ також область цілісності
\end{mdframed}

\begin{proof}
    \[\alpha,\beta \in R[x]\]
    \[\alpha = a_0 + a_1x + a_2x^2 + ...\]
    \[\beta = b_0 + b_1x + b_2x^2 + ...\]
    
    Позначимо $\deg \alpha$ найвищу степінь полінома з ненульовим коефіцієнтом або $-\infty$ якщо поліном нульовий.
    Нехай $\deg \alpha = k, \deg \beta = l$.
    Зауважу, що $a_k,b_l\ne 0$ (якщо $k,l\ne-\infty$).

    Нехай
    \[\alpha\beta = 0_{R[x]}\]
    Нехай $\deg \alpha = k \ge 0$.
    Припустимо, що $\deg \beta = l \ge 0$. Тоді $b_l \ne 0_R$.
    \[\alpha\beta = ... + a_kb_lx^{k+l} = 0\]
    \[a_kb_l = 0\]
    Що суперечить тому, що $R$ - область цілісності.
    Таким чином $\alpha \ne 0_{R[x]} \implies \beta = 0_{R[x]}$.

    Аналогічно можна довести що $\beta \ne 0 \implies \alpha = 0$.

    Отже, довели, що $R[x]$ - область цілісності
\end{proof}

\section*{Ex 4.4}
\begin{mdframed}
Нехай $\exists 0_R^{-1}$. Довести, що $R = \{0\}$.   
\end{mdframed}
\begin{proof}
    Нехай $z = 0_R^{-1}$. $z0_R = 0_Rz = 1_R$.\\
    Відомо, що $\forall a\in R: a\cdot 0_R = 0_R$

    Візьмемо довільний $a\in R$
    \[a = a \cdot 1_R = az0_R = (az)\cdot 0_R = 0_R\]

    Тобто $\forall a\in R: a=0_R$.
    Кільце - тривіальне.
\end{proof}

\newpage
\section*{Ex 4.5}
\begin{mdframed}
    \[R = (\R^+, \oplus, \odot)\]
    \[\R^+ = \{x\in\R: x>0\}\]
    \[a \oplus b = ab, \qquad a \odot b = x^{\ln y}\]
\end{mdframed}
\subsection*{a)}
\begin{mdframed}
    Довести що $R$ - кільце. Показати одиницю та нуль.
\end{mdframed}
\begin{enumerate}
    \item Коректність операцій\\
    $ab \in \R^+$, $x^{\ln y} \in R^+$
    \item Асоціативність: \\
    $(a \oplus b) \oplus c = abc = a \oplus (b\oplus c)$
    \item Нуль: \\
    $1 \oplus a = 1\cdot a = a = a \cdot 1 = a \oplus 1$\\
    \begin{mdframed}[backgroundcolor=yellow!20]
        \[0_R = 1\]
    \end{mdframed}
    \item Обернені відносно додавання:\\
    $a \oplus (\frac{1}{a}) = 1$
    \item Комутативність додавання:\\
    $a \oplus b = ab = ba = b \oplus a$
    \item Асоціативність множення:\\
    $(a \odot b) \odot c = (a^{\ln b})^{\ln c} = a^{\ln b \cdot \ln c}$\\
    $a \odot (b\odot c) = a^{\ln \left(b^{\ln c}\right)} = a^{\ln \left(e^{\ln b \cdot \ln c}\right)} = a^{\ln b \cdot \ln c}$
    \item Одиниця:\\
    $a \odot e = a^{\ln e} = a^1 = a$\\
    $e \odot a = e^{\ln a} = a$\\
    \begin{mdframed}[backgroundcolor=yellow!20]
        \[1_R = e\]
    \end{mdframed}
    \item Дистрибутивність:\\
    $a\odot (b\oplus c) = a^{\ln bc} = a^{\ln b + \ln c} = a^{\ln b}\cdot a^{\ln c} = (a\odot b) \oplus (a\odot c)$\\
    $(a\oplus b)\odot c = (ab)^{\ln c} = a^{\ln c} \cdot b^{\ln c} = (a\odot c) \oplus (b\odot c)$
\end{enumerate}

\subsection*{б) Комутативність кільця}
\[a \odot b = a^{\ln b} = \left(e^{\ln a}\right)^{\ln b} = e^{\ln a \cdot \ln b} = \left(e^{\ln b}\right)^{\ln a} = b^{\ln a} = b \odot a\]
Отже, кільце є комутативним.

\subsection*{в) Дільники нуля}
Припустимо, що $ab = 0_R = 1$.
\begin{gather*}
    a\odot b = e^{\ln a \cdot \ln b} = 0_R = 1 = e^{0}\\
    \ln a \cdot \ln b = 0\\
    \ln a = 0 \vee \ln b = 0\\
    a = 1 \vee b = 1
\end{gather*}

Тобто, $ab=0_R \implies a=0_R \vee b=0_R$. Отже, кільце без дільників нуля.

\subsection*{г) Оборотні елементи}
Покажу, що існують обернені елементи:
\begin{gather*}
    a\odot b = e^{\ln a \cdot \ln b} = e^{1} = e\\
    \ln a \cdot \ln b = 1\\
    \ln b = \frac{1}{\ln a}\\
    b = e^{\frac{1}{\ln a}} > 0\\
    \ln a \in \R; \quad (\ln a)^{-1} \in R; \quad e^{\R} = \R^+
\end{gather*}
Єдина особлива точка - $\ln a = 0$ або $a = 1 = 0_R$.
Отже, 
\[\forall a\in R, a\ne 1=0_R: \exists a^{-1} = e^{\frac{1}{\ln a}}\]
Тобто
\[R^* = R\setminus \{0_R\}\]

\section*{Ex 4.6}
\begin{mdframed}
    \[\F_2 = \Z/2\Z\]
    \[\text{Let } \F_4 = \left\{\begin{pmatrix}
        a & b \\ b & a+b
    \end{pmatrix}\in M_2(\F_2)\right\}\]
\end{mdframed}

\subsection*{а) Довести що $\F_4$ - кільце.}
Нехай 
\begin{gather*}
    A,B,C \in \F_4, \quad 
    A = \begin{pmatrix}
        a_1 & a_2\\ a_2 & a_1 + a_2
    \end{pmatrix}, \quad
    B = \begin{pmatrix}
        b_1 & b_2 \\ b_2 & b_1 + b_2
    \end{pmatrix}
\end{gather*}, $C$ аналогічно.

Зауважу, що для $a,b\in\F_2$ виконуються наступні властивості:
\begin{gather*}
    ab = ba,\qquad a+a = 0
\end{gather*}

\begin{enumerate}
    \item Коректність операцій\\
    \[C = A+B = \begin{pmatrix}
        a_1 + b_1 & a_2 + b_2 \\ a_2 + b_2 & (a_1 + b_1) + (a_2 + b_2)
    \end{pmatrix} \in R\]
    $c_{12} = c_{21}, c_{22} = c_{11}+c_{12}$.
    Додавання коректне.
    \[C = AB = 
    \begin{pmatrix}
        a_1 & a_2 \\ a_2 & a_1 + a_2
    \end{pmatrix} 
    \begin{pmatrix}
        b_1 & b_2\\ b_2 & b_1 + b_2
    \end{pmatrix} = 
    \begin{pmatrix}
        a_1b_1 + a_2b_2 & a_1b_2 + a_2b_1 + a_2b_2 \\ b_1a_2 + a_1b_2 + a_2b_2 & a_2b_2 + a_1b_1 + a_1b_2 + a_2b_1 + a_2b_2 
    \end{pmatrix}\]
    $c_{12} = c{21}$.
    \[c_{22} = \cancel{a_2b_2} + a_1b_1 + \cancel{a_1b_2} + \cancel{a_2b_1} + \cancel{b_2b_2}  = a_1b_1 \]
    \[c_{11}+c_{12} = a_1b_1+ \cancel{a_2b_2}+\cancel{a_1b_2} + \cancel{a_2b_1} + \cancel{a_2b_2} = a_1b_1 = c_{22}\]
    Множення коректне.

    \item Асоціативність: наслідується з властивостей матричних операцій та $\F_2$
    \item Нуль: наслідується з властивостей матричних операцій та $\F_2$
    \[0_R = \begin{pmatrix}
        0 & 0 \\ 0 & 0
    \end{pmatrix} \in R\]
    \item Обернені відносно додавання:\\
    \[A + A = \begin{pmatrix}
        a_1 + a_1 & a_2 + a_2 \\ a_2 + a_2 & a_1 + a_1 + a_2 + a_2
    \end{pmatrix} = 0_R\] з властивості $\forall a\in\F_2: a+a=0$
    \item Комутативність додавання: наслідується з властивостей матричних операцій та $\F_2$
    \item Асоціативність множення: наслідується з властивостей матричних операцій та $\F_2$
    \item Одиниця: наслідується з властивостей матричних операцій та $\F_2$
    \[A \cdot \begin{pmatrix}
        1 & 0 \\ 0 & 1
    \end{pmatrix} = A = \begin{pmatrix}
        1 & 0 \\ 0 & 1
    \end{pmatrix} \cdot A;\qquad
    1_R = \begin{pmatrix}
        1 & 0 \\ 0 & 1
    \end{pmatrix} \in R\]
    \item Дистрибутивність: наслідується з властивостей матричних операцій та $\F_2$.
\end{enumerate}

\subsection*{б) Довести що $\F_4$ - це поле, що складається з 4 елементів}
Зауважу, що $\F_2 = \{0,1\}$.
Елемент $\F_4$ залежить від двох значень - $a,b$. Кожне приймає значення $0,1$. Переберу всі можливі комбінації $a,b$.
Позначатиму $f(a,b) = \begin{pmatrix}
    a & b \\ b & a+b
\end{pmatrix}, \quad a,b\in\F_2$
\begin{gather*}
    f(0,0) = \begin{pmatrix}
        0 & 0 \\ 0 & 0
    \end{pmatrix} = 0_R,
    \qquad
    f(1,0) = \begin{pmatrix}
        1 & 0 \\ 0 & 1
    \end{pmatrix} = 1_R,\\
    f(0,1) = \begin{pmatrix}
        0 & 1 \\ 1 & 1
    \end{pmatrix} = \alpha,
    \qquad
    f(1,1) = \begin{pmatrix}
        1 & 1 \\ 1 & 0
    \end{pmatrix} = \beta
\end{gather*}

Поле - це комутативне кільце з $R^* = R\setminus \{0_R\}$.

Комутативність:
\[AB = \begin{pmatrix}
    a_1b_1 + a_2b_2 & a_1b_2 + a_2b_1 + a_2b_2 \\ b_1a_2 + a_1b_2 + a_2b_2 & a_2b_2 + a_1b_1 + a_1b_2 + a_2b_1 + b_2b_2 
\end{pmatrix} = \begin{pmatrix}
    a_1b_1 + a_2b_2 & a_2b_2 \\ a_2b_2 & a_1b_1
\end{pmatrix}\]
\[BA = \begin{pmatrix}
    b_1a_1 + b_2a_2 & b_2a_2 \\ b_2a_2 & b_1a_1
\end{pmatrix} = AB\]

Обернені елементи:
\[\alpha\beta = \begin{pmatrix}
    0 & 1 \\ 1 & 1
\end{pmatrix}\begin{pmatrix}
    1 & 1 \\ 1 & 0
\end{pmatrix} = \begin{pmatrix}
    0\cdot 1 + 1\cdot 1 & 0\cdot 1 + 1\cdot 0 \\
    1\cdot 1 + 1\cdot 1 & 1\cdot 1 + 0\cdot 1
\end{pmatrix} = \begin{pmatrix}
    1 & 0 \\ 0 & 1
\end{pmatrix} = 1_R\]
Отже, $\alpha^{-1} = \beta$.

\[R^* = \{1_R, \alpha, \beta\} = R\setminus \{0_R\}\]

Таким чином, $\F_4$ - дійсно поле з 4 елементів.

\section*{Ex 4.7}
\begin{mdframed}
    $\alpha \in \mathbb H, \quad \alpha = a+bi+cj+dk \ne 0, \qquad \alpha^{-1} = ?$
\end{mdframed}

Зауважу, що $0_H = 0 + 0i + 0j + 0k$, $1_H = 1 + 0i + 0j + 0k$.

% Введу нове позначення
% \[\alpha = a+bi+cj+dk = (r, \vec v); \quad r=a; \; \vec v = (b,c,d)\]
% Тоді 
% \[\alpha + \beta = (r+r', \vec v+\vec v')\]
% \[\alpha\beta = (rr'-\vec v\cdot \vec v', r\vec v' + r'\vec v + \vec v \times \vec v')\]
% Ці тотожності легко доводяться використовуючи Дистрибутивність та таблицю множення $e,i,j,k$ між собою.

% \begin{gather*}
%     \alpha\beta = (rr'-\vec v\cdot \vec v', r\vec v' + r'\vec v + \vec v \times \vec v') = 1\\
%     rr'-\vec v\cdot \vec v' = 1, \qquad r\vec v' + r'\vec v + \vec v \times \vec v' = \vec 0\\
%     r = \frac{1+v\cdot v'}{r'}\\
%     \frac{1+v\cdot v'}{r'}\cdot v' + r'v + v\times v' = 0\\
%     (1+v\cdot v')\cdot v' + (r')^2v + r'(v\times v') = 0\\
%     \text{Let } u = v\times v', \quad p = 1+v\cdot v';\\
%     r' = \frac{}{}
% \end{gather*}

\[\alpha = a_1+b_1i+c_1j+d_1k\]
\[\beta = a_2+b_2i+c_2j+d_2k;\]
\[\alpha\beta = 
\begin{matrix}
    {}&a_{1}a_{2}&-b_{1}b_{2}&-c_{1}c_{2}&-d_{1}d_{2}\\
    {}+{}(&a_{1}b_{2}&+b_{1}a_{2}&+c_{1}d_{2}&-d_{1}c_{2})\mathbf{i} \\
    {}+{}(&a_{1}c_{2}&-b_{1}d_{2}&+c_{1}a_{2}&+d_{1}b_{2})\mathbf{j} \\
    {}+{}(&a_{1}d_{2}&+b_{1}c_{2}&-c_{1}b_{2}&+d_{1}a_{2})\mathbf{k} 
\end{matrix}
\]

% Нехай $\alpha\beta=1$.
% Тоді
% \begin{gather*}
%     \begin{cases}
%         a_{1}a_{2}-b_{1}b_{2}-c_{1}c_{2}-d_{1}d_{2} = 1\\
%         a_{1}b_{2}+b_{1}a_{2}+c_{1}d_{2}-d_{1}c_{2} = 0\\
%         a_{1}c_{2}-b_{1}d_{2}+c_{1}a_{2}+d_{1}b_{2} = 0\\
%         a_{1}d_{2}+b_{1}c_{2}-c_{1}b_{2}+d_{1}a_{2} = 0
%     \end{cases}
% \end{gather*}

В комплексних числах $z^{-1} = \frac{\bar z}{|z|^2}$. \\
Спробую побудувати тут зворотній елемент за аналогією.

Нехай $\beta = \frac{a-bi-cj-dk}{a^2+b^2+c^2+d^2}$. Обчислимо добуток $\alpha\beta = \gamma = a'+b'i+c'j+d'k$
\begin{gather*}
    a' = a_{1}a_{2}-b_{1}b_{2}-c_{1}c_{2}-d_{1}d_{2} = \frac{1}{a^2+b^2+c^2+d^2}\left(a^2+b^2+c^2+d^2\right) = 1\\
    b' = a_{1}b_{2}+b_{1}a_{2}+c_{1}d_{2}-d_{1}c_{2} = \frac{1}{a^2+b^2+c^2+d^2}\left(-ab+ba-cd-(-dc)\right) = 0\\
    c' = a_{1}c_{2}-b_{1}d_{2}+c_{1}a_{2}+d_{1}b_{2} = \frac{1}{a^2+b^2+c^2+d^2}\left(-ac-(-bd)+ca-db\right) = 0\\
    d' = a_{1}d_{2}+b_{1}c_{2}-c_{1}b_{2}+d_{1}a_{2} = \frac{1}{a^2+b^2+c^2+d^2}\left(-ad-bc-(-cb)+da\right) = 0\\
    \alpha\beta = 1+0i+0j+0k = 1_H
\end{gather*}

Отже, обернений елемент для $\alpha$ це 
\[\alpha^{-1} = \frac{\alpha^*}{|\alpha|^2},\]
де 
\[\alpha^* = a-bi-cj-dk, \qquad |\alpha| = a^2+b^2+c^2+d^2\] 

\end{document}

