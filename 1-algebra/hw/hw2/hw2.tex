\documentclass[11pt, a4paper]{article} % use larger type; default would be 10pt

\usepackage{fontspec} % Font selection for XeLaTeX; see fontspec.pdf for documentation
\defaultfontfeatures{Mapping=tex-text} % to support TeX conventions like ``---''
\usepackage{xunicode} % Unicode support for LaTeX character names (accents, European chars, etc)
\usepackage{xltxtra} % Extra customizations for XeLaTeX
\usepackage{tikz}
\usetikzlibrary{arrows,calc,patterns}


% other LaTeX packages.....
\usepackage{fullpage}
\usepackage[top=2cm, bottom=4.5cm, left=2.5cm, right=2.5cm]{geometry}
\usepackage{amsmath,amsthm,amsfonts,amssymb,amscd,systeme}
\usepackage{unicode-math}
\usepackage{cancel}
\geometry{a4paper} 
%\usepackage[parfill]{parskip} % Activate to begin paragraphs with an empty line rather than an indent
\usepackage{fancyhdr}
\usepackage{listings}
\usepackage{graphicx}
\usepackage{hyperref}
\usepackage{multicol}

% FONTS
% \setmainfont[Ligatures=TeX]{Cambria Math} % set the main body font (\textrm), assumes Charis SIL is installed
%\setsansfont{Deja Vu Sans}
% \setmonofont[Ligatures=TeX]{Fira Code}
\setmathfont[Ligatures=TeX]{NewCMMath-Regular}

\setmainfont{Cambria}
\setmonofont[Ligatures=TeX]{Roboto Mono}

\renewcommand\lstlistingname{Algorithm}
\renewcommand\lstlistlistingname{Algorithms}
\def\lstlistingautorefname{Alg.}
\lstdefinestyle{mystyle}{
    % backgroundcolor=\color{backcolour},   
    % commentstyle=\color{codegreen},
    % keywordstyle=\color{magenta},
    % numberstyle=\tiny\color{codegray},
    % stringstyle=\color{codepurple},
    basicstyle=\ttfamily\footnotesize,
    breakatwhitespace=false,         
    breaklines=true,                 
    captionpos=b,                    
    keepspaces=true,                 
    numbers=left,                    
    numbersep=5pt,                  
    showspaces=false,                
    showstringspaces=false,
    showtabs=false,                  
    tabsize=2
}
\lstset{style=mystyle}

\newcommand\course{MS1 - Algebra}
\newcommand\hwnumber{HW 2}                   % <-- homework number
\newcommand\idgroup{111-2023}                
\newcommand\idname{Mykhailo Koreshkov}  

\usepackage[framemethod=TikZ]{mdframed}
\mdfsetup{%
	backgroundcolor = black!5,
}
\mdfdefinestyle{ans}{%
    backgroundcolor = green!5,
    linecolor = green!50,
    linewidth = 1pt,
}

\pagestyle{fancyplain}
\headheight 35pt
\lhead{\idgroup \\ \idname}
\chead{\textbf{\Large \hwnumber}}
\rhead{\course \\ \today}
\lfoot{}
\cfoot{}
\rfoot{\small\thepage}
\headsep 1.5em

\linespread{1.2}

\newcommand{\R}{\mathbb{R}}
\newcommand{\N}{\mathbb{N}}
\newcommand{\Z}{\mathbb{Z}}
\DeclareMathOperator{\lcm}{lcm}
\DeclareMathOperator{\cd}{CD}


\begin{document}


\section*{Ex 2.1}
\begin{mdframed}
    Let $a,b \in \Z,\; a,b>0, \; \gcd(a,b)=1$.
    Довести, що
    $$\left(\exists m\in\N: ab=m^2\right) \implies \exists s,t\in\N: a=s^2 \land b=t^2$$
\end{mdframed}

\begin{proof}
    Одразу використаю основну теорему арифметики. ().
    \begin{gather*}
        ab = m^2\\
        a = \prod_{p\in\mathcal{P}} p^{\alpha_p}\\
        b = \prod_{p\in\mathcal{P}} p^{\beta_p}\\
        m = \prod_{p\in\mathcal{P}} p^{\mu_p}\\
        m^2 = \prod_{p\in\mathcal{P}} p^{2\mu_p}\\
        \prod_{p\in\mathcal{P}} p^{\alpha_p+\beta_p} = \prod_{p\in\mathcal{P}} p^{2\mu_p}\\
        \alpha_p+\beta_p = 2\mu_p\\
    \end{gather*}
    Тепер використаю той факт, що $\gcd(a,b)=1$.
    З цього слідуватиме що $\forall p: \alpha_p \beta_p = 0$, тобто ненульовим є лише один з степенів незвідного множника.

    Отож $\alpha_p+\beta_p = 2\mu_p$ значить що $\alpha_p = 2\mu_p, \beta_p = 0$ або $\beta_p = 2\mu_p, \alpha_p=0$.\\
    Таким чином $\forall p: 2\mid \alpha_p \land 2\mid \beta_p$.\\
    Нехай $s = \prod_p p^{\alpha_p/2}, \; t = \prod_p p^{\beta_p/2}$.\\
    Тоді $a = s^2, b = t^2$.
\end{proof}

\section*{Ex 2.2}
\begin{mdframed}
    Нехай $a,b\in\Z, \; \gcd(a,b) = 1$.\\
    Нехай $\exists m\in\Z: ab=m^2$.
    Довести що $\exists s,t\in\Z: a=s^2, b=t^2$.
\end{mdframed}

Різниця з 2.1 у тому, що тут a та b можуть бути 0 та від'ємними.
Тут $\mathcal{P} = \{2,3,5,...\}$ - множина \textbf{додатніх} простих чисел.

Нехай 
\begin{gather*}
    a = \pm\prod_{p\in\mathcal{P}} p^{\alpha_p}\\
    b = \pm\prod_{p\in\mathcal{P}} p^{\beta_p}\\
    m^2 = +\prod_{p\in\mathcal{P}} p^{2\mu_p}
\end{gather*}

Тоді з 2.1 маємо $\alpha_p = 2\mu_p, \beta_p = 0$ або $\beta_p = 2\mu_p, \alpha_p=0$ для $p\ne 1$.\\
Нехай $s = \prod_p p^{\alpha_p/2}, \; t = \prod_p p^{\beta_p/2}$.

\begin{gather*}
    a = (-1)^\alpha_1 \cdot s
    b = (-1)^\beta_1 \cdot t
    ab = (-1)^{\alpha_1+\beta_1} \cdot st = m^2
\end{gather*}

Очевидно $\alpha_1+\beta_1 \equiv 0 \mod 2$.
Тобто $a$ та $b$ матимуть один знак.

Припустимо, що $a<0, b<0$.

Або я щось не так зрозумів, або задача практично еквівалентна 2.1. 
Але якщо розглядати від'ємні випадки, то я сумніваюся, що від'ємні числа можна представити як повний квадрат.

\begin{gather*}
    a = -16, \quad b = -25\\
    ab = +400 = 20^2\\
    n = 20
\end{gather*}
Але $a$ та $b$ можна представити лише як $-n^2$.

\section*{Ex 2.3}
\begin{mdframed}
    Нехай $p$ - просте. Нехай $a,n\in\Z, n\ge 0$. 
    Довести
    $$p\mid a^n \implies p^n\mid a^n$$
\end{mdframed}

\begin{proof}
    Перший випадок $n=0$.
    \[p\mid 1 \implies p = 1 \implies 1 \mid 1\]

    Другий випадок $n=1$.
    \[p\mid a \implies p \mid a\]

    Нетривіальні випадки $n>1$.
    Розкладаємо $a$:
    \[a = \pm \prod_{q\in\mathcal{P}}q^{a_q}, \quad a_q\in\N\]
    Тоді
    \[a^n = \pm \prod_{q\in\mathcal{P}}q^{n \cdot a_q}\]

    Оскільки $p\mid a^n$, то $(n\cdot a_p) > 0$ як степінь при факторі $p$ у розкладі $a^n$.
    Ми прийняли $n>1$, отож маємо $a_p > 0$.
    А звідси $na_p \ge n$, з чого слідує $a^n = p^{n} \cdot p^{n(a_p-1)} \cdot \prod_{q\in\mathcal{P}, q\ne p}q^{na_q}$.
    Отже
    \[p^n \mid a^n\]
\end{proof}

\section*{Ex 2.4}
Тут і далі позначатиму $(a\equiv b \mod n)$ як $a\equiv_n b$.

\begin{mdframed}
    Нехай $a,b\in\Z$. Нехай $P=\{2,3,5,7,11,...\}$ - множина незвідних цілих чисел.
    Довести:
    \[\left(\forall q\in P: a \equiv_q b\right) \implies a=b\]
\end{mdframed}

\begin{proof}
    Нехай $g = \gcd(a,b)$.
    Тоді 
    \begin{gather*}
        a = gx\\
        b = gy\\
        \gcd(x,y) = 1
    \end{gather*}
    Далі
    \begin{gather*}
        \forall q\in P: a-b = qk \\
        gx-gy = g(x-y) = qk \\
        \forall q\in P: q \mid g(x-y) \\
    \end{gather*}

    Єдине таке число, яке ділиться на всі прості числа, це $0$.
    Тобто $g(x-y) = 0$
    \[
    \left[\begin{matrix}
        g = 0 &\implies& a=b=0\\
        (x-y)=0 &\implies& x=y \implies a=b
    \end{matrix}\right.
    \]

    Отже, $a=b$.
\end{proof}

\section*{Ex 2.5}
\begin{mdframed}
    Нехай $q>5$, $q$ - незвідне.
    Довести що
    \[[q]_6 = [1]_6 \vee [q]_6 = [5]_6\]     
\end{mdframed}

\begin{proof}
    Що означає твердження?

    По-перше зазначу, що число не може бути в двох різних класах еквівалентності за одним відношенням еквівалентності (для одного $n$).
    \begin{gather*}
    [q]_6 = [1]_6 \vee [q]_6 = [5]_6\\
    q \equiv_6 1 \vee q \equiv_6 = 5\\
    6 \mid (q-1) \vee 6 \mid (q-5)
    \end{gather*}

    Розглянемо всі можливі залишки при діленні $q$ на 6.
    \begin{gather*}
        q \mod 6 = 0 \implies 6 \mid q \implies q \notin \mathcal{P}\\ 
        q \mod 6 = 2 \implies q = 6k+2 = 2(3k+1) \implies 2 \mid q \implies q \notin \mathcal{P}\\
        q \mod 6 = 3 \implies q = 6k+3 = 3(2k+1) \implies 3 \mid q \implies q \notin \mathcal{P}\\
        q \mod 6 = 4 \implies q = 6k+4 = 2(3k+2) \implies 2 \mid q \implies q \notin \mathcal{P}\\
    \end{gather*}

    Для інших залишків продемонструю прикладами їх можливість
    \begin{gather*}
        q = 11 \implies q \equiv_6 5\\
        q = 13 \implies q \equiv_6 1
    \end{gather*}
\end{proof}

\section*{Ex 2.6}
\begin{mdframed}
    Перевірити твердження
    \[\left([a]_n \ne [0]_n \land [a]_n \cdot [b]_n=[a]_c \cdot [c]_n\right) \implies [b]_n = [c]_n\]
\end{mdframed}

Що це все означає?
\begin{gather*}
    a \mod n \ne 0\\
    ab \mod n = ac \mod n\\
\end{gather*}
тобто
\begin{gather*}
    n \nmid a\\
    n \mid (ab-ac)\\
    n \mid a(b-c)
\end{gather*}

Якщо $n$ - просте, то дійсно $n \mid (b-c)$, бо $n \nmid a$.
Але якщо $n$ - складене, то може не виконуватись.

Візьмемо
\begin{gather*}
    n = 6 = 3\cdot 2\\
    a = 2\\
    b = 7\\
    c = 4\\
\end{gather*}
Тоді
\begin{gather*}
    2 \mod 6 = 2 \ne 0\\
    ab = 14 \equiv_6 2\\
    ac = 8 \equiv_6 2\\
    b \mod 6 = 1
    c \mod 6 = 4 \ne 1
\end{gather*}

Знайшли контрприклад. 

\section*{Ex 2.7}
\begin{mdframed}
    Довести:
    \[[a]_n=[1]_n \implies \gcd(a,n)=1\]
    Але не навпаки
\end{mdframed}

\begin{proof}
    \[[a]_n = [1]_n\]
    Тобто
    \[n \mid (a-1) \]
    Або інакше
    \[\exists k\in\Z: a = kn + 1\]
    Тобто
    \[a = kn + 1\]
    Або
    \[1a-kn = 1\]
    Що означає
    \[\gcd(a,n) = 1\]
\end{proof}

Знайдемо контрприклад до оберненого твердження

\begin{gather*}
    a = 6, n = 7\\
    \gcd(a,n) = 1\\
    a \mod n = 6 \mod 7 = 6\\
    1 \mod n = 1 \mod 7 = 1
\end{gather*}
Тобто $[a]_n \ne [1]_n$

\end{document}

