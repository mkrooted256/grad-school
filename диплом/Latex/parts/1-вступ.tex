% !TeX root = ..\dyplom-template.tex
У задачах сегментації замість значень пікселів зображення в основному розлядаються деякі скалярні чи векторні похідні величини,
обчислені для всього зображення, його перетинних чи неперетинних частин, чи у кожному пікселі \cite{belsare2015, simon2018, esteva2017}. 
Такі похідні величини називатимемо ознаками (англ. features) або векторами ознак.
В процесі обробки зображення мається на увазі деяка модель зображення, яка описує локальну та глобальну поведінку пікселів, їх взаємовідношення, статистичні властивості тощо.
Ознаки, породжені моделями, які здатні розрізняти зображення із різними текстурами, називатимемо \emph{текстурними дескрипторами}.
Проте, поняття текстури є переважно інтуїтивним і різні його визначення відповідають різним моделям, що описують зображення.  
Текстурні дескриптори доречні у задачах пошуку нерегулярностей, відхилень від звичної поведінки в околах пікселів, 
або для розрізнення внутрішньо однорідних, проте якісно різних між собою областей зображення. 
Такі задачі виникають у гістопаталогії \cite{simon2018}, \todo{ще приклади}.

У цій роботі переважно досліджується текстурний дескриптор Local Binary Pattern (LBP, локальний бінарний шаблон) \cite{ojala2002} та його модифікації.
Розглядаються математична модель текстури згідно цього дескриптора, проблеми вибору оптимального радіуса й дискретизації LBP, 
та можливі модифікації вектору ознак LBP для зменшення його розмірності та покращення подальшої обробки.

Основним об'єктом наших досліджень у лабораторії системної біології Інституту молекулярної біології та генетики (ІМБГ) були 
фотографії мікропрепаратів біологічних тканин із особливо великою роздільною здатністю (сотні МП), зняті на цифровий мікроскоп.
Задачею було дослідити застосовність дескрипторів LBP до сегментування зображення у напів-автоматичному режимі без попереднього навчання моделі.
У процесі ми приділили багато часу пошукам оптимальних гіперпараметрів моделі, якими, в тому числі, були параметри радіуса та дискретизації дескрипторів LBP.
Багато результатів отримано для досить великих векторів ознак (від 300 до декількох тисяч float64), що вимагало великих витрат часу та потужних машин для обчислення.
Однією з цілей дослідження є знайти спосіб зменшити кількість ознак не погіршивши якість сегментації зображень.

Проміжні результати практичної частини досліджень були представлені на конференції ECCB 2024 \cite{fastlbp2024}.
