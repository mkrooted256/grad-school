В роботі описана побудова теорії моделювання текстур через розподіли текстурних дескрипторів,
а також проаналізовано сам дескриптор, його варіанти, і припущення моделі щодо даних.
Коректність теоретичного формулювання та сформульовані припущення підтверджено 
у практичній задачі класифікації текстур.

Продемонстровано, що текстуру доцільно моделювати як розподіл значень дескрипторів LBP на зображенні.
Показано, що утворена з однієї реалізації текстури модель є достатньо загальною, 
щоб описувати і інші реалізації текстури, проте лише схожих масштабів; 
розподіли дескрипторів суттєво відрізняються для зображень різних масштабів.
Наведено підстави вважати, що "рівномірні" LBP дескриптори дійсно описують текстуру достатньо повно: 
за умови правильно підібраного класифікатора, короткі вектори ознак від рівномірних дескрипторів можуть 
давати точність класифікації співставну із значно довшими ознаками від стандартних дескрипторів. 

Код практичної частини і статистичних досліджень доступний у відкритому репозиторії \url{https://github.com/mkrooted256/master-thesis}.
Код пакету fastLBP доступний у відкритому репозиторії \url{https://github.com/imbg-ua/fastLBP}.
