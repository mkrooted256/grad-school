\documentclass{thesis}

\renewcommand{\baselinestretch}{1.20}


\begin{document}
\allowdisplaybreaks

\renewcommand{\chaptername}{Chapter}
\renewcommand{\contentsname}{Contents}
\renewcommand{\partname}{Part}
\renewcommand{\chaptername}{Chapter}
\renewcommand{\listfigurename}{List of figures}
\renewcommand{\listtablename}{List of tables}
\renewcommand{\bibname}{References}
\renewcommand{\indexname}{Index}
\renewcommand{\tablename}{Table}
\renewcommand{\abstractname}{Abstract}
\renewcommand{\figurename}{Figure}
\renewcommand{\appendixname}{Appendix}
\renewcommand{\proofname}{Proof}

\large

\setcounter{page}{1}
\thispagestyle{empty}
\centerline{Національна академія наук України}
\centerline{Міністерство освіти і науки України}
\centerline{Державна наукова установа «Київський академічний університет»}

\vspace{10mm}

\begin{flushright}
\begin{minipage}{100mm}
\begin{center}\large {\bf <<Допущено до захисту>>}\\
Завідувач кафедри математики,\\
доктор фіз.-мат. наук\\
{\bf Вячеслав БОЙКО}\\
<<\underline{\hspace{8mm}}>> травня 2023 р.
\end{center}
\end{minipage}
\end{flushright}

\vspace{10mm}

\centerline{\Large \bf Прізвище Ім'я}

\begin{center}
{\bf КВАЛІФІКАЦІЙНА РОБОТА}\\
на здобуття освітнього ступеня <<магістр>>\\
Спеціальність 111 <<Математика>>\\[4mm]
{\Large \bf Тема: <<AAAAAAA>>}
\end{center}


\vspace{5mm}

\noindent
{Засвідчую, що кваліфікаційна робота містить результати власних досліджень. Використання ідей, результатів і~текстів інших авторів мають посилання на відповідне джерело.
\underline{\hspace{18mm}} І.Б.~ПРІЗВИЩЕ \par}

\vspace{5mm}

\begin{flushright}
\begin{minipage}{90mm}
\large {\bf Науковий керівник}\\
доктор фіз.-мат. наук, професор\\
{\bf ?????????}\\
\underline{\hspace{48mm}}

\end{minipage}
\end{flushright}


\vfill

\centerline{\bf Київ --- 2023}

\newpage


\begin{center}
\Large \bf Анотація
\end{center}

\noindent
\textbf{Прізвище І.Б.}, \textbf{??????}, Кваліфікаційна робота на здобуття освітнього ступеня <<магістр>> за спеціальністю 111 Математика, Київський академічний університет, Київ, 2023, ??~с., ??~джерел.

\bigskip


???Текст анотації???

\bigskip

\noindent
{\bf MSC:} ???????

\bigskip

\noindent
{\bf Ключові слова:} ???????
\bigskip

\newpage

\begin{center}
\Large \bf Abstract
\end{center}

\noindent
\textbf{LastName N.S.}, \textbf{?????????}, Master Thesis, speciality 111 Mathematics.~--
Kyiv Academic University, Kyiv, 2023, ??~pages, ??~references.

\bigskip
??????

\bigskip

\noindent
{\bf MCS:} ??????? %for 111, see https://mathscinet.ams.org/mathscinet/msc/msc2020.html

%\noindent
%{\bf ACM:} ??????? %for 122, see https://cran.r-project.org/web/classifications/ACM.html

\bigskip

\noindent
{\bf Key words:} ???????


\newpage


\tableofcontents

\newpage
\phantomsection
\section*{Перелік умовних позначень}
\addcontentsline{toc}{chapter}{Перелік умовних позначень}

\bigskip


\begin{tabular}{ll}
$G$ & група Лі\\
$\mathfrak g$ & алгебра Лі\\
$Q$ & оператор однопараметричної групи Лі\\[1mm]
$\mathbb R^n$ & $n$-вимірний евклідів простір\\[1mm]
$X\simeq {\mathbb R}^n$ & простір незалежних змінних~$x=(x^1,x^2,\ldots, x^n)$ \\[1mm]
$U\simeq {\mathbb R}^m$ & простір залежних змінних $u=\big(u^1, u^2, \ldots, u^m\big)$\\[1mm]
$u^\alpha_i=\dfrac{\partial u^\alpha}{\partial x^i}$ & частинна похідна від залежної змінної $u^\alpha$\\[1mm]
& за незалежною змінною $x^i$\\[1mm]
$D_{i}$ & оператор повної похідної за змінною $x^{i}$\\[1mm]
$\underset{r}{Q}$ & $r$-те продовження оператора $Q$\\[1mm]
$I$ & набір інваріантів нульового порядку\\[1mm]
$I_{(r)}$ & набір інваріантів $r$-го порядку, $r\geq 1$

\end{tabular}



\newpage

\phantomsection
\chapter*{Вступ}\label{Introduction}
\addcontentsline{toc}{chapter}{Вступ}

??????????
\cite{boyko-thesis,boyko2021,PopovychBoykoNesterenkoLutfullin2003}

\newpage

\chapter{Назва розділу}\label{chaper1}

\section{Назва секції}\label{section1.1}

\newpage

\chapter{Назва розділу}\label{chaper2}

\section{Назва секції}\label{section2.1}


\newpage

\phantomsection
\chapter*{Висновки}
\addcontentsline{toc}{chapter}{Висновки}

У роботі ??????

\newpage

%\renewcommand{\bibname}{Список використаних джерел}

\begin{thebibliography}{99}
\addcontentsline{toc}{chapter}{Список використаних джерел}
\itemsep=0pt

\bibitem{boyko-thesis}
Бойко В.М.,
Узагальненi оператори Казiмiра,
сингулярнi модулi редукцiї
та симетрiї диференцiальних рiвнянь,
Дис. \dots\ док. фіз.-мат. наук,  Інституту математики НАН України, Київ, 2018, 338~с., \url{https://www.imath.kiev.ua/~boyko/BoykoThesis.pdf}.



\bibitem{boyko2021}
Boyko V.M., Lokaziuk O.V., Popovych R.O.,
Admissible transformations and Lie symmetries of linear systems of second-order ordinary differential equations, \href{https://arxiv.org/abs/2105.05139}{arXiv:2105.05139}.

\bibitem{Maple17}
Maple 17, \url{https://www.maplesoft.com/products/Maple/}.


\bibitem{Olver1995}
Olver P.J., Equivalence, invariants, and symmetry, Cambridge, University Press Cambridge, 1995, xvi+525~pp.,
\url{https://doi.org/10.1017/CBO9780511609565}.


\bibitem{PopovychBoykoNesterenkoLutfullin2003}
Popovych R.O., Boyko V.M., Nesterenko M.O., Lutfullin M.V., Realizations of real low-dimensional Lie algebras, \textit{J.~Phys.~A} \textbf{36} (2003), no.~26,
7337--7360, \url{https://doi.org/10.1088/0305-4470/36/26/309}; \href{https://arxiv.org/abs/math-ph/0301029}{math-ph/0301029}.



\end{thebibliography}

\appendix

\chapter{Назва додатку}\label{appendix1}

\section{Назва секції додатку}\label{appendix1.1}

\end{document}

\newpage

\bibliographystyle{plain} %plain %sigma %amsalpha %ugost2008
\bibliography{ref}



