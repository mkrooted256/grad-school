\documentclass[11pt, a4paper]{article} % use larger type; default would be 10pt

\usepackage{fontspec} % Font selection for XeLaTeX; see fontspec.pdf for documentation
\defaultfontfeatures{Mapping=tex-text} % to support TeX conventions like ``---''
\usepackage{xunicode} % Unicode support for LaTeX character names (accents, European chars, etc)
\usepackage{xltxtra} % Extra customizations for XeLaTeX
\usepackage{tikz}
\usetikzlibrary{arrows,calc,patterns}


% other LaTeX packages.....
\usepackage{fullpage}
\usepackage[top=2cm, bottom=4.5cm, left=2.5cm, right=2.5cm]{geometry}
\usepackage{amsmath,amsthm,amsfonts,amssymb,amscd,systeme}
\usepackage{unicode-math}
\usepackage{cancel}
\geometry{a4paper} 
%\usepackage[parfill]{parskip} % Activate to begin paragraphs with an empty line rather than an indent
\usepackage{fancyhdr}
\usepackage{listings}
\usepackage{graphicx}
\usepackage{hyperref}
\usepackage{multicol}

% FONTS
\setmainfont[Ligatures=TeX]{Cambria Math} % set the main body font (\textrm), assumes Charis SIL is installed
%\setsansfont{Deja Vu Sans}
\setmonofont[Ligatures=TeX]{Fira Code}
\setmathfont[Ligatures=TeX]{NewCMMath-Regular.otf}

\renewcommand\lstlistingname{Algorithm}
\renewcommand\lstlistlistingname{Algorithms}
\def\lstlistingautorefname{Alg.}
\lstdefinestyle{mystyle}{
    % backgroundcolor=\color{backcolour},   
    % commentstyle=\color{codegreen},
    % keywordstyle=\color{magenta},
    % numberstyle=\tiny\color{codegray},
    % stringstyle=\color{codepurple},
    basicstyle=\ttfamily\footnotesize,
    breakatwhitespace=false,         
    breaklines=true,                 
    captionpos=b,                    
    keepspaces=true,                 
    numbers=left,                    
    numbersep=5pt,                  
    showspaces=false,                
    showstringspaces=false,
    showtabs=false,                  
    tabsize=2
}
\lstset{style=mystyle}

\newcommand\course{6 - Функціональний аналіз}
\newcommand\hwnumber{ДЗ №1}                   % <-- homework number
\newcommand\idgroup{ФІ-91}                
\newcommand\idname{Михайло Корешков}  

\usepackage[framemethod=TikZ]{mdframed}
\mdfsetup{%
	backgroundcolor = black!5,
}
\mdfdefinestyle{ans}{%
    backgroundcolor = green!5,
    linecolor = green!50,
    linewidth = 1pt,
}

\pagestyle{fancyplain}
\headheight 35pt
\lhead{\idgroup \\ \idname}
\chead{\textbf{\Large \hwnumber}}
\rhead{\course \\ \today}
\lfoot{}
\cfoot{}
\rfoot{\small\thepage}
\headsep 1.5em

\linespread{1.2}

\begin{document}

% 10.15-10.21, 11.27

\section*{№ 10.15}
\subsection*{1. $M\xi$}
\begin{align*}
    M\xi &= 0\cdot 0.2 + 1\cdot 0.3 + 3\cdot 0.5 = \\
    &= 0.3 + 1.5 = \\
    &= 1.8
\end{align*}

\subsection*{2. $D\xi$}
\begin{align*}
    M\xi^2 &= 0\cdot 0.2 + 1\cdot 0.3 + 9\cdot 0.5 = \\
    &= 0.3 + 4.5 = \\
    &= 4.8
\end{align*}
\begin{align*}
    D\xi &= M\xi^2 - (M\xi)^2 = 4.8 - 3.24 = \\
    &= 1.56
\end{align*}

\subsection*{3. $M\xi^5$}
\begin{align*}
    M\xi^5 &= 0\cdot 0.2 + 1\cdot 0.3 + 3^5\cdot 0.5 = \\
    &= 0.3 + 121.5 = \\
    &= 121.8
\end{align*}

\subsection*{4. $Me^\xi$}
\begin{align*}
    Me^\xi &= 1 \cdot 0.2 + e \cdot 0.3 + e^3 \cdot 0.5 
\end{align*}

\section*{№ 10.16}
$\xi$ - кількість випадань шістки

$$P(\text{в i-му підкиданні випала 6}) = 1/6$$
$$P(\text{серед n підкидань рівно в k випала 6}) = С_n^k (1/6)^k(5/6)^{n-k}$$

У цій нотації, $\xi = k$

Ми також можемо представити $\xi$ як
$$\xi = \sum_{i=1}^n B(1/6)$$
суму незалежних повторних експериментів з двома результатами та ймовірністю успіху $1/6$

$$M\xi = \sum_{i=1}^n M\;B(1/6) = n \cdot 1/6$$

\begin{mdframed}[style=ans]
    $$M\xi = \frac{n}{6}$$
    $$P(\xi = k) = С_n^k (1/6)^k(5/6)^{n-k},\quad k=\overline{1,n}$$
\end{mdframed}



\section*{№ 10.17}
$$\xi_1, ..., \xi_n - \text{ очки, що випали на n гральних кубиках}$$
$$P(\xi_i = a) = 1/6;\quad a=\overline{1,6}$$
$$\xi_i - \text{ незалежні}$$

Одразу обчислю
$$M\xi_i = \sum_{j=1}^6 j/6 = 21/6 = \frac{7}{2}$$

\begin{mdframed}[style=ans]
    $$M(\sum \xi_i) = \sum M\xi_i = n\cdot \frac{7}{2}$$
    $$M(\prod \xi_i) = \left(\text{за незалежністю }\xi_i\right) = \prod M\xi_i = \left(\frac{7}{2}\right)^n$$
\end{mdframed}

\section*{№ 10.18}
\begin{mdframed}
    $$p(x) = \frac{1}{\sqrt{2\pi}\sigma}e^{-\frac{(x-a)^2}{2\sigma^2}}$$
    $$M(\xi-a) = ?$$
\end{mdframed}

$$M(\xi-a) = M\xi - a$$

$$ \int (x-a)\cdot \frac{1}{\sqrt{2\pi}\sigma}e^{-\frac{(x-a)^2}{2\sigma^2}} dx = $$
$$= \frac{1}{2}\cdot \frac{1}{\sqrt{2\pi}\sigma} \int e^{-\frac{(x-a)^2}{2\sigma^2}} d(x-a)^2 = $$
Вводимо заміну: $y = -\frac{(x-a)^2}{2\sigma^2}$
$$= -\frac{\sigma}{\sqrt{2\pi}} \int e^{y} dy = $$
$$= -\frac{\sigma}{\sqrt{2\pi}} e^y = -\frac{\sigma}{\sqrt{2\pi}} e^{-\frac{(x-a)^2}{2\sigma^2}}$$

$$M(\xi-a) = \int_{-\infty}^\infty (x-a)\cdot \frac{1}{\sqrt{2\pi}\sigma}e^{-\frac{(x-a)^2}{2\sigma^2}} = $$
$$=\left. -\frac{\sigma}{\sqrt{2\pi}} e^{-\frac{(x-a)^2}{2\sigma^2}}\right|_{x=-\infty}^{\infty} = 0$$

\begin{mdframed}[style=ans]
    $$M(\xi-a) = 0$$
    Як і має бути, бо $\xi$ - просто зсунутий нормальний розподіл, а $\xi-a$ центрує його.
\end{mdframed}

\section*{№ 10.19}

\begin{mdframed}
    $$p(x) = \begin{cases}
        \frac{\sin x}{2}, & x\in [0;\pi]\\
        0,& x\notin [0;\pi]
    \end{cases}$$
\end{mdframed}

$$F(x) = P(\xi \le x) = \int_{-\infty}^x p(x) dx = \int_0^{\min(x,\pi)} \frac{\sin x}{2} dx$$
$$F(x) = \begin{cases}
    0, & x<0\\
    1-\frac{\cos x}{2}, & x\in [0;\pi]\\
    1, & x>\pi
\end{cases}$$


\subsection*{1. $P(\xi \in [\pi/4; \pi/3])$}

$$P(\xi \in [\pi/4; \pi/3]) = F(\pi/3) - F(\pi/4) = 1-\frac{\cos \pi/3}{2} - 1 + \frac{\cos \pi/4}{2} = \frac{\sqrt{2}-1}{2}$$

\begin{mdframed}[style=ans]
    $$P(\xi \in [\pi/4; \pi/3]) = \frac{\sqrt{2}-1}{2} \approx 0.2$$
\end{mdframed}

\subsection*{2. $M\xi$}
$$M\xi = \int_0^\pi x \frac{\sin x}{2} dx = - \int_0^\pi x  d\frac{\cos x}{2} = -x\frac{\cos x}{2} |_0^\pi + \int_0^\pi \frac{\cos x}{2} dx = $$
$$= +\frac{\pi}{2} + \frac{\sin x}{2} |_0^\pi = \frac{\pi}{2}  $$

\begin{mdframed}[style=ans]
    $$M\xi =\frac{\pi}{2}$$
\end{mdframed}

\subsection*{3. $M\cos \xi$}
$$M\cos \xi = \int_0^\pi \cos x \frac{\sin x}{2} dx = \int_0^\pi \frac{\sin 2x}{4} dx = -\frac{\cos 2x}{4} |_0^\pi = 1 - 1 = 0 $$

\begin{mdframed}[style=ans]
    $$M\cos\xi =0$$
\end{mdframed}

\section*{№ 10.20}
\begin{mdframed}
    $\xi, \eta$ - нехалежні в.в.\\
    $$M\xi = 1, \quad M\eta = 2, \quad D\xi = 1, \quad D\eta = 4$$
\end{mdframed}

\begin{mdframed}[backgroundcolor=violet!15]
    $$D\xi = M\xi^2 - (M\xi)^2$$
\end{mdframed}

\subsection*{a)}
$$\begin{gathered}
    M(\xi+\eta+1)^2 = M(\xi^2 + \eta^2 + 1 + 2\xi\eta + 2\xi + 2\eta) = \\
    = M(\xi^2) + M(\eta^2) + 2M(\xi)M(\eta) + 2M\xi + 2M\eta + 1 = \\
    = D\xi+(M\xi)^2 + D\eta + (M\eta)^2 + 2M(\xi)M(\eta) + 2M\xi +2M\eta + 1 = \\
    = 1 + 1 + 4 + 4 + 2\cdot2\cdot1 + 2\cdot 1 + 2\cdot 2 = \\
    = 20
\end{gathered} $$
\begin{mdframed}[style=ans]
    $$M(\xi+\eta+1)^2 = 20$$
\end{mdframed}

\subsection*{b)}
$$\begin{gathered}
    D\xi\eta = M(\xi^2\eta^2) - (M\xi\eta)^2 = \\
    = M(\xi^2) M(\eta^2) - (M\xi)^2(M\eta)^2 = \\
    = \bigl(D\xi+(M\xi)^2\bigr)\bigl( D\eta + (M\eta)^2\bigr) - (M\xi)^2(M\eta)^2 = \\
    = (1 + 1)(4 + 4) - 1\cdot 4 = \\
    = 12
\end{gathered}$$
\begin{mdframed}[style=ans]
    $$D\xi\eta = 12$$
\end{mdframed}
\pagebreak

\section*{№ 10.21}
\begin{mdframed}
    $$\xi_1 ... \xi_n \sim U[0;1]$$
    $$p(x) = \mathbb{1}(x\in[0;1])$$
    Всі незалежні в сукупності
\end{mdframed}

\subsection*{1) $M(\xi_1 + ... + \xi_n)$}
\begin{mdframed}[style=ans]
$$
\begin{gathered}
    M(\xi_1 + ... + \xi_n) = M\xi_1 + ... + M\xi_n = n \cdot \frac{1}{2}
\end{gathered}
$$
\end{mdframed}

\subsection*{2) $D(\xi_1 + ... + \xi_n)$}

$$M\xi^2 = \int_{[0,1]} x^2 dx = \frac{1^3}{3} - 0 = \frac{1}{3}$$
$$D\xi = M\xi^2 - (M\xi)^2 = \frac{1}{3} - \frac{1}{4} = \frac{1}{12}$$
\begin{mdframed}[style=ans]
    $$
    \begin{gathered}
        D(\xi_1 + ... + \xi_n) = D\xi_1 + ... + D\xi_n = n \cdot \frac{1}{12}
    \end{gathered}
    $$
\end{mdframed}


\subsection*{3) $M(\sqrt[n]{\xi_1 \cdot ... \cdot \xi_n})$}

$$M\sqrt[n]{\xi} = \int_{[0;1]} x^{1/n} dx = \frac{n+1}{n} 1^{\frac{n+1}{n}} - 0 = \frac{n+1}{n} $$

$$M(\sqrt[n]{\xi_1 \cdot ... \cdot \xi_n}) = $$
$$= \idotsint_{\mathbb{R}^n} \sqrt[n]{x_1} \cdot ... \cdot \sqrt[n]{x_n} f_\xi(x_1, ..., x_n) dx_1 ... dx_n = $$
$$= \prod_{i=1}^n \int_{\mathbb R} \sqrt[n]{x_i} f_\xi (x_i) dx_i = $$
$$= \prod_{i=1}^n M\sqrt[n]{\xi_i} = (M\sqrt[n]{\xi})^n = \left(\frac{n+1}{n}\right)^n $$
\begin{mdframed}[style=ans]
    $$M(\sqrt[n]{\xi_1 \cdot ... \cdot \xi_n}) = \left(\frac{n+1}{n}\right)^n$$
\end{mdframed}


\subsection*{4) $D(\xi_1 \cdot ... \cdot \xi_n)$}

$$
\begin{gathered}
    D(\xi_1 \cdot ... \cdot \xi_n) = M(\prod_{i=1}^n \xi_i)^2 - (M \prod_{i=1}^n \xi_i)^2;
\end{gathered}
$$

$$\begin{gathered}
    M(\prod_{i=1}^n \xi_i)^2 = M\prod_{i=1}^n \xi_i^2 = \prod_{i=1}^n M\xi_i^2 = \\
= (M\xi^2)^n = \frac{1}{3^n} = 3^{-n}
\end{gathered}$$

$$\begin{gathered}
    M \prod_{i=1}^n \xi_i = \prod_{i=1}^n M\xi_i = \frac{1}{2^n} = 2^{-n}
\end{gathered}$$

\begin{mdframed}[style=ans]
    $$D(\xi_1 \cdot ... \cdot \xi_n) = 3^{-n} - 2^{-n}$$
\end{mdframed}


\section*{№ 11.27}
\begin{mdframed}
    $$X,Y \sim Exp(\theta) - \text{ незалежні в.в.}$$
    Знайти сумісну щільність розподілів
    $$X+Y, \frac{X}{X+Y}$$
    Довести що вони незалежні
\end{mdframed}



$$p_X(x) = p_Y(x) = p(x) = \theta e^{-\theta x} \mathbb{1}_{\ge 0}(x)$$

Нехай $\gamma = (X+Y, \frac{X}{X+Y})$

$$\begin{cases}
    y_1 = x_1+x_2 > 0\\
    y_2 = \frac{x_1}{x_1+x_2} > 0\\
\end{cases}$$

Також:
$$\boxed{0 < y_2 \le 1}$$

$$\begin{cases}
    x_2 = y_1 - x_1 \\
    y_2 = \frac{x_1}{x_1+ y_1 - x_1}\\
\end{cases}$$
$$\begin{cases}
    x_1 = y_1 y_2\\
    x_2 = y_1 - x_1 = y_1 - y_1y_2 = y_1(1-y_2) \\
\end{cases}$$


$$
J = \begin{vmatrix}
    \frac{\partial x_1}{\partial y_1} & \frac{\partial x_1}{\partial y_2} \\
    \frac{\partial x_2}{\partial y_1} & \frac{\partial x_2}{\partial y_2} \\
\end{vmatrix} 
= \begin{vmatrix}
    y_2 & y_1 \\
    1-y_2 & -y_1 \\
\end{vmatrix} = -y_1y_2 - y_1 + y_1 y_2 = y_1
$$

$$p_\gamma(y_1,y_2) = p_{(X,Y)} (y_1 y_2, y_1(1-y_2)) \cdot y_1$$

$$p_{(X,Y)} (x y, x (1-y)) = p_X(xy) p_Y(x(1-y)) 
= \theta e^{-\theta xy} \cdot \theta e^{-\theta x(1-y)} \mathbb{1}_{> 0}(y)\mathbb{1}_{\le 1}(x)\mathbb{1}_{> 0}(y) $$

$$p_\gamma(x,y) = \theta e^{-\theta xy} \cdot \theta e^{-\theta x(1-y)} \cdot x
= \theta^2 xe^{-\theta x} \mathbb{1}_{> 0}(x)\mathbb{1}_{\le 1}(y)\mathbb{1}_{> 0}(y)$$

$$p_{X+Y}(x) = p_{\gamma_1}(x) = \int_{-\infty}^\infty p_\gamma(x,y)\mathbb{1}_{> 0}(x)\mathbb{1}_{\le 1}(y)\mathbb{1}_{> 0}(y) dy 
= \int_{-\infty}^\infty \theta^2 xe^{-\theta x}\mathbb{1}_{> 0}(x)\mathbb{1}_{\le 1}(y)\mathbb{1}_{> 0}(y) dy =$$
$$= \theta^2 xe^{-\theta x}\mathbb{1}_{> 0}(x) \int_{0}^1 dy = \theta^2 xe^{-\theta x}\mathbb{1}_{> 0}(x) $$

$$p_{\frac{X}{X+Y}}(y) = p_{\gamma_2}(y) = \int_{-\infty}^\infty p_\gamma(x,y)\mathbb{1}_{> 0}(x)\mathbb{1}_{\le 1}(y)\mathbb{1}_{> 0}(y) dx 
= \int_{-\infty}^\infty \theta^2 xe^{-\theta x}\mathbb{1}_{> 0}(x)\mathbb{1}_{\le 1}(y)\mathbb{1}_{> 0}(y) dx =$$
$$= -\frac{1}{\theta}\cdot \theta^2 \mathbb{1}_{[0;1]}(y) \cdot \left(xe^{-\theta x}|_{0}^\infty - \int_{0}^\infty e^{-\theta x}dx \right) = $$
\begin{mdframed}[backgroundcolor=violet!10]
    $$\lim_{x\to\infty} xe^{-\theta x} = \lim_{x\to\infty} \frac{1}{-\theta e^{\theta x}} = \frac{1}{\infty} = 0$$
\end{mdframed}
$$= -\theta \mathbb{1}_{[0;1]}(y) \cdot \left(0 + \frac{1}{\theta}  e^{-\theta x} |_0^\infty\right) = $$
$$= \mathbb{1}_{[0;1]}(y) \cdot 1$$

\subsection*{Незалежність}

$$p_{X+Y}(x)p_{\frac{X}{X+Y}}(y) = \theta^2 xe^{-\theta x}\mathbb{1}_{> 0}(x) \cdot \mathbb{1}_{[0;1]}(y) = p_\gamma(x,y) \; \qedsymbol$$

\begin{mdframed}[style=ans]
    
    $$p_\gamma(x,y) =\theta^2 xe^{-\theta x} \mathbb{1}_{> 0}(x)\mathbb{1}_{[0;1]}(y)$$
    Незалежність доведена вище
\end{mdframed}



\end{document}

