
\section{Текстура як розподіл}\label{section1.2}\hfill

Розглянемо деяку статистику $T = T(c)$, визначену в кожному пікселі, що залежить від значень зображення у точках околу $N_c$ навколо цього пікселя.
Припускаємо, що зображення можуть містити лише одну текстуру, тобто текстура гомогенна.
Розглядатимемо текстуру як певний розподіл статистики $T$ на зображенні. 
За цієї гіпотези, візуально різні текстури матимуть різні розподіли $T$.
Ми прагнемо описати ці різні невідомі розподіли вектором параметрів, який далі можна використовувати як вектор ознак у задачі класифікації.

\subsection{Текстура як розподіл одного дескриптора}\label{section1.2a}\hfill

Гарною оцінкою розподілу буде гістограма значень $T$ на зображенні, вважаючи спостереженнями значення $T(c), \; c \in K$ для кожного пікселя~$c$ із координатами з множини координат $K$.
Ми також нехтуватимемо залежністю між значеннями $T$ в сусідніх пікселях, вважатимемо значення $T$ незалежними один від одного, і однаково розподіленими на зображенні.
Працюватимемо в імовірнісному просторі $(\Omega, \mathcal F, \mu)$.
Маємо незалежні однаково розподілені випадкові величини 
\[ T_c \colon \Omega \to D, \quad c \in K, \quad T_c \sim F \] 
із законом розподілу (функцією розподілу) $F \colon D \to [0;1]$, де $D = \{T_{\min},\dots,T_{\max}\} \subset \Z$.
При цьому вважаємо $T_{\min}, T_{\max}$ відомими; зазвичай $T_{\min} = 0$, $T_{\max} < 2^{64}$.

Емпірична функція розподілу $\hat F(d) = \sum_{c\in K} \1\{\ T_c \le d \}$
дає гарну (консистентну та асимптотично нормальну) оцінку справжньої функції розподілу $F(d) = P(T \le d)$.
Нам буде зручніше працювати із імовірностями окремих значень $D$, $f \colon D \to [0;1]$, $f(d) = P(T = d)$.
При цьому відповідною оцінкою $f$ буде 
\[\hat f(d) = \frac{1}{\# K}\sum_{c\in K}\1\{T_c = d\} = \frac{\hat\nu_d}{\# K}.\]

Послідовність $H = \{ \hat \nu_d, \; d\in D \}$ називатимемо \textit{гістограмою} $T$ на зображенні.
Послідовність $H_1 = \{ \hat \nu_d / \# K, \; d\in D \} = \{\hat f(d), \; d\in D\}$ називатимемо нормованою гістограмою, або вектором відносних частот.

Зауважу, що для багатьох дескрипторів складно ввести змістовний лінійний порядок на області значень. 
Наприклад, значення дескрипторів $\mathrm{LBP}_{R,P}$ та $\mathrm{LBP}^{ri}_{R,P}$ є природнім числовим представленням бінарного вектора, що відповідає лексикографічному порядку для цих бінарних векторів.
Малі зміни у вихідному векторі можуть привести до великих змін у його значенні (наприклад, бітфліп на старших позиціях).
Враховуючи сенс дескрипторів, на множині значень можна було би ввести квазіпорядок домінування, який би краще відобразив її структуру, проте це ускладнило би визначення функції ймовірності дескриптора.
Інтуітивно, це означає, що статистичні процедури потрібно вводити незалежно від способу упорядкування множини значень, 
а також підтверджує доречність частішого використання дискретної щільності ймовірності замість функції імовірності.

Одним з найбільш поширених підходів є розглядати розподіл $T$ як \textit{мультиноміальний}. 
У цьому випадку параметром розподілу буде вектор $\Theta = \{p_d, \; d\in D\}$, тобто імовірності кожного значення $T$.
Нормована гістограма буде особливо гарною оцінкою, $\hat \Theta = \{\hat f(d), \; d\in D\} = H_1$.
Більша кількість параметрів не має сенсу за умови незалежності $T_c$, а менша потребує нетривіального підбору моделі.
Область значень дескриптора іноді ділять на інтервали $D = \sqcup_{g\in G} D_g$.
Поділ на інтервали стає необхідним, якщо кількість пікселів (спостережень) співрозмірна або менша за кількість різних значень дескриптора, 
або використовуються неперервні текстурні дескриптори, наприклад, скалярні статистики дескриптора GLCM, перелічені у \cite{belsare2015}.

Отже, вектором ознак текстури згідно дескриптора $T$ може бути його гістограма $H$,
\begin{equation*}
    \xi = \{\hat\nu_d, \; d\in D\}.
\end{equation*}

\subsection{Текстура як розподіл декількох дескрипторів}\label{section1.2b}\hfill

Один дескриптор нечасто дає достатньо інформації для того, щоб відрізнити текстури між собою, особливо в роботі із спорідненими текстурами, 
такими як різні стани одного матеріалу, чи різні підтипи одного типу біологічної тканини.
З іншого боку, різні текстури можуть проявляти свої особливості у різних просторових масштабах.
Відповідно, часто використовуються набори різних статистик $T^m$ (наприклад, LBP із різними параметрами $R$ та $P$).
Особливо інформативними вважаються моделі, побудовані на сумісних розподілах дискретних статистик разом з неперервними \cite{guo2010lbpv}.

Нехай статистики $T^m$ мають області значень $D^{(m)}$. 
У загальному випадку, сумісний розподіл всіх $T^m\colon\ \Omega \to D^{(m)}$, $m=1,\dots,M$ оцінюється багатовимірною гістограмою. 
У мультиноміальній моделі, кількість параметрів сумісного розподілу зростатиме як добуток потужностей області значень $\prod_{m=1}^M \# D^{(m)}$, або відповідних розбиттів областей значень.
Розглядати сумісні розподіли багатьох статистик недоцільно, бо із зростанням кількості статистик експоненційно зростає кількість класів гістограми, 
збільшується довжина вектора ознак, зменшуючи кількість спостережень кожного класу, та статистичну значущість висновувань.
Тому, дескриптори $T^m$ часто розглядають як незалежні, від чого кількість параметрів сумісного розподілу зростатає не як добуток, а лише як сума $\sum_{m=1}^M \# D^{(m)}$.

Отже, вектор ознак текстури згідно декількох незалежних дескрипторів $T^m$ можна утворити конкатенацією декількох гістограм,
\begin{equation*}
    \xi = \bigsqcup_{m=1}^M \{\hat \nu^{(m)}_d, \; d\in D^{(m)}\}.
\end{equation*}



% ================================

