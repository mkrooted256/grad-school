\documentclass[11pt, a4paper]{article} % use larger type; default would be 10pt

\usepackage{fontspec} % Font selection for XeLaTeX; see fontspec.pdf for documentation
\defaultfontfeatures{Mapping=tex-text} % to support TeX conventions like ``---''
\usepackage{xunicode} % Unicode support for LaTeX character names (accents, European chars, etc)
\usepackage{xltxtra} % Extra customizations for XeLaTeX
\usepackage{tikz}
\usetikzlibrary{arrows,calc,patterns}


% other LaTeX packages.....
\usepackage{fullpage}
\usepackage[top=2cm, bottom=4.5cm, left=2.5cm, right=2.5cm]{geometry}
\usepackage{amsmath,amsthm,amsfonts,amssymb,amscd,systeme}
\usepackage{unicode-math}
\usepackage{cancel}
\geometry{a4paper} 
%\usepackage[parfill]{parskip} % Activate to begin paragraphs with an empty line rather than an indent
\usepackage{fancyhdr}
\usepackage{listings}
\usepackage{graphicx}
\usepackage{hyperref}
\usepackage{multicol}

% FONTS
% \setmainfont[Ligatures=TeX]{Cambria Math} % set the main body font (\textrm), assumes Charis SIL is installed
%\setsansfont{Deja Vu Sans}
% \setmonofont[Ligatures=TeX]{Fira Code}
\setmathfont[Ligatures=TeX]{NewCMMath-Regular}

\setmainfont{Cambria}
\setmonofont[Ligatures=TeX]{Roboto Mono}

\renewcommand\lstlistingname{Algorithm}
\renewcommand\lstlistlistingname{Algorithms}
\def\lstlistingautorefname{Alg.}
\lstdefinestyle{mystyle}{
    % backgroundcolor=\color{backcolour},   
    % commentstyle=\color{codegreen},
    % keywordstyle=\color{magenta},
    % numberstyle=\tiny\color{codegray},
    % stringstyle=\color{codepurple},
    basicstyle=\ttfamily\footnotesize,
    breakatwhitespace=false,         
    breaklines=true,                 
    captionpos=b,                    
    keepspaces=true,                 
    numbers=left,                    
    numbersep=5pt,                  
    showspaces=false,                
    showstringspaces=false,
    showtabs=false,                  
    tabsize=2
}
\lstset{style=mystyle}

\newcommand\course{MS1 - Algebra}
\newcommand\hwnumber{HW 1}                   % <-- homework number
\newcommand\idgroup{111-2023}                
\newcommand\idname{Mykhailo Koreshkov}  

\usepackage[framemethod=TikZ]{mdframed}
\mdfsetup{%
	backgroundcolor = black!5,
}
\mdfdefinestyle{ans}{%
    backgroundcolor = green!5,
    linecolor = green!50,
    linewidth = 1pt,
}

\pagestyle{fancyplain}
\headheight 35pt
\lhead{\idgroup \\ \idname}
\chead{\textbf{\Large \hwnumber}}
\rhead{\course \\ \today}
\lfoot{}
\cfoot{}
\rfoot{\small\thepage}
\headsep 1.5em

\linespread{1.2}

\newcommand{\R}{\mathbb{R}}
\newcommand{\N}{\mathbb{N}}
\newcommand{\Z}{\mathbb{Z}}
\DeclareMathOperator{\lcm}{lcm}
\DeclareMathOperator{\cd}{CD}


\begin{document}


\section*{Ex 1.1}
\begin{mdframed}
    Довести що подільність є відношенням (строгого) порядку на $\N$
\end{mdframed}

\begin{mdframed}[backgroundcolor=blue!10]
    \[\forall a,b \in \N: \quad a|b \iff \exists c\in \N: b=ac \]
\end{mdframed}

Note: $0 \in \N$

\begin{enumerate}
    \item Рефлексивність ($\forall a\in\N: a|a$).
    \begin{proof}
    Let $a\in\N$. $a = 1\cdot a$. That is, $c = 1$. Therefore, $a|a$.\\
    Note: also true for $a=0$.
    \end{proof}

    \item Антисиметричність ($\forall a,b\in\N : (a|b \land b|a) \implies a=b$).
    \begin{proof}
    Let $a,b\in \N\setminus\{0\},  \; a|b,\; b|a$. \\
    That is, $\exists c_1,c_2\in\N : b = ac_1, \; a = bc_2$. Then $b = c_1c_2b$. \\
    Consequently, $|b| = |c_1||c_2||b|$. And that means $|c_1||c_2|=1$. So $c_1 = 1, c_2 = 1$.\\
    Finally, $\;b = 1 \cdot a, \; a = 1 \cdot b\;$ and therefore $a=b$. \\
    So, $(a|b \land b|a) \implies a=b$

    Let $a = 0, \; b\ne 0$. Then $a|b$ means $0\ne b = ac = 0$, which is a contradiction.

    Let $a=0, b=0$. Then $a=b$ and claim is true.

    \end{proof}

    \item Транзитивність ($\forall a,b,c\in\N : (a|b \land b|c)\implies a|c$)
    \begin{proof}
        $b|c \implies \exists \alpha \in \N: c = \alpha b$.\\
        $a|b \implies \exists \beta \in \N: b = \beta a$.\\
        So, $c = \alpha\beta a$. That is, $\exists \gamma = \alpha\beta \in \N: c = \gamma a$
        which means $a|c$.
    \end{proof}
\end{enumerate}

\begin{mdframed}[backgroundcolor=blue!10]
    тут і надалі використовуватиму:\\
    $\gcd(a,b) \equiv \text{нсд}(a,b)$\\ 
    $\lcm(a,b) \equiv \text{нск}(a,b)$
\end{mdframed}

\newpage
\section*{№1.2}
\begin{mdframed}
    Довести, що \[\forall a \in \Z:\; \gcd(a,0) = |a|\]
\end{mdframed}

\begin{mdframed}[backgroundcolor=blue!10]
    $c$ is a common divisor (CD) of $a,b$ if $c|a \land c|b$.\\
    $\gcd(a,b) = l$ if $l \in \cd(a,b)$ and $\forall c\in \cd(a,b): c|l$
\end{mdframed}


\begin{proof}
    Розглянемо спільні дільники $a$ та $0$.\\
    $\forall k \in \Z: k|0, \;\text{бо}\quad \exists c = 0: 0 = c\cdot k$\\
    В тому числі, $(|a|)\;|\;0$. З іншого боку, $|a|$ завжди ділить $a$.\\
    Дільники 0: $d(0) = \Z$\\
    Дільники a: $d(a)$ - деяка множина, $\max d(a) = |a|$\\
    $\cd(a,0) = \Z \cap d(a) = d(a)$ \\ 
    \\
    Нехай $c\in \cd(a,0) = d(a)$. За визначенням $d(a)$, $c\;|\;(|a|)$.\\
    Тобто, $|a| \in \cd(a,0) \land \; \forall c\in \cd(a,0): c\;|\;(|a|)$, 
    що значить що $\gcd(a,0) = |a|$
\end{proof}


\section*{№1.3}
\begin{mdframed}
    Нехай $a,b,n \in \Z, \; n>0$.
    Довести, що твердження еквівалентні:
    \begin{enumerate}
        \item $a \mod n = b \mod n$
        \item $\exists k\in\Z: a-b = nk$
    \end{enumerate}
\end{mdframed}

$1 \implies 2$
\begin{proof}
    Let $a \mod n = b \mod n$. Тобто
    \[\exists s,t,r\in\Z: \quad a = sn + r; \quad b = tn + r\]
    Тоді
    \[a-b = sn + r - tn - r = (s-t)n\]
    Тобто $n|(a-b)$
\end{proof}
\newpage
$2 \implies 1$
\begin{proof}
    Let $a,b: \; \exists k\in\Z: a-b = nk$.
    Можемо записати результат ділення з остачею на $n$:
    \begin{align*}
        a &= sn + r \\
        b &= tn + p
    \end{align*}
    Розглянемо різницю
    \[a-b = sn+r - tn - p = (s-t)n + (r-p)\]
    \[n\;|\;\left((s-t)n + (r-p)\right)\]
    \[|r-p| < n \;\text{бо}\; |r|,|p|<n\]
    Значить $r-p=0$ та $r=p$.
\end{proof}


\section*{№1.4}
\begin{mdframed}
    Дослідити які значення може приймати
    \begin{enumerate}
        \item $\gcd(n,n+2)$
        \item $\gcd(n,n+6)$
    \end{enumerate}
\end{mdframed}

\subsection*{1}
Нехай $\;a|n, \; a|(n+2)\;$. Тоді $a|(n+2-n)$. Тобто $\;a|2$.
А це означає, що $a \in \{1, 2\}$

Приклади
\begin{itemize}
    \item $n=3, n+2=5, \quad \gcd(n,n+2) = 1$
    \item $n=10, n=12, \quad \gcd(n,n+2) = 2$
\end{itemize}

\subsection*{2}
Нехай $\;a|n,\; a|(n+6)\;$. Тоді $a|(n+6-n)$, тобто $\;a|6$.
Таким чином $a \in \{1,2,3,6\}$.

Приклади
\begin{itemize}
    \item $n=5, n+6=11, \quad \gcd(n,n+6) = 1$
    \item $n=2, n+6=8, \quad \gcd(n,n+6) = 2$
    \item $n=3, n+6=9, \quad \gcd(n,n+6) = 3$
    \item $n=6, n+6=12, \quad \gcd(n,n+6) = 6$
\end{itemize}

Додаткове твердження: $\left(\gcd(a,b) = 1 \land c|ab \right) \implies \left(c|a \vee c|b\right)$

\begin{mdframed}    
\begin{proof}
    Припустимо, що $c\nmid a$. Достатньо довести, що тоді $c\mid b$.

    По-перше, зазначу, що $1$ - єдиний спільний дільник $a$ та $b$.

    Оскільки $c\nmid a$, то $\exists s,r\in\N: a = sc + r,\; 1 \le r < c$.
    Оскільки $c|ab$, то $\exists k: ab = kc$.
    \begin{gather*}
        ab = kc\\
        (sc + r)b = kc\\
        scb + rb = kc\\
        rb = c(k-sb)
    \end{gather*}

    $c\nmid r$, адже $0<r<c$, отож $c\mid b$.
\end{proof}
\end{mdframed}

\section*{№1.5}
\begin{mdframed}
    Довести
    \[\gcd(a,c)=1 \land \gcd(b,c)=1 \implies \gcd(ab,c)=1\]
\end{mdframed}

\begin{proof}
З умови слідує, що $\exists x_1,x_2,y_1,y_2 \in \Z:$ 
\begin{gather*}
   ax_1 + cx_2 = 1\\
   by_1 + cy_2 = 1
\end{gather*}
З цього слідує, що 
\begin{gather*}
    (ax_1+cx_2)(by_1+cy_2) = abx_1y_1 + bcx_2y_1 + acx_1y_2 + c^2x_2y_2 = 1\\
    ab\cdot x_1y_1 + c \cdot (bx_2y_1 + ax_1y_2 + cx_2y_2) = 1
\end{gather*}
Таким чином ми виразили одиницю через лінійну комбінацію $ab$ й $c$, що еквівалнетно 
\[\gcd(ab,c)=1\]. 
\end{proof}

\newpage
\section*{№1.6}
\begin{mdframed}
    Довести
    \[c>0 \implies \gcd(ca,cb) = c\gcd(a,b)\]
\end{mdframed}

\begin{proof}
Позначимо $s_{a,b}(x,y) = ax+by$. 

Нехай $\gcd(a,b) = g$. 
Тоді існують $x,y\in\Z$ такі що $ax+by=g$.
Домножимо обидва боки рівняння на $c$ і маємо 
\begin{gather*}
    ax+by = s_{a,b}(x,y) = g\\
    cax+cby = s_{a,b}(cx,cy) = s_{ca,cb}(x,y) = cg
\end{gather*}

Припустимо, що $\exists x',y',h\in\Z, 0<h<cg$ такі що $s_{ca,cb}(x',y')=h$.
Тоді $h = c(ax'+by') < c(ax+by) = cg$, звідки $ax'+by' < ax+by$, що неможливо, бо $g=\gcd(a,b)$ 
це найменше додатне значення лінійної комбінації $a$ та $b$ і воно має єдині відповідні значення $x,y$.
Тому, $cg$ дійсно є найменшим значенням лінійної комбінації $ca$ та $cb$, а отже, 
\[cg = \gcd(ca,cb)\]
\end{proof}

Деякі тверження які я доводив щоб розібратися з темою, але які не знадобилися в доведенні явним чином:

\begin{mdframed}
Нехай $\gcd(a,b)=1 \land ax+by=0$. Доведемо що $x=y=0$.
\begin{proof}
    По-перше, якщо $ax = by$, то $ax+by=2bx=0$, що можливо лише якщо $x=y=0$.
    Припустимо, що $ax \ne by$. Тоді маємо
    \[
    \begin{cases}
    (ax-by)(ax+by) = a^2x^2 + b^2y^2 = 0\\
    a,b>0\\
    ax-by \ne 0
    \end{cases}
    \]
    З цього слідує, що $x=y=0$.
    Отже, 
    \[\left(\gcd(a,b)=1 \land a,b>0 \land ax+by = 0\right) \iff x=y=0\]
\end{proof}
\end{mdframed}

\begin{mdframed}
    Тепер я хочу довести, що для фіксованих $a,b$ таких що $\gcd(a,b)=1$ справедливо \\
    $ax+by=ax'+by' \implies \left(x=x' \land y=y'\right)$
    \begin{proof}
        Нехай $ax+by = ax'+by'$.
        Маємо $a(x-x') + b(y-y')=0$. Раніше доведено, що з цього слідує $x-x'=y-y'=0$.
        Отже, $x=x' \land y=y'$
    \end{proof}
\end{mdframed}

\begin{mdframed}
    Наостанок доведу, що для фіксованих $a,b$ таких що $\gcd(a,b)=1$ \\
    відображення
    $ax+by : \Z^2 \to \Z$ є сюр'єктивним.
    \begin{proof}
        $\exists x,y\in\Z: ax+by=1$. Тоді $\forall k\in\Z: k=a(xk)+b(yk)$.
    \end{proof}
\end{mdframed}

\begin{mdframed}
    Таким чином довели, що для фіксованих $a,b$ таких що $\gcd(a,b)=1$ \\
    відображення
    $ax+by : \Z^2 \to \Z$ є бієкцією.

    Більше того, якщо $\gcd(a,b)=g>1$, то $s_{a,b}(x,y):\Z^2 \to g\Z = \{gk : k\in \Z\}$ також буде бієкцією.
\end{mdframed}

\section*{№1.7}
\begin{mdframed}
    Довести 
    \[\left(c\mid ab \land \gcd(a,c)=d\right) \implies c\mid db\]
\end{mdframed}

\begin{proof}
    З лівої частини логічного слідування маємо
    \[\exists k,\alpha,\gamma,x,y\in\Z: 
    \begin{cases}
        ab = kc\\
        a=\alpha d\\
        c=\gamma d\\
        ax+cy=d
    \end{cases}
    \]
    Далі маємо
    \begin{gather*}
        axb+cyb=db\\
        (ab)x+cyb=db\\
        c(kx + yb)=db
    \end{gather*}
    А отже, \[c\mid db\]
\end{proof}

\newpage
\section*{№1.8}
\begin{mdframed}
    \[a = 5188;\quad b = 4709\]
    Знайти $g=\gcd(a,b)$ та його представлення у вигляді лін.комбінації $a$ та $b$.
\end{mdframed}

Застосую розширений алгоритм Евкліда.

\begin{lstlisting}
a = n_0 = 5188
b = n_1 = 4709
---------------------
5188 = 4709 * 1 + 479
4709 = 479 * 8 + 877
479 = 877 * 0 + 479
877 = 479 * 1 + 398
479 = 398 * 1 + 81
398 = 81 * 4 + 74
81 = 74 * 1 + 7
74 = 7 * 10 + 4
7 = 4 * 1 + 3
4 = 3 * 1 + 1
3 = 1 * 3 + 0
\end{lstlisting}

Тобто, $\gcd(5188,4709) = 1$, бо це останній ненульовий залишок в послідовності.

Зворотній хід для визначення коефіцієнтів:
\begin{lstlisting}
479 = 5188 - 4709*1
479 = a - b
877 = b - 479*8 = b - (a-b)*8 = 9b-8a
398 = 877 - 479*1 = (9b-8a) - (a-b)*1 = 10b-9a
81 = 479 - 398*1 = (a-b) - (10b-9a)*1 = 10a-11b
74 = 398 - 81*4 = (10b-9a) - (10a-11b)*4 = 54b-49a
7 = 81 - 74*1 = (10a-11b) - (54b-49a)*1 = 59a-65b
4 = 74 - 7*10 = (54b-49a) - (59a-65b)*10 = 704b-639a
3 = 7 - 4*1 = (59a-65b) - (704b-639a)*1 = 698a-769b
1 = 4 - 3*1 = (704b-639a) - (698a-769b)*1 = 1473b-1337a
\end{lstlisting}

Тобто маємо
\[1473b-1337a=1\]
\includegraphics[width=0.5\textwidth]{gcd1.png}

\newpage
\section*{№1.9}
\begin{mdframed}
    Let $\gcd(a,b)=1$.
    Довести
    \[a\mid c \land b\mid c \implies ab\mid c\]
\end{mdframed}

\begin{proof}
    Нехай $\gcd(a,b) = 1, c\in\Z$. $\exists x,y,\alpha,\beta\in\Z:$
    \begin{gather*}
        ax+by=1\\
        c=\alpha a\\
        c=\beta b
    \end{gather*}
    Тоді
    \begin{gather*}
        \alpha\beta ax + \alpha\beta by = \alpha\beta\\
        \alpha a\cdot \beta x + \beta b \cdot \alpha y = \alpha\beta\\
        c(\beta x + \alpha y) = \alpha\beta\\
        \alpha\beta = c\gamma, \quad (\gamma = \beta x + \alpha y)\\
        c^2 = \alpha\beta ab = c \gamma ab\\
        c = \gamma ab
    \end{gather*}

    Отже, $ab\mid c$

\end{proof}

\end{document}

