\documentclass{thesis}

\renewcommand{\baselinestretch}{1.20}

\newcommand{\MyFullName}{КОРЕШКОВ~Михайло Олександрович}
\newcommand{\MyInitialName}{Корешков~М.О.}
\newcommand{\MyReverseInitialName}{М.О.~Корешков}
\newcommand{\ThesisName}{[ТИМЧАСОВА] Розробка високопаралельного алгоритму локального бінарного шаблону з оптимальним набором векторів ознак для семантичної сегментації гістопатологічних зображень}

\newcommand{\SuperviserFullName}{ФРОЛОВА Аліна Олександрівна}
\newcommand{\SuperviserInitialName}{Фролова~А.О.}
\newcommand{\SuperviserReverseInitialName}{А.О.~Фролова}
\newcommand{\SuperviserAffiliation}{Молодший науковий співробітник\\Інституту молекулярної біології\\і генетики НАН~України}

\newcommand{\NumPages}{??}
\newcommand{\NumCitations}{??}

\newcommand{\MyFullNameEn}{Koreshkov~Mykhailo}
\newcommand{\MyInitialNameEn}{Koreshkov~M.O.}
\newcommand{\MyReverseInitialNameEn}{M.O.~Koreshkov}
\newcommand{\ThesisNameEn}{[TEMPORARY] Development of highly parallel local binary patterns algorithm with a modified feature vector for semantic segmentation of histopathological images}

\newcommand{\SuperviserFullNameEn}{Frolova Alina}
\newcommand{\SuperviserInitialNameEn}{Frolova~A.O.}
\newcommand{\SuperviserReverseInitialNameEn}{A.O.~Frolova}
\newcommand{\SuperviserAffiliationEn}{Junior Research Fellow at Institute of molecular biology and genetics of National academy of sciences of Ukraine}

% TODO
% Розшифровування не потрібно!
\newcommand{\MSC}{62H30~Класифікація та дискримінація; кластерний аналіз, 62\nobreakdash-08~Обчислювальні методи для задач статистики, 62P10~Застосування статистики в біології та природничих науках, 68T10~Розпізнавання образів}
\newcommand{\MSCEn}{62H30~Classification and discrimination; cluster analysis (statistical aspects), 62-08~Computational methods for problems pertaining to statistics, 62P10~Applications of statistics to biology and medical sciences, 68T10~Pattern recognition}

\newcommand{\E}{\operatorname{\mathbb E}}
\newcommand{\D}{\operatorname{\mathbb D}}
\renewcommand{\P}{\operatorname{\mathbb P}}

% З титулки прибрати "Тема" і лапки

% Диплом можна англійською. Анотація та титулка має бути обома мовами.
% Презентація диплому має бути українською (слайди можна англійською!).
% В'ячеслав Миколайович рекомендує починати з літератури (bib).
% Ключові слова у порядку важливості.
% Подумати про сортування умовних позначень

\begin{document}
\allowdisplaybreaks

\large

\setcounter{page}{1}
\thispagestyle{empty}
\centerline{Національна академія наук України}
\centerline{Міністерство освіти і науки України}
\centerline{Державна наукова установа «Київський академічний університет»}

\vspace{10mm}

\begin{flushright}
\begin{minipage}{100mm}
\begin{center}\large {\bf <<Допущено до захисту>>}\\
Завідувач кафедри математики,\\
доктор фіз.-мат. наук\\
{\bf Вячеслав БОЙКО}\\
<<\underline{\hspace{8mm}}>> травня 2025 р.
\end{center}
\end{minipage}
\end{flushright}

\vspace{10mm}

\centerline{\Large \bf \MyFullName}

\begin{center}
{\bf КВАЛІФІКАЦІЙНА РОБОТА}\\
на здобуття освітнього ступеня <<магістр>>\\
Спеціальність 111 <<Математика>>\\[4mm]
{\Large \bf Тема: <<\ThesisName>>}
\end{center}


\vspace{5mm}

\noindent
{Засвідчую, що кваліфікаційна робота містить результати власних досліджень. Використання ідей, результатів і~текстів інших авторів мають посилання на відповідне джерело.
\underline{\hspace{18mm}} \MyReverseInitialName \par}

\vspace{5mm}

\begin{flushright}
\begin{minipage}{90mm}
\large {\bf Науковий керівник}\\
% доктор фіз.-мат. наук, професор\\
\SuperviserAffiliation\\
{\bf \SuperviserFullName }\\
\underline{\hspace{48mm}}

\end{minipage}
\end{flushright}


\vfill

\centerline{\bf Київ --- 2025}

\newpage

\begin{center}
\Large \bf Анотація
\end{center}

\noindent
\textbf{\MyInitialName}, \textbf{\ThesisName}, Кваліфікаційна робота на здобуття освітнього ступеня <<магістр>> за спеціальністю 111 Математика, Київський академічний університет, Київ, 2025, \NumPages~с., \NumCitations~джерел.

\bigskip


{\bf ???Текст анотації???}

\bigskip

\noindent
{\bf MSC:} \MSC

\bigskip

\noindent
{\bf Ключові слова:} сегментація зображень, текстурний дескриптор, локальні бінарні шаблони, метрики якості кластеризації.
\bigskip

\newpage

\begin{center}
\Large \bf Abstract
\end{center}

\noindent
\textbf{\MyReverseInitialNameEn}, \textbf{\ThesisNameEn}, Master Thesis, speciality 111 Mathematics.~--
Kyiv Academic University, Kyiv, 2025, \NumPages~pages, \NumCitations~references.

\bigskip
{\bf ???Abstract???}

\bigskip

\noindent
{\bf MCS:} \MSCEn %for 111, see https://mathscinet.ams.org/mathscinet/msc/msc2020.html

%\noindent
%{\bf ACM:} ??????? %for 122, see https://cran.r-project.org/web/classifications/ACM.html

\bigskip

\noindent
{\bf Key words:} image segmentation, texture descriptor, local binary patterns, clustering metrics


\newpage


\tableofcontents

\newpage

\phantomsection
\section*{Перелік умовних позначень}
\addcontentsline{toc}{chapter}{Перелік умовних позначень}

\bigskip


\begin{tabular}{ll}
$\mathbb R^n$ & $n$-вимірний евклідів простір\\[1mm]
$2^A$ & Булеан множини $A$, $2^A = \{B \mid B \subset A\}$ \\
$\Omega, E$ & Простори елементарних подій \\
$\mathcal F \subset 2^\Omega, \; \mathcal E \subset 2^E$ & $\sigma$-алгебри на просторі елементарних подій \\
$A, B, C \in \mathcal F$ & Випадкові події, елементи $\sigma$-алгебри \\
$1_A(x)$ & Індикатор події (вимірної множини) $A$ \\
$\P(A)$ & Ймовірність події $A$ \\
$\P(A\mid B)$ & Умовна ймовірність події $A$ за умови $B$ \\
$X, Y, Z, \xi, \psi, \zeta$ & Випадкові величини (в.в.) \\
$F, F_X$ & Функція розподілу в.в. \\
$\E X, \D X$ & Математичне сподівання та дисперсія в.в. \\
$\Theta \ni \theta$ & Простір параметрів. $\theta = (\theta_1, ..., \theta_n) \in \Theta \subset \mathbb R^n$ \\
$\P_\theta(A),\;\E_\theta X,\;\D_\theta X$ & Позначення для $\P(A\mid \theta), \E(X \mid \theta), \D( X \mid \theta)$ \\
$\text{iid}$ & незалежні в сукупності однаково розподілені в.в. \\
$\text{a.s.}$ & майже скрізь; із ймовірністю 1 \\
$X_1, ..., X_n \sim F \quad \text{iid}$ & Послідовність незалежних однаково розподілених \\
~ & випадкових величин із законом розподілу $F$ \\
$\nu_a$ & Частота значення $a$ у послідовності iid; \\
~ & $\nu_a = \sum_{k=1}^n 1_{\{a\}}(X_k)$ \\

\end{tabular}


\newpage

\phantomsection
\chapter*{Вступ}\label{Introduction}
\addcontentsline{toc}{chapter}{Вступ}

??????????
\cite{boyko-thesis,boyko2021,PopovychBoykoNesterenkoLutfullin2003}

\newpage

\chapter{Назва розділу}\label{chaper1}

\section{Назва секції}\label{section1.1}

\newpage

\chapter{Назва розділу}\label{chaper2}

\section{Назва секції}\label{section2.1}


\newpage

\phantomsection
\chapter*{Висновки}
\addcontentsline{toc}{chapter}{Висновки}

У роботі ??????

\newpage

\phantomsection
\renewcommand{\bibname}{Список використаних джерел}

\begin{thebibliography}{99}
\addcontentsline{toc}{chapter}{Список використаних джерел}
\itemsep=0pt

\bibitem{boyko-thesis}
Бойко В.М.,
Узагальненi оператори Казiмiра,
сингулярнi модулi редукцiї
та симетрiї диференцiальних рiвнянь,
Дис. \dots\ док. фіз.-мат. наук,  Інституту математики НАН України, Київ, 2018, 338~с., \url{https://www.imath.kiev.ua/~boyko/BoykoThesis.pdf}.



\bibitem{boyko2021}
Boyko V.M., Lokaziuk O.V., Popovych R.O.,
Admissible transformations and Lie symmetries of linear systems of second-order ordinary differential equations, \href{https://arxiv.org/abs/2105.05139}{arXiv:2105.05139}.

\bibitem{Maple17}
Maple 17, \url{https://www.maplesoft.com/products/Maple/}.


\bibitem{Olver1995}
Olver P.J., Equivalence, invariants, and symmetry, Cambridge, University Press Cambridge, 1995, xvi+525~pp.,
\url{https://doi.org/10.1017/CBO9780511609565}.


\bibitem{PopovychBoykoNesterenkoLutfullin2003}
Popovych R.O., Boyko V.M., Nesterenko M.O., Lutfullin M.V., Realizations of real low-dimensional Lie algebras, \textit{J.~Phys.~A} \textbf{36} (2003), no.~26,
7337--7360, \url{https://doi.org/10.1088/0305-4470/36/26/309}; \href{https://arxiv.org/abs/math-ph/0301029}{math-ph/0301029}.



\end{thebibliography}

\appendix

\chapter{Назва додатку}\label{appendix1}

\section{Назва секції додатку}\label{appendix1.1}

\end{document}

\newpage

\bibliographystyle{plain} %plain %sigma %amsalpha %ugost2008
\bibliography{ref}



