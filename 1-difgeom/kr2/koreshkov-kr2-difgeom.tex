\documentclass[10pt, a4paper]{article} % use larger type; default would be 10pt

\usepackage{fontspec} % Font selection for XeLaTeX; see fontspec.pdf for documentation
\defaultfontfeatures{Mapping=tex-text} % to support TeX conventions like ``---''
\usepackage{xunicode} % Unicode support for LaTeX character names (accents, European chars, etc)
\usepackage{xltxtra} % Extra customizations for XeLaTeX
\usepackage{tikz}
% \usepackage{coloremoji}
\usetikzlibrary{arrows,calc,patterns}


% other LaTeX packages.....
\usepackage{fullpage}
\usepackage[top=2cm, bottom=4.5cm, left=2.5cm, right=2.5cm]{geometry}
\usepackage{amsmath,amsthm,amsfonts,amssymb,amscd,systeme}
\usepackage{unicode-math}
\usepackage{cancel}
\geometry{a4paper} 
\usepackage[parfill]{parskip} % Activate to begin paragraphs with an empty line rather than an indent
\usepackage{fancyhdr}
\usepackage{listings}
\usepackage{graphicx}
\usepackage{hyperref}
\usepackage{multicol}

\usepackage{xcolor}

% FONTS
% \setmainfont[Ligatures=TeX]{Cambria Math} % set the main body font (\textrm), assumes Charis SIL is installed
%\setsansfont{Deja Vu Sans}
% \setmonofont[Ligatures=TeX]{Fira Code}
\setmathfont[Ligatures=TeX]{NewCMMath-Regular}

\setmainfont{Cambria}
\setmonofont[Ligatures=TeX]{Roboto Mono}

\renewcommand\lstlistingname{Algorithm}
\renewcommand\lstlistlistingname{Algorithms}
\def\lstlistingautorefname{Alg.}
\lstdefinestyle{mystyle}{
    % backgroundcolor=\color{backcolour},   
    % commentstyle=\color{codegreen},
    % keywordstyle=\color{magenta},
    % numberstyle=\tiny\color{codegray},
    % stringstyle=\color{codepurple},
    basicstyle=\ttfamily\footnotesize,
    breakatwhitespace=false,         
    breaklines=true,                 
    captionpos=b,                    
    keepspaces=true,                 
    numbers=left,                    
    numbersep=5pt,                  
    showspaces=false,                
    showstringspaces=false,
    showtabs=false,                  
    tabsize=2
}
\lstset{style=mystyle}

\newcommand\course{MS1 - Differential geometry and topology}
\newcommand\hwnumber{Контрольна Робота 2}                   % <-- homework number
\newcommand\idgroup{111-2023}                
\newcommand\idname{Михайло Корешков}  

\usepackage[framemethod=TikZ]{mdframed}
\mdfsetup{%
	backgroundcolor = black!5,
    skipabove = 4pt,
}
\mdfdefinestyle{ans}{%
    backgroundcolor = green!5,
    linecolor = green!50,
    linewidth = 1pt,
}

\pagestyle{fancyplain}
\headheight 35pt
\lhead{\idgroup \\ \idname}
\chead{\textbf{\Large \hwnumber}}
\rhead{\course \\ \today}
\lfoot{}
\cfoot{}
\rfoot{\small\thepage}
\headsep 1.5em

\linespread{1.2}

\newcommand{\R}{\mathbb{R}}
\newcommand{\N}{\mathbb{N}}
\newcommand{\Z}{\mathbb{Z}}
\newcommand{\J}{J}
\DeclareMathOperator{\lcm}{lcm}
\DeclareMathOperator{\cd}{CD}
\DeclareMathOperator{\id}{id}
\DeclareMathOperator{\rank}{rank}

\renewcommand{\B}{\mathcal{B}}

\newtheorem*{definition}{Визначення}

\newcommand{\todo}[1]{\colorbox{red}{\textbf{TODO}: #1}}


\begin{document}

\subsection*{Ex 2.1.}
\begin{mdframed}
    $\gamma : (a;b) \to \R^n, \quad \gamma \in C^1$\\
    $\gamma'(t) \ne 0, \quad \left(\gamma(t), \gamma'(t)\right) = 0$

    Довести
    $\left|\gamma(t)\right| = \text{const}$
\end{mdframed}

Нехай $\gamma(t) = (\gamma_1(t), ..., \gamma_n(t))$.\\
Тоді
\begin{align*}
    \frac{d}{dt} \left|\gamma(t)\right| &= \frac{d}{dt} \sqrt{\left(\gamma(t), \gamma(t)\right)} = 
    \frac{1}{2\sqrt{\left(\gamma(t), \gamma(t)\right)}} \cdot \frac{d}{dt}\left(\gamma(t), \gamma(t)\right) = \\
    &= \frac{1}{2\sqrt{\left(\gamma(t), \gamma(t)\right)}} \cdot \left(\cancelto{0}{(\gamma'(t), \gamma(t))} + \cancelto{0}{(\gamma(t), \gamma'(t))}\right) = \\
    &= 0
\end{align*}


\section*{Ex 2.2}
\begin{mdframed}
    Нехай $\gamma : (a;b) \to \R^2, \quad \gamma C^2$ - пласка крива.\\
    Довести, що кривина $\gamma$ визначається як 
    \[k(t) = \frac{\begin{vmatrix}
        \dot x & \dot y \\ \ddot x & \ddot y
    \end{vmatrix}}{(\dot x^2 + \dot y^2)^{3/2}}\]
\end{mdframed}

Нехай $\gamma(t) : (a;b) \to \R^n$ - гладенька пласка крива.
Нехай $\phi(s) = \gamma(t(s))$ - натуральна параметризація $\gamma$. 
Тобто
\[|\frac{d\phi}{ds}| = 1, \quad (\frac{d^2\phi}{ds^2}, \frac{d\phi}{ds}) = 0, \quad k(s) := \left|\frac{d^2\phi}{ds^2}\right|\]

Нехай також $s'(t) > 0$.

Тоді 
\[\frac{d\phi}{ds} = \frac{d\gamma}{dt} \cdot \frac{dt}{ds} = \frac{\dot \gamma(t)}{s'(t)}\]
\begin{equation*}\label{eqn:2.2.1}
    |\dot \gamma(t)| = |s'(t)| = s'(t)
\end{equation*}

\begin{align*}
    \frac{d^2\phi}{ds^2} &= \frac{d}{ds}\frac{d\phi}{ds} = \frac{d}{dt}\left(\frac{d\phi}{ds}\right) \cdot \frac{1}{s'(t)} = \\
    &= \frac{d}{dt}\left(\frac{\dot \gamma(t)}{s'(t)}\right) \cdot \frac{1}{s'(t)} = 
    \frac{\ddot \gamma(t) s'(t) - \dot\gamma(t) s''(t)}{(s'(t))^3} = \\
    &= \frac{\ddot \gamma(t) |\dot \gamma(t)| - \dot\gamma(t) s''(t)}{|\dot \gamma(t)|^3}
\end{align*}
\begin{align*}
    \left(\frac{d^2\phi}{ds^2}, \frac{d\phi}{ds}\right) &= 
    \left(\frac{d^2\phi}{ds^2}, \frac{\dot \gamma(t)}{|\dot \gamma(t)|}\right) 
    = \left(\frac{\ddot \gamma(t) |\dot \gamma(t)| - \dot\gamma(t) s''(t)}{|\dot \gamma(t)|^3}, \frac{\dot \gamma(t)}{|\dot \gamma(t)|}\right) = \\
    &= \frac{(\ddot\gamma(t), \dot\gamma(t)) - |\dot\gamma(t)|\cdot s''(t)}{|\dot\gamma(t)|^3} = 0
\end{align*}
\begin{equation*}\label{eqn:2.2.2}
    (\ddot\gamma(t), \dot\gamma(t)) - |\dot\gamma(t)|\cdot s''(t) = 0; \quad |\dot\gamma(t)|s''(t) = (\ddot\gamma(t), \dot\gamma(t))
\end{equation*}

Далі
\begin{align*}
    k := \left|\frac{d^2\phi}{ds^2}\right|^2 &= 
    \left(\frac{\ddot \gamma(t) |\dot \gamma(t)| - \dot\gamma(t) s''(t)}{|\dot \gamma(t)|^3}, \frac{\ddot \gamma(t) |\dot \gamma(t)| - \dot\gamma(t) s''(t)}{|\dot \gamma(t)|^3}\right) =\\
    &= \frac{1}{|\dot \gamma(t)|^6} \left(  |\ddot\gamma(t)|^2|\dot\gamma(t)|^2 + |\dot\gamma(t)|^2|s''(t)|^2 - 2 (\ddot\gamma(t), \dot\gamma(t)) \cdot |\dot\gamma(t)|\cdot s''(t)  \right) = \\
    &= \frac{1}{|\dot \gamma(t)|^6} \left(  |\ddot\gamma(t)|^2|\dot\gamma(t)|^2 + (\ddot\gamma(t), \dot\gamma(t))^2 - 2 (\ddot\gamma(t), \dot\gamma(t))^2  \right) = \\
    &= \frac{1}{|\dot \gamma(t)|^6} \left(  |\ddot\gamma(t)|^2|\dot\gamma(t)|^2 - (\ddot\gamma(t), \dot\gamma(t))^2 \right)
\end{align*}
Тобто 
\[k(t) = \frac{\sqrt{|\dot\gamma|^2|\ddot\gamma|^2 - (\dot\gamma,\ddot\gamma)^2}}{|\dot\gamma|^3}\]

Нехай $\gamma : (a;b) \to \R^3$. \\
Використаємо те, що
$|A \times B|^2 + (A,B)^2 = |A|^2|B|^2(\sin^2(\angle(A,B)) + \cos^2(\angle(A,B))) = |A|^2|B|^2$\\
Тоді маємо
\[k(t) = \frac{|\ddot\gamma(t) \times \dot\gamma(t)|}{|\dot\gamma(t)|^3}\]

Для планарної кривої $\gamma : (a;b) \to \R^2$ можна вважати, що $\gamma(t) = (x(t), y(t), 0) : (a;b) \to \R^3$, а тоді
\[k(t) = \frac{1}{|\dot\gamma(t)|^3} \begin{vmatrix}
    \hat i & \hat j & \hat k \\
    \dot x & \dot y & 0 \\
    \ddot x & \ddot y & 0
\end{vmatrix} = \frac{\begin{vmatrix}\dot x & \dot y \\ \ddot x & \ddot y \end{vmatrix}}{\sqrt{\dot x^2 + \dot y^2}^3}\]

\section*{Ex 2.3}
\begin{mdframed}
    Обчислити кривини пласких кривих $\gamma: (a;b) \to \R^2$.\\
    Знайти точки та значення екстремумів кривини.
\end{mdframed}

\subsection*{Ex 2.3.1}
\[\gamma(t) = (t^2, t^3)\]

$\gamma'(t) = (2t, 3t^2)$\\
$\gamma''(t) = (2, 6t)$\\
$|[\gamma'(t), \gamma''(t)]| = 12t^2 - 6t^2 = 6t^2$\\
$|\gamma'(t)| = \sqrt{2t^2+9t^4}$

$k(t) = \frac{6t^2}{(2t^2+9t^4)^{3/2}} = 6t^2(2t^2+9t^4)^{-3/2}$

Шукаємо екстремуми

$k'(t) = \frac{12t(2t^2+9t^4)-9t^2(4t+36t^3)}{(2t^2+9t^4)^{3/2}} = 0$\\
$12t^3(1+18t^2) = 0$\\
$t=0$ або $1+18t^2=0$\\
$t\in \R$, отож лише $t=0$.\\
Але в цій точці похідна не визначена.

\begin{mdframed}[backgroundcolor=green!20]
    \[k(t) = \frac{6t^2}{(2t^2+9t^4)^{3/2}}\]
    Екстремумів немає

    Критична точка в $t=0, \gamma(0) = (0,0)$.
\end{mdframed}


\subsection*{Ex 2.3.2}
\[\gamma(t) = (\sin(t^2), \cos(t^2))\]

$\gamma(s) = (\sin(|s|), \cos(|s|))$\\
Це взагалі параметричне рівняння кола одиничного радіусу, 
що матиме кривину $k=1$ окрім точки $t=0$, але давайте доведемо.

Нехай $s>0$.
$\gamma'(s) = (\cos(s), -\sin(s))$\\
$\gamma''(s) = (-\sin(s), -\cos(s))$\\
$|[\gamma'(s), \gamma''(s)]| = |-\cos^2(s)-\sin^2(s)| = |-1| = 1$\\
$|\gamma'(s)| = \sqrt{\sin^2(s) + \cos^2(s)} = 1$\\
$|\gamma(s)|^{3} = 1$

$k(s) = \frac{1}{1} = 1$

Критична точка лише $k(0)$, в інших точках це константа. 

\begin{mdframed}[backgroundcolor=green!20]
    \[k(t) = 1, \; t\ne 0\]
    Екстремумів немає

    Критична точка в $t=0$
\end{mdframed}


\subsection*{Ex 2.3.3}
\[\gamma(t) = (2t, 3t-7)\]

$\gamma(t) = (2, 3) t + (0, -7)$. Маємо, що $gamma$ це пряма. Кривина 0. Доведемо це.

$\gamma'(t) = (2, 3)$\\
$\gamma''(t) = (0, 0)$\\
$|[\gamma'(t), \gamma''(t)]| = 0$\\
$|\gamma'(t)| = \sqrt{4+9} = \sqrt{13}$

$k(t) = \frac{0}{\sqrt{13}} = 0$,
$k'(t) = 0$\\

\begin{mdframed}[backgroundcolor=green!20]
    \[k(t) = 0\]
    Екстремумів немає
\end{mdframed}

\newpage
\subsection*{Ex 2.3.4}
\[\gamma(t) = (at, a/t), \quad \gamma : (0;+\infty) \to \R^2, \quad a>0\]

$\gamma'(t) = (a, -a/t^2)$\\
$\gamma''(t) = (0, 2a/t^3)$\\
$|[\gamma'(t), \gamma''(t)]| = 2a^2/t^3$\\
$|\gamma'(t)| = \sqrt{a^2+a^2/t^4}$

$k(t) = \frac{2a^2/t^3}{\sqrt{a^2+a^2/t^4}^3} = \frac{2a^2}{t^3(a^2+a^2/t^4)^{3/2}} = \frac{2a^2}{(a^2t^2+a^2/t^2)^{3/2}}$

$k'(s) = -\frac{3 a^2 (-((2 a^2)/t^3)+2 t)}{(a^2/t^2+t^2)^{5/2}} = 0 $\\
$t^4 = 1$\\
$t = \pm 1$. Але $t>0$, отже маємо лише $t=1$.
Цій точці відповідає $\gamma(t=1) = (a,a)$.

\begin{mdframed}[backgroundcolor=green!20]
    \[k(t) = \frac{2a^2}{(a^2t^2+a^2/t^2)^{3/2}} = \frac{2}{at^3(1+1/t^4)^{3/2}}\]

    Екстремум в точці $t=1$, $\gamma(1)=(a,a)$, $k(1) = \frac{1}{\sqrt{a}}$
\end{mdframed}


\subsection*{Ex 2.3.5}
\[\gamma(t) = (t, \sinh(t))\]

$\gamma'(t) = (1, \cosh(t))$\\
$\gamma''(t) = (0, \sinh(t))$\\
$|[\gamma'(t), \gamma''(t)]| = \sinh(t)$\\
$|\gamma'(t)| = \sqrt{1+\cosh^2(t)}$

$k(t) = \frac{\sinh(t)}{(1 + \cosh^2(t))^{3/2}}$

$k'(t) = \frac{\cosh(t)(1+\cosh^2(t))-3\cosh(t)\sinh^2(t)}{(1+\cosh^2(t))^{5/2}} 
= \frac{\cosh(t)(1+\cosh^2(t)-3\sinh^2(t))}{(1+\cosh^2(t))^{5/2}}=$
$= \frac{2\cosh(t)(1-\sinh^2(t))}{(1+\cosh^2(t))^{5/2}} = 0$

$(1-\sinh^2(t)) = 0$\\
$\sinh(t) = \pm 1$\\
$t = \operatorname{arsinh}(1) = \ln(1+\sqrt{2})$\\
$t = \operatorname{arsinh}(-1) = -\ln(1+\sqrt{2})$\\

\begin{mdframed}[backgroundcolor=green!20]
    \[k(t) = \frac{\sinh(t)}{(1 + \cosh^2(t))^{3/2}}\]
    Екстремуми в $t = \pm\ln(1+\sqrt{2}) = \pm\operatorname{arsinh}(1)$.
\end{mdframed}

\newpage
\subsection*{Ex 2.3.6}
\[\gamma(t) = (t, \cosh(t))\]

$\gamma'(t) = (1, \sinh(t))$\\
$\gamma''(t) = (0, \cosh(t))$\\
$|[\gamma'(t), \gamma''(t)]| = \cosh(t)$\\
$|\gamma'(t)| = \sqrt{1+\sinh^2(t)} = \cosh(t)$

$k(t) = \frac{\cosh(t)}{\cosh^3(t)} = (\cosh(t))^{-2}$

$k'(t) = -2(\cosh(t))^{-3}\cdot \sinh(t) = 0$

$\cosh(t) \ne 0$, $\sinh(t) = 0 \implies t = 0$.

\begin{mdframed}[backgroundcolor=green!20]
    \[k(t) = \frac{1}{\cosh^2(t)}\]
    Екстремум в $t = 0, \gamma(0) = (0,1)$, $k(0)=1$.
\end{mdframed}


% ===========================================

\section*{Ex 2.4}
\begin{mdframed}
    $f : \R^2 \to \R, \quad f\in C^2$\\
    Нехай $c\in \R, A = f^{-1}(c)$ та $A$ не містить критичних точок.
    Тобто $\forall p\in A: \exists f'_x(p), f'_y(p) \land f'_x(p)f'_y(p) \ne 0$.

    Довести що 
    \begin{enumerate}
        \item A - одновимірний підмноговид в $\R^2$
        \item Кривина кривої A визначається як
        \[k(p) = \frac{f'^T \cdot Hf \cdot f'}{|f'|^3}\]
        де
        \[f' = \begin{pmatrix}
            f'_x \\ f'_y
        \end{pmatrix}, \quad Hf = \begin{pmatrix}
            f''_{xx} & f''_{xy}\\ f''_{yx} & f''_{yy}
        \end{pmatrix}\]
    \end{enumerate}
\end{mdframed}

\subsection*{Ex 2.4.1}
Прообраз регулярного значення функції буде підмноговидом.
Покажемо це в цьому випадку.

Нехай $c\in \R$ - регулярне значення $f$.
Тоді $f'_x, f'_y$ визначені та не дорівнюють нулю одночасно.

Зверну увагу, що $f'$ буде градієнтом і $\forall p \in A: (f', A) = 0$, 
бо градієнт перпендикулярний лініям рівню.

Візьмемо довільну точку $p \in A = f^{-1}(c)$

$df_p : T\R^2 \to T\R, \quad df_p = Jf^T$
$df_p$ - сюр'єктивне, бо має максимальний ранг (ранг 1, бо рядок не обертається в 0).

Виберемо дотичний вектор $v_p$, що задовільняє умовам $|v_p|=1, (v_p,f'(p))=0$. 
Він буде єдиний з точністю до напрямку (тобто заміни $v \to -v$).

Ядро $\ker df_p$ буде одновимірним лінійним простором 
$\ker df_p = \left<v_p\right>=\{tv_p : t\in\R\}$. 
Геометрично це буде лінія, дотична до лінії рівня, перпендикулярна до градієнту.

Візьмемо достатньо малий окіл $p \in U_p \subset \R^2$ такий, що
$\forall q \in U_p: (f'(p), f'(q)) \ne 0$. 
Існування такого околу очевидне, але як саме це довести я поки не знаю.

Нехай $L_p: \R^2 \to \R, \quad L_p = v_p^T$. 
Зверну увагу, що $L_p(\ker df_p) = \R$, тобто $L_p$ - сюр'єкція і має повний ранг.

Нехай $\phi: U_p \to \R \times \R, \quad \phi(q) = (f(q); L_p q) = (f(q); v_p^Tq)$.\\
$d\phi_p(w) = (df_p(w); L_p w)$

$w \in \ker df_p \implies d\phi_p(w) = (0; v_p^T w) \ne 0\;$ за вибором $\;v_p$\\
$w \notin \ker df_p \implies d\phi_p(w) = (df_p(w); L_p w) \ne 0, \;$ бо $\;df_p(w) \ne 0$

З цього та лінійності $df_P$ та $L_p$, $d\phi_p$ буде лінійним ізоморфізмом.
Тоді, за теоремою про локальний дифеоморфізм, $\phi$ це локальний дифеоморфізм,
при чому $\phi(A \cap U_p) = \{c\} \times \R$.

Тобто, для кожної точки $p\in A$ існує окіл $U$ та локальна карта $(U, \phi)$, 
при чому одна координата $\phi(A\cap U)$ є константою, що і доводить, що $A$ - одновимірний підмноговид.

% ідея доведення взята з
% \href{https://math.stackexchange.com/questions/1824631/understanding-milnors-proof-of-the-fact-that-the-preimage-of-a-regular-value-is}{math.stackexchange.com}

% \newpage
\subsection*{Ex 2.4.2}

Знаємо, що $A = f^{-1}(c)$ - одновимірний підмноговид $\R^2$ aka гладенька крива.
Розглянемо нормальну параметризацію цієї кривої $\gamma$ таку що
\[\begin{array}{ll}
    \gamma : (a;b) \to A
    & \gamma(t) := (x(t), y(t))\\
    f(\gamma(t)) \equiv c
    & |\dot\gamma| = 1\\
    (\ddot\gamma, \dot\gamma) = 0
    & k(t) := |\ddot\gamma(t)|
\end{array}\]

\[\nabla f := \begin{pmatrix}
    f'_x & f'_y
\end{pmatrix}^T\]
Визначимо наступне 
\[N := \frac{\nabla f}{|\nabla f|}; \quad T := \frac{\begin{pmatrix}-f'_y & f'_x\end{pmatrix}^T}{|\nabla f|}\]
Тоді
$|T|=1$ та $(N,T)=0$

Також $\frac{d}{dt}f(\gamma(t)) = \nabla f(\gamma(t)) \cdot \dot \gamma(t) = |\nabla f|N(t) \cdot \dot\gamma(t) = 0$,  тобто $(N(t), \dot\gamma(t)) = 0$.

А отже $T(t) \parallel \dot\gamma(t)$. Також знаємо, що $|T(t)|=|\dot\gamma(t)|=1$.\\
Отже $T=\dot\gamma$ з точністю до напрямку. \\
Тоді $\frac{d}{dt}T(t)=\frac{d}{dt}\dot\gamma(t)=\ddot\gamma(t)$\\
Та $N(t) \parallel \ddot \gamma(t)$.\\
Звісно також $(T, \frac{d}{dt}T) = 0$

\[\ddot\gamma(t) = |\ddot\gamma(t)| N(t) = k(t)N(t)\]

\[\frac{d}{dt}f(\gamma(t)) = \begin{pmatrix}f'_x & f'_y\end{pmatrix}_{\gamma(t)} \begin{pmatrix}\dot x(t) \\ \dot y(t)\end{pmatrix} = \nabla f(\gamma(t))^T \dot \gamma(t) = \nabla f^T\dot \gamma(t) = 0\]
\[\frac{d}{dt}\nabla f(\gamma(t)) = \frac{d}{dt} \begin{pmatrix}
        f'_x(x(t), y(t)) \\ f'_y(x(t), y(t))
    \end{pmatrix} = Hf(\gamma(t)) \dot \gamma(t) = \begin{pmatrix}
        f''_{xx} & f''_{xy} \\ f''_{yx} & f''_{yy}
    \end{pmatrix}_{\gamma(t)}\begin{pmatrix}
        \dot x(t) \\ \dot y(t)
    \end{pmatrix}
\]
\[\frac{d^2}{dt^2}(f(\gamma(t))) = \frac{d(\nabla f^T)}{dt} \dot \gamma(t) + \nabla f(\gamma(t))^T \ddot \gamma(t) = \dot \gamma(t)^T (Hf)^T \dot \gamma(t) + \nabla f^T \ddot \gamma(t) = 0\]

Замінимо $\dot\gamma \to T, \;\; \nabla f \to |\nabla f|N, \;\; \ddot \gamma \to k \cdot N$

\[T(t)^T (Hf)^T T(t) + |\nabla f|N(t)^T N(t) k(t) = 0\]
\[k(t) = \frac{T(t)^T (Hf) T(t)}{|\nabla f||N(t)|^2} = \frac{T(t)^T (Hf) T(t)}{|\nabla f|} = \frac{\frac{1}{|\nabla f|}\begin{pmatrix}-f'_y & f'_x\end{pmatrix} (Hf) \frac{1}{|\nabla f|}\begin{pmatrix}-f'_y \\ f'_x\end{pmatrix}}{|\nabla f|} \]

\[k(t) = \frac{\begin{pmatrix}-f'_y & f'_x\end{pmatrix} (Hf) \begin{pmatrix}-f'_y \\ f'_x\end{pmatrix}}{|\nabla f|^3} \]



% \[\frac{d}{dt}|\nabla f(\gamma(t))|^2 = 2\left(\frac{d}{dt}\nabla f(\gamma(t)), \nabla f(\gamma(t))\right) = 2\left((Hf(\gamma(t))) \dot \gamma(t), \nabla f(\gamma(t))\right)\]
% \[\frac{d}{dt}|\nabla f(\gamma(t))| = \frac{1}{2|\nabla f(\gamma(t))|} \cdot \frac{d}{dt}|\nabla f(\gamma(t))|^2 = \left((Hf(\gamma(t))) \dot \gamma(t), \frac{\nabla f}{|\nabla f|}\right)\]
% \[\frac{d}{dt}|\nabla f(\gamma(t))| = \left((Hf) T(t), N(t)\right)\]

% \[\frac{d}{dt}\begin{pmatrix}
%     -f'_y \\ f'_x 
% \end{pmatrix} = \begin{pmatrix}
%     -f''_{yx}\dot x - f''_{yy}\dot y \\ 
%     f''_{xx}\dot x + f''_{xy}\dot y 
% \end{pmatrix}\]

% \[k(t) = \frac{\sqrt{|\dot\gamma|^2|\ddot\gamma|^2 - (\dot\gamma,\ddot\gamma)^2}}{|\dot\gamma|^3}\]
% \[k(t) = \frac{\sqrt{|T|^2|\frac{d}{dt}T|^2 - (T,\frac{d}{dt}T)^2}}{|T|^3}\]

% Використаємо те, що $T=\dot\gamma$.
% \[T^T(Hf)T+\nabla f^T\ddot\gamma(t) = 0\]

\newpage
\section*{Ex 2.5}
\subsection*{2.5.1}
\begin{mdframed}
    $f(x,y) = x^2+y^2, \quad c = r^2$
\end{mdframed}
Маємо коло радіусу $r$. Знаємо що кривина буде константною і рівною $1/r$. Доведемо це.

\[\nabla f = \begin{pmatrix} 2x & 2y\end{pmatrix}; \quad |\nabla f| = 2\sqrt{x^2+y^2} = 2r\]
\[Hf = \begin{pmatrix}
    2 & 0\\ 0 & 2
\end{pmatrix}\]
\[k = \frac{1}{8r^3}\begin{pmatrix} -2y & 2x\end{pmatrix}\begin{pmatrix}
    2 & 0\\ 0 & 2
\end{pmatrix}\begin{pmatrix} -2y \\ 2x\end{pmatrix} = \]
\[= \frac{8y^2 + 8x^2}{8r^3} = \frac{r^2}{r^3} = \frac{1}{r}\]

\[k(p) = \frac{1}{r}\]
Без екстремумів, бо константа

\subsection*{2.5.2}
\begin{mdframed}
    $f(x,y) = xy, \quad c \ne 0$
\end{mdframed}

\[xy=c; \quad y=\frac{c}{x}\]

\[\nabla f = \begin{pmatrix} y & x\end{pmatrix} = \begin{pmatrix} \frac{c}{x} & x\end{pmatrix}\]
\[|\nabla f| = \sqrt{\frac{c^2}{x^2} + x^2}\]
\[Hf = \begin{pmatrix}
    0 & 1 \\ 1 & 0
\end{pmatrix}\]
\[k(x) = \frac{1}{\left(\frac{c^2}{x^2} + x^2\right)^{3/2}} \begin{pmatrix} -x & y\end{pmatrix} \begin{pmatrix}
    0 & 1 \\ 1 & 0
\end{pmatrix} \begin{pmatrix} -x \\ y\end{pmatrix} = \frac{2xy}{\left(\frac{c^2}{x^2} + x^2\right)^{3/2}} = \frac{2c}{\left(\frac{c^2}{x^2} + x^2\right)^{3/2}}\]

$k$ парна функція та має критичну точку в $x=0$. Розглядаємо $x>0$. 

\[k'(x) = \frac{x^4-c^2}{...} = 0, \quad x=\pm \sqrt{c}\]
\[k(\sqrt{c}) = \frac{1}{\sqrt{c}}\]

Отже
\[k(x, c/x) = \frac{2c}{\left(\frac{c^2}{x^2} + x^2\right)^{3/2}}\]
Екстремуми
\[k(\pm\sqrt{c}) = \frac{1}{\sqrt{c}}, \quad \text{в точці} \; (\sqrt{c}, \sqrt{c})\] 


\subsection*{2.5.3}
\begin{mdframed}
    $f(x,y) = x^3 - y^2, \quad c \ne 0$
\end{mdframed}

\[ x^3 - y^2=c; \quad y^2 = x^3-c; \quad x^3 = y^2+c; \quad y=\pm\sqrt{x^3-c}\]

\[\nabla f = \begin{pmatrix} 3x^2 & -2y\end{pmatrix}\]
\[|\nabla f| = \sqrt{9x^4+4y^2}\]
\[Hf = \begin{pmatrix}
    6x & 0 \\ 0 & -2
\end{pmatrix}\]
\[k(x,y) = \frac{1}{(9x^4+4y^2)^{3/2}} \begin{pmatrix} 2y & 3x^2\end{pmatrix} \begin{pmatrix}
    6x & 0 \\ 0 & -2
\end{pmatrix} \begin{pmatrix} 2y & 3x^2\end{pmatrix} = \frac{54xy^2 - 8x^4}{(9x^4+4y^2)^{3/2}}\]
\[k(x) = \frac{54x(x^3-c) - 8x^4}{(9x^4+4(x^3-c))^{3/2}} 
= \frac{54x^4-xc - 8x^4}{(9x^4+4x^3-4c)^{3/2}} = \frac{46x^4-xc}{(9x^4+4x^3-4c)^{3/2}}\]

\section*{Ex 2.6}

Формули Френе:
\[{\displaystyle {\frac {\mathrm {d} }{\mathrm {d} t}}{\begin{bmatrix}\mathbf {T} \\\mathbf {N} \\\mathbf {B} \end{bmatrix}}=\|\gamma'(t)\|{\begin{bmatrix}0&\kappa &0\\-\kappa &0&\tau \\0&-\tau &0\end{bmatrix}}{\begin{bmatrix}\mathbf {T} \\\mathbf {N} \\\mathbf {B} \end{bmatrix}}}\]
\[k = \frac{|\dot\gamma \times \ddot\gamma|}{|\dot\gamma|^3}, \quad \tau = \frac{\det(\dot\gamma, \ddot\gamma, \dddot\gamma)}{|\dot\gamma \times \ddot\gamma|^2}\]

Також
\[
T = \frac{\dot\gamma}{|\dot\gamma|}; \quad
B = \frac{\dot\gamma(t) \times \ddot\gamma(t)}{|\dot\gamma(t) \times \ddot\gamma(t)|}; \quad
N = B \times T
\]

\subsection*{2.6.1}
\begin{mdframed}
    $\gamma(t) = \begin{pmatrix}
        t & t^2 & t^3
    \end{pmatrix}$
\end{mdframed}

$\dot\gamma = \begin{pmatrix}1 & 2t & 3t^2\end{pmatrix}$.\\
$\ddot\gamma = \begin{pmatrix}0 & 2 & 6t\end{pmatrix}$.\\
$\dddot\gamma = \begin{pmatrix}0 & 0 & 6\end{pmatrix}$\\
$|\dot\gamma| = \sqrt{1+4t^2+9t^4}$

\[A = \dot\gamma(t) \times \ddot\gamma(t) = \begin{vmatrix}
    \hat i & \hat j & \hat k \\ 0 & 2 & 6t \\ 0 & 0 & 6
\end{vmatrix} = \begin{pmatrix}
    12 & 0 & 0
\end{pmatrix}\]
\[T(t) = \frac{\dot \gamma}{|\dot\gamma|} = \frac{\begin{pmatrix}1 & 2t & 3t^2\end{pmatrix}}{\sqrt{1+4t^2+9t^4}}\]
\[B(t) = \frac{A}{|A|} = \begin{pmatrix}
    1 & 0 & 0
\end{pmatrix}\]
\[N(t) = \begin{vmatrix}
    \hat i & \hat j & \hat k \\ 1 & 0 & 0 \\ 1 & 2t & 3t^2
\end{vmatrix} = \begin{pmatrix}
    0 & -3t^2 & 2t
\end{pmatrix}\]

\[k = \frac{|A|}{|\dot\gamma|^3} = \frac{12}{(1+4t^2+9t^4)^{3/2}}\]
\[\tau = \frac{\det(\dot\gamma, \ddot\gamma, \dddot\gamma)}{12^2} = \frac{1}{144}\begin{vmatrix}
    1 & 2t & 3t^2 \\ 0 & 2 & 6t \\ 0 & 0 & 6
\end{vmatrix} = \frac{12}{144} = \frac{1}{12}\]

\subsection*{2.6.2}
\begin{mdframed}
    $\gamma(t) = \begin{pmatrix}
        \cos(at) & \sin(at) & bt
    \end{pmatrix}$
\end{mdframed}

$\dot\gamma = \begin{pmatrix}-a\sin(at) & a\cos(at) & b\end{pmatrix}$.\\
$\ddot\gamma = \begin{pmatrix}-a^2\cos(at) & -a^2\sin(at) & 0\end{pmatrix}$.\\
$\dddot\gamma = \begin{pmatrix}a^3\sin(at) & -a^3\cos(at) & 0\end{pmatrix}$\\
$|\dot\gamma| = \sqrt{a^2+b^2}$

\[A = \dot\gamma(t) \times \ddot\gamma(t) = \begin{vmatrix}
    \hat i & \hat j & \hat k  \\ -a\sin(at) & a\cos(at) & b \\ -a^2\cos(at) & -a^2\sin(at) & 0
\end{vmatrix} = \begin{pmatrix}
    a^2b\sin(at) & -a^2b\cos(at) & a^3\cos^2(at)-a^3\sin^2(at)
\end{pmatrix} = \]
\[= \begin{pmatrix}
    a^2b\sin(at) & -a^2b\cos(at) & a^3
\end{pmatrix} \]
\[|A| = \sqrt{a^4b^2+a^6} = a^2\sqrt{a^2+b^2}\]
\[T(t) = \frac{\dot \gamma}{|\dot\gamma|} = \frac{\begin{pmatrix}-a\sin(at) & a\cos(at) & b\end{pmatrix}}{\sqrt{a^2+b^2}}\]
\[B(t) = \frac{A}{|A|} = \frac{1}{\sqrt{a^4b^2+a^6}}\begin{pmatrix}
    a^2b\sin(at) & -a^2b\cos(at) & a^3
\end{pmatrix} = \frac{1}{\sqrt{a^2+b^2}}\begin{pmatrix}
    b\sin(at) & -b\cos(at) & a
\end{pmatrix}\]
\[N(t) = \frac{1}{a^2+b^2}\begin{vmatrix}
    \hat i & \hat j & \hat k \\ b\sin(at) & -b\cos(at) & a \\ -a\sin(at) & a\cos(at) & b
\end{vmatrix} = \frac{1}{a^2+b^2}\begin{pmatrix}
    -\cos(at)(a^2+b^2) & -\sin(at)(a^2+b^2) & 0
\end{pmatrix} = \]
\[= \begin{pmatrix}
    -\cos(at) & -\sin(at) & 0
\end{pmatrix}\]
Це взагалі-то щастя, бо нормальний вектор дійсно має бути напрямленим до осі гвинтової лінії.

\[k = \frac{|A|}{|\dot\gamma|^3} = \frac{a^2\sqrt{a^2+b^2}}{\sqrt{a^2+b^2}^3} = \frac{a^2}{a^2+b^2}\]
\[\tau = \frac{\det(\dot\gamma, \ddot\gamma, \dddot\gamma)}{|A|^2} = \frac{1}{a^2+b^2}\begin{vmatrix}
    -a\sin(at) & a\cos(at) & b \\ -a^2\cos(at) & -a^2\sin(at) & 0 \\ a^3\sin(at) & -a^3\cos(at) & 0
\end{vmatrix} = \frac{b(a^5\cos^2(at)+a^5\sin^2(at))}{a^2+b^2} = \frac{a^5b}{a^2+b^2}\]

\newpage
\subsection*{2.6.3}
\begin{mdframed}
    $\gamma(t) = \begin{pmatrix}
        t\cos(at) & t\sin(at) & bt
    \end{pmatrix}$
\end{mdframed}

Далі я дещо рахував у Wolfram Cloud бо це капець

$\dot\gamma = \begin{pmatrix}-at\sin(at)+\cos(at) & at\cos(at)+\sin(at) & b\end{pmatrix}$.\\
$\ddot\gamma = (wolfram) = \begin{pmatrix}-a^2 t \cos(a t) - 2a \sin(a t) & 2a \cos(a t)-a^2 t \sin(a t) & 0 \end{pmatrix}$.\\
$\dddot\gamma = (wolfram) = \begin{pmatrix}-3 a^2 \cos(a t)+a^3 t \sin(a t) & -a^3 t \cos(a t)-3 a^2 \sin(a t) & 0 \end{pmatrix}$\\
$|\dot\gamma| = \sqrt{b^2 + a^2t^2\cos^2(at)+\sin^2(at)+2at\sin(at)\cos(at)+\cos^2(at)+a^2t^2\sin^2(at)-2at\sin(at)\cos(at)} =$\\ 
$=\sqrt{b^2 + a^2t^2 + 1}$

\[A = \dot\gamma(t) \times \ddot\gamma(t) = \begin{vmatrix}
    \hat i & \hat j & \hat k  \\ -at\sin(at)+\cos(at) & at\cos(at)+\sin(at) & b \\ -a^2 t \cos(a t) - 2a \sin(a t) & 2a \cos(a t)-a^2 t \sin(a t) & 0
\end{vmatrix} = \]
\[= (wolfram) = \begin{pmatrix}
    -2 a b \cos(a t)+a^2 b t \sin(a t) \\
    -a^2 b t \cos(a t)-2 a b \sin(a t) \\
    2 a \cos(a t)^2+a^3 t^2 \cos(a t)^2+2 a \sin(a t)^2+a^3 t^2 \sin(a t)^2
\end{pmatrix} = \]
\[= \begin{pmatrix}
    -2 a b \cos(a t)+a^2 b t \sin(a t) & -a^2 b t \cos(a t)-2 a b \sin(a t) & 2 a + a^3t^3
\end{pmatrix}\]
\[|A| = (wolfram) =  a \sqrt{(2+a^2 t^2)^2+b^2 (4+a^2 t^2)} \]
\[T(t) = \frac{\dot \gamma}{|\dot\gamma|} = \frac{\begin{pmatrix}-at\sin(at)+\cos(at) & at\cos(at)+\sin(at) & b\end{pmatrix}}{\sqrt{b^2 + a^2t^2 + 1}}\]
\[B(t) = \frac{A}{|A|} = \frac{1}{a \sqrt{(2+a^2 t^2)^2+b^2 (4+a^2 t^2)}}\begin{pmatrix}
    -2 a b \cos(a t)+a^2 b t \sin(a t) & -a^2 b t \cos(a t)-2 a b \sin(a t) & 2 a + a^3t^3
\end{pmatrix} = \]
\[=\frac{1}{\sqrt{(2+a^2 t^2)^2+b^2 (4+a^2 t^2)}}\begin{pmatrix}
    -2 b \cos(a t)+a b t \sin(a t) & -a b t \cos(a t)-2 b \sin(a t) & 2 + a^2t^3
\end{pmatrix}\]

\[N(t) = B \times T = (wolfram) = \frac{\begin{pmatrix}
    -a t (2+b^2+a^2 t^2) \cos(a t)-(2+2 b^2+a^2 t^2) \sin(a t)\\
    (2+2 b^2+a^2 t^2) \cos(a t)-a t (2+b^2+a^2 t^2) \sin(a t)\\
    -a b t
\end{pmatrix}}{\sqrt{(1+b^2+a^2 t^2)((2+a^2 t^2)^2+b^2 (4+a^2 t^2))}}\]

\[k = \frac{|A|}{|\dot\gamma|^3} = \frac{a \sqrt{(2+a^2 t^2)^2+b^2 (4+a^2 t^2)}}{\sqrt{b^2 + a^2t^2 + 1}^3}\]
\[\tau = \frac{\det(\dot\gamma, \ddot\gamma, \dddot\gamma)}{|A|^2} = (wolfram) = \frac{a^3 b (6+a^2 t^2) }{b^2 + a^2t^2 + 1}\]


\end{document}

