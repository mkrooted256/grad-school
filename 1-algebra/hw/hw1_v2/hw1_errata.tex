\documentclass[11pt, a4paper]{article} % use larger type; default would be 10pt

\usepackage{fontspec} % Font selection for XeLaTeX; see fontspec.pdf for documentation
\defaultfontfeatures{Mapping=tex-text} % to support TeX conventions like ``---''
\usepackage{xunicode} % Unicode support for LaTeX character names (accents, European chars, etc)
\usepackage{xltxtra} % Extra customizations for XeLaTeX
\usepackage{tikz}
\usetikzlibrary{arrows,calc,patterns}


% other LaTeX packages.....
\usepackage{fullpage}
\usepackage[top=2cm, bottom=4.5cm, left=2.5cm, right=2.5cm]{geometry}
\usepackage{amsmath,amsthm,amsfonts,amssymb,amscd,systeme}
\usepackage{unicode-math}
\usepackage{cancel}
\geometry{a4paper} 
%\usepackage[parfill]{parskip} % Activate to begin paragraphs with an empty line rather than an indent
\usepackage{fancyhdr}
\usepackage{listings}
\usepackage{graphicx}
\usepackage{hyperref}
\usepackage{multicol}

% FONTS
% \setmainfont[Ligatures=TeX]{Cambria Math} % set the main body font (\textrm), assumes Charis SIL is installed
%\setsansfont{Deja Vu Sans}
% \setmonofont[Ligatures=TeX]{Fira Code}
\setmathfont[Ligatures=TeX]{NewCMMath-Regular}

\setmainfont{Cambria}
\setmonofont[Ligatures=TeX]{Roboto Mono}

\renewcommand\lstlistingname{Algorithm}
\renewcommand\lstlistlistingname{Algorithms}
\def\lstlistingautorefname{Alg.}
\lstdefinestyle{mystyle}{
    % backgroundcolor=\color{backcolour},   
    % commentstyle=\color{codegreen},
    % keywordstyle=\color{magenta},
    % numberstyle=\tiny\color{codegray},
    % stringstyle=\color{codepurple},
    basicstyle=\ttfamily\footnotesize,
    breakatwhitespace=false,         
    breaklines=true,                 
    captionpos=b,                    
    keepspaces=true,                 
    numbers=left,                    
    numbersep=5pt,                  
    showspaces=false,                
    showstringspaces=false,
    showtabs=false,                  
    tabsize=2
}
\lstset{style=mystyle}

\newcommand\course{MS1 - Algebra}
\newcommand\hwnumber{HW 1. Додаток 1}                   % <-- homework number
\newcommand\idgroup{111-2023}                
\newcommand\idname{Mykhailo Koreshkov}  

\usepackage[framemethod=TikZ]{mdframed}
\mdfsetup{%
	backgroundcolor = black!5,
}
\mdfdefinestyle{ans}{%
    backgroundcolor = green!5,
    linecolor = green!50,
    linewidth = 1pt,
}

\pagestyle{fancyplain}
\headheight 35pt
\lhead{\idgroup \\ \idname}
\chead{\textbf{\Large \hwnumber}}
\rhead{\course \\ \today}
\lfoot{}
\cfoot{}
\rfoot{\small\thepage}
\headsep 1.5em

\linespread{1.2}

\newcommand{\R}{\mathbb{R}}
\newcommand{\N}{\mathbb{N}}
\newcommand{\Z}{\mathbb{Z}}
\DeclareMathOperator{\lcm}{lcm}
\DeclareMathOperator{\cd}{CD}


\begin{document}


\section*{Ex 1.1, пункт 2}
\begin{mdframed}
    Чи існує найбільший елемент відносно подільності як відношення (строгого) порядку на $\N$?
\end{mdframed}

\begin{mdframed}[backgroundcolor=blue!10]
    \[\forall a,b \in \N: \quad a|b \iff \exists c\in \N: b=ac \]
\end{mdframed}

$g$ - найбільший елемент, якщо
\[\forall x \in \N: x \mid g \]

Я стверджую, що $g=0$ і хочу це перевірити. Нехай $n\in\N$ - довільне.
\[0 = 0 \cdot n\]
Тобто, для довільного $n$ знайшли таке число $k\in\N$ що $0 = kn$.

Отже, $0$ - найбільший елемент відносно відношення подільності.

Перевіримо, чи є інші найбільші елементи (не мають бути через властивості часткового порядку).
Припустимо, що $g\in\N, g\ne 0$ - також найбільший елемент.\\
Тоді $g\mid 0 \land 0 \mid g$. Тобто все-таки $g=0$.

Отже, $0$ - єдиний найбільший елемент відносно відношення подільності.

\section*{№1.3, уточнення 2->1}
\begin{mdframed}
    Нехай $a,b,n \in \Z, \; n>0$.
    Довести, що твердження еквівалентні:
    \begin{enumerate}
        \item $a \mod n = b \mod n$
        \item $\exists k\in\Z: a-b = nk$
    \end{enumerate}
\end{mdframed}

$2 \implies 1$
\begin{proof}
    Let $a,b: \; \exists k\in\Z: a-b = nk$.
    Можемо записати результат ділення з остачею на $n$:
    \begin{align*}
        a &= sn + r \\
        b &= tn + p
    \end{align*}
    Розглянемо різницю
    \[a-b = sn+r - tn - p = (s-t)n + (r-p)\]
    \[n\;|\;\left((s-t)n + (r-p)\right)\]
    \[\exists l\in\Z: (s-t)n + (r-p) = nl\]
    \[(r-p) = nl - (s-t)n = n(l-s+t) = nh, h\in\Z\]
    Але $|r-p| < n$ бо $0<r,p<n$ як лишок ділення.
    З іншого боку, $n\ne 0$ за умовою.
    Значить $h=0$, $r-p=0$ та $r=p$.
\end{proof}

\section*{№1.9 уточнення}
\begin{mdframed}
    Let $\gcd(a,b)=1$.
    Довести
    \[a\mid c \land b\mid c \implies ab\mid c\]
\end{mdframed}

\begin{proof}
    Припустимо, що $c=0$. Тоді $ab \mid c$ для будь яких $a,b\in\Z$. 
    Надалі вважаю $c\ne 0$.

    Нехай $\gcd(a,b) = 1, c\in\Z$. $\exists x,y,\alpha,\beta\in\Z:$
    \begin{gather*}
        ax+by=1\\
        c=\alpha a\\
        c=\beta b
    \end{gather*}

    Тоді
    \begin{gather*}
        \alpha\beta ax + \alpha\beta by = \alpha\beta\\
        \alpha a\cdot \beta x + \beta b \cdot \alpha y = \alpha\beta\\
        c(\beta x + \alpha y) = \alpha\beta\\
        \alpha\beta = c\gamma, \quad (\gamma = \beta x + \alpha y)\\
        c^2 = \alpha\beta ab = c \gamma ab, \quad (c\ne 0)\\
        c = \gamma ab
    \end{gather*}

    Отже, $ab\mid c$

\end{proof}

\end{document}

