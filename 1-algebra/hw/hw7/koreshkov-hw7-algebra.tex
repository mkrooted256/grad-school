\documentclass[11pt, a4paper]{article} % use larger type; default would be 10pt

\usepackage{fontspec} % Font selection for XeLaTeX; see fontspec.pdf for documentation
\defaultfontfeatures{Mapping=tex-text} % to support TeX conventions like ``---''
\usepackage{xunicode} % Unicode support for LaTeX character names (accents, European chars, etc)
\usepackage{xltxtra} % Extra customizations for XeLaTeX
\usepackage{tikz}
\usetikzlibrary{arrows,calc,patterns}


% other LaTeX packages.....
\usepackage{fullpage}
\usepackage[top=2cm, bottom=4.5cm, left=2.5cm, right=2.5cm]{geometry}
\usepackage{amsmath,amsthm,amsfonts,amssymb,amscd,systeme}
\usepackage{unicode-math}
\usepackage{cancel}
\geometry{a4paper} 
%\usepackage[parfill]{parskip} % Activate to begin paragraphs with an empty line rather than an indent
\usepackage{fancyhdr}
\usepackage{listings}
\usepackage{graphicx}
\usepackage{hyperref}
\usepackage{multicol}


\usepackage{faktor}


% FONTS
% \setmainfont[Ligatures=TeX]{Cambria Math} % set the main body font (\textrm), assumes Charis SIL is installed
%\setsansfont{Deja Vu Sans}
% \setmonofont[Ligatures=TeX]{Fira Code}
\setmathfont[Ligatures=TeX]{NewCMMath-Regular}

\setmainfont{Cambria}
\setmonofont[Ligatures=TeX]{Roboto Mono}

\renewcommand\lstlistingname{Algorithm}
\renewcommand\lstlistlistingname{Algorithms}
\def\lstlistingautorefname{Alg.}
\lstdefinestyle{mystyle}{
    % backgroundcolor=\color{backcolour},   
    % commentstyle=\color{codegreen},
    % keywordstyle=\color{magenta},
    % numberstyle=\tiny\color{codegray},
    % stringstyle=\color{codepurple},
    basicstyle=\ttfamily\footnotesize,
    breakatwhitespace=false,         
    breaklines=true,                 
    captionpos=b,                    
    keepspaces=true,                 
    numbers=left,                    
    numbersep=5pt,                  
    showspaces=false,                
    showstringspaces=false,
    showtabs=false,                  
    tabsize=2
}
\lstset{style=mystyle}

\hypersetup{
  colorlinks   = true, %Colours links instead of ugly boxes
  urlcolor     = blue, %Colour for external hyperlinks
  linkcolor    = blue, %Colour of internal links
  citecolor   = red %Colour of citations
}


\newcommand\course{MS1 - Algebra}
\newcommand\hwnumber{HW 7}                   % <-- homework number
\newcommand\idgroup{111-2023}                
\newcommand\idname{Mykhailo Koreshkov}  

\usepackage[framemethod=TikZ]{mdframed}
\mdfsetup{%
	backgroundcolor = black!5,
}
\mdfdefinestyle{ans}{%
    backgroundcolor = green!5,
    linecolor = green!50,
    linewidth = 1pt,
}

\pagestyle{fancyplain}
\headheight 35pt
\lhead{\idgroup \\ \idname}
\chead{\textbf{\Large \hwnumber}}
\rhead{\course \\ \today}
\lfoot{}
\cfoot{}
\rfoot{\small\thepage}
\headsep 1.5em

\linespread{1.1}

\newcommand{\R}{\mathbb{R}}
\newcommand{\N}{\mathbb{N}}
\newcommand{\Z}{\mathbb{Z}}
\newcommand{\F}{\mathbb{F}}
\newcommand{\Q}{\mathbb{Q}}
\DeclareMathOperator{\lcm}{lcm}
\DeclareMathOperator{\cd}{CD}
\DeclareMathOperator{\ch}{char}
\DeclareMathOperator{\ob}{Ob}
\DeclareMathOperator{\mor}{Mor}
\DeclareMathOperator{\catring}{Ring}

\newtheorem*{proposition}{Твердження}
\newtheorem*{definition}{Визначення}

\begin{document}


\section*{Ex 7.1}
\begin{mdframed}
Нехай $R,S$ - кільця. Нехай $I$ - ідеал в $R$, $J$ - ідеал в $S$.\\
Довести що $I \times J$ - ідеал в $R \times S$
\end{mdframed}

\begin{mdframed}[backgroundcolor=blue!10]
    $I$ - (двосторонній) ідеал кільця $R$ якщо $I$ \textbf{замкнена 
    відносно додавання} та виконується \textbf{властивість поглинання}:
    \[\forall a \in R: aI \subset I, Ia \subset I,\]
    де $aI = \{ai : i\in I\}$, $Ia = \{ia : i\in I\}$
\end{mdframed}

\begin{proof}
    Маємо 
    \[\forall a,b \in I: a+b \in I; \qquad \forall a,b \in J: a+b \in J\]
    \[\forall a\in R, u\in I: au \in I\]
    \[\forall b\in S, v\in J: bv \in J\]

    Тоді розглянемо $K = I \times J \subset R \times S$
    \[\forall x=(a,b),y=(c,d) \in I \times J: x+y = (a+c, b+d)\]
    З замкненості ідеалів $I,J$:
    \[a+c \in I, \quad b+d \in J\]
    тобто
    \[x+y = (a+c, b+d) \in I \times J\]
    Довели замкненість $K$ відносно додавання.

    \[\forall x=(a,b) \in R \times S, \; y=(u,v) \in I \times J: xy = (au, bv)\]
    З властивості поглинання ідеалів
    \[au \in I, \quad bv \in J\]
    тобто
    \[xy = (au, bv) \in I \times J\]

    Таким чином довели, що $I \times J$ - відповідний (правий, лівий, чи двосторонній) ідеал кільця $R \times S$.
\end{proof}

\newpage
\section*{Ex 7.2}
\begin{mdframed}
    $\{I_\alpha\}_{\alpha\in A}$ - довільна множина ідеалів кільця $R$ (не обов'язково всі ідеали кільця, просто деякі).

    Довести, що 
    \begin{itemize}
        \item $I = \cap_{\alpha\in A} I_\alpha$ - ідеал $R$
        \item $\forall B\subset R: \exists I$ - ідеал $R: \left(B \subset I\right) \land \left(J \text{ - ideal} \land (B \subset J) \implies I \subset J\right)$.\\
        Тобто існує $I$ - найменший ідеал $R$ що містить $B$.
    \end{itemize}
\end{mdframed}

\begin{proof}[Доведення 1.]
    За визначенням перетину множин,
    \[I = \{x \in R: \forall \alpha \in A: x \in I_\alpha\}\]
    тобто це елементи кільця $R$ що містяться одночасно в усіх ідеалах $I_\alpha$. 

    Розглянемо довільну суму елементів $I$. Нехай $a,b \in I$.\\
    Зрозуміло, що $a,b$ є елементами одночасно всіх ідеалів $I_\alpha$ і виконується
    \[\forall \alpha\in A: a+b \in I_\alpha\]
    Тобто, за визначенням,
    \[a+b \in I\]
    Довели замкненість $I$ відносно додавання.

    Нехай $a \in R, u \in I$.\\
    Знову, зауважу, що $u$ є елементом одночасно всіх ідеалів $I_\alpha$.
    Справедливо
    \[\forall \alpha\in A: au \in I_\alpha\]
    Тобто, за визначенням
    \[au \in I\]
    Довели властивість поглинання для $I$.

    Таким чином довели, що $I$ - дійсно ідеал кільця $R$.
\end{proof}

\begin{proof}[Доведення 2.]
    Розгляну множину $T = \{I: I$ - ідеал $R\}$ всіх ідеалів кільця.
    % На ній можна задати частковий порядок за відношенням вкладення множин, тобто $(T, \subseteq)$.
    % Це дійсно частковий порядок (рефлексивність, транзитивність, антисиметричність), але не повний.

    Для довільної підмножини $B \subseteq R$ розгляну множини
    \[T_B = \{I \in T: B \subseteq I\}\]
    \[I_B = \bigcap_{I \in T_B}I\]

    З першого пункту, $I_B$ - також ідеал $R$. Доведу, що він дійсно найменший з усіх, що містить $B$.

    Нехай $J$ - ідеал кільця $R$, такий що $B \subseteq J$.\\
    Зрозуміло, що $J \in T_B$ за побудовою $T_B$. Тоді $I_B \subseteq J$ як перетин сім'ї множин що містить $J$.\\
    Тобто маємо
    \[\forall J \text{ - ideal of } R : (B \subset J) \implies I_B \subset J\]
    
    Таким чином довели, що $I=I_B$ - дійсо найменший ідеал що містить $B$.
\end{proof}


\section*{Ex 7.3}
\begin{mdframed}
    Нехай $I,J$ - ідеали кільця $R$.\\
    Нехай $I \cup J$ задовільняє властивості поглинання.\\
    Показати, що $I\cup J$ в загальному випадку не буде ідеалом $R$.
\end{mdframed}

За критерієм ідеалу, множині $I \cup J$ не вистачає замкненості відносно додавання.

Маємо
\[\forall a\in R, u \in I, v \in J: au \in I, av \in J\]
\[\forall a\in R, u \in (I\cup J): au \in (I \cup J)\]

Побудуємо відображення $f : R \to R$ таке що $f(x)\cdot f(y) = f(x\cdot y)$, але $\exists x,y: f(x)+f(y) \ne f(x+y)$.

Зауважу, що $\forall n\in\N: n\Z = \{nk : k\in \Z\}$ є ідеалом $\Z$:
\[\forall a,b\in n\Z: a+b = n(k_a + k_b) \in n\Z\]
\[\forall x\in\Z, u\in n\Z: xu = xnk = n(xk) \in n\Z\]

Нехай
\[A = 2\Z \cup 3\Z\]
\[\forall a \in A: \left[\begin{array}{l}
    \exists k\in Z: a = 2k\\
    \exists k\in Z: a = 3k
\end{array} \right.\]
Тоді
\[\forall x \in \Z, a \in A: \left[\begin{array}{l}
    xa = x\cdot 2k = 2kx = 2l\\
    xa = x\cdot 3k = 3kx = 3l
\end{array} \right. \implies xa \in A\]
тобто, $A$ має властивість поглинання.

Але для $a=2, b=3 \in A$ маємо
\[a+b = 5 \notin A\]
Тобто, $A$ не замкнена відносно додавання.

Отже, показали, що $I\cup J$ в загальному випадку не буде ідеалом.

\section*{Ex 7.4}
\begin{mdframed}
    Нехай $I$ - ідеал кільця $\Z$.\\
    Довести, що $\exists n\in\Z: I = n\Z$.
\end{mdframed}

\begin{proof}
    Нехай $I$ - ідеал кільця $\Z$.

    Розгляну $A = \{a \in I : a > 0\} \subset \N$.\\

    Припустимо, що $A = \varnothing$. Тоді $\forall a \in I: a \le 0$. 
    Але тоді $(-1) \cdot a \notin I$, тобто не виконується властивість поглинання. 
    Отже, множина $A$ непорожня.

    Тоді $A$ має найменший елемент $m = \min \{a \in I: a>0\}$ як непорожня підмножина цілком впорядкованої $\N$.


    Припустимо, що $m=1$. Тоді $\forall k \in \Z: k\cdot 1 = k \in I$ з чого $I = \Z = (1) = 1\Z$\\
    Далі вважаємо, що $m\ge 2$.\\

    З одного боку, з властивості поглинання та оскільки $m \in I$ 
    \[\forall k \in \Z: km \in I\]
    \[(m) \subseteq I\]
    де $(m)=m\Z$ - головний ідеал в $\Z$, породжений $m$.\\

    Нехай $a \in I$. Виконаю ділення з остачею на $m$:
    \[a = mk + r, \quad k\in\Z, 0\le r < m\]
    Тоді
    \[a-r = mk \implies (a-r) \in (m) \subseteq I\]
    \[\begin{cases}
        a \in I\\
        (a-r) \in I
    \end{cases} \implies r \in I\]
    За визначенням $m$,
    \[\forall u \in I: u>0 \implies m \le u\]
    З іншого боку, $0 \le r < m$ з визначення остачі.
    Маємо
    \[\begin{cases}
        0 \le r < m\\
        r>0 \implies r \ge m
    \end{cases} \implies r=0\]
    отож, ділення виконується без остачі
    \[\forall a\in I: \exists k\in\Z: a = mk \in (m)\]
    \[I \subseteq (m)\]

    Маємо $(m) \subseteq I \land I \subseteq (m)$
    \[I = (m),\]
    де $m = \min \{a \in I: a>0\}$
\end{proof}

\section*{Ex 7.5}
\begin{mdframed}
    
\end{mdframed}



\end{document}

