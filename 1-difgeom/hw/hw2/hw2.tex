\documentclass[11pt, a4paper]{article} % use larger type; default would be 10pt

\usepackage{fontspec} % Font selection for XeLaTeX; see fontspec.pdf for documentation
\defaultfontfeatures{Mapping=tex-text} % to support TeX conventions like ``---''
\usepackage{xunicode} % Unicode support for LaTeX character names (accents, European chars, etc)
\usepackage{xltxtra} % Extra customizations for XeLaTeX
\usepackage{tikz}
\usetikzlibrary{arrows,calc,patterns}


% other LaTeX packages.....
\usepackage{fullpage}
\usepackage[top=2cm, bottom=4.5cm, left=2.5cm, right=2.5cm]{geometry}
\usepackage{amsmath,amsthm,amsfonts,amssymb,amscd,systeme}
\usepackage{unicode-math}
\usepackage{cancel}
\geometry{a4paper} 
%\usepackage[parfill]{parskip} % Activate to begin paragraphs with an empty line rather than an indent
\usepackage{fancyhdr}
\usepackage{listings}
\usepackage{graphicx}
\usepackage{hyperref}
\usepackage{multicol}

% FONTS
% \setmainfont[Ligatures=TeX]{Cambria Math} % set the main body font (\textrm), assumes Charis SIL is installed
%\setsansfont{Deja Vu Sans}
% \setmonofont[Ligatures=TeX]{Fira Code}
\setmathfont[Ligatures=TeX]{NewCMMath-Regular}

\setmainfont{Cambria}
\setmonofont[Ligatures=TeX]{Roboto Mono}

\renewcommand\lstlistingname{Algorithm}
\renewcommand\lstlistlistingname{Algorithms}
\def\lstlistingautorefname{Alg.}
\lstdefinestyle{mystyle}{
    % backgroundcolor=\color{backcolour},   
    % commentstyle=\color{codegreen},
    % keywordstyle=\color{magenta},
    % numberstyle=\tiny\color{codegray},
    % stringstyle=\color{codepurple},
    basicstyle=\ttfamily\footnotesize,
    breakatwhitespace=false,         
    breaklines=true,                 
    captionpos=b,                    
    keepspaces=true,                 
    numbers=left,                    
    numbersep=5pt,                  
    showspaces=false,                
    showstringspaces=false,
    showtabs=false,                  
    tabsize=2
}
\lstset{style=mystyle}

\newcommand\course{MS1 - Differential geometry and topology}
\newcommand\hwnumber{HW 2}                   % <-- homework number
\newcommand\idgroup{111-2023}                
\newcommand\idname{Mykhailo Koreshkov}  

\usepackage[framemethod=TikZ]{mdframed}
\mdfsetup{%
	backgroundcolor = black!5,
}
\mdfdefinestyle{ans}{%
    backgroundcolor = green!5,
    linecolor = green!50,
    linewidth = 1pt,
}

\pagestyle{fancyplain}
\headheight 35pt
\lhead{\idgroup \\ \idname}
\chead{\textbf{\Large \hwnumber}}
\rhead{\course \\ \today}
\lfoot{}
\cfoot{}
\rfoot{\small\thepage}
\headsep 1.5em

\linespread{1.2}

\newcommand{\R}{\mathbb{R}}
\newcommand{\N}{\mathbb{N}}
\newcommand{\Z}{\mathbb{Z}}
\DeclareMathOperator{\lcm}{lcm}
\DeclareMathOperator{\cd}{CD}


\begin{document}


\section*{Ex 2.1}
\begin{mdframed}
Нехай $X$ - топологічний простір. Показати еквівалентність
\begin{enumerate}
    \item $\forall x \in X: \exists n\ge 0, U \text{ - open in } X, \varphi:U \to \R^n \text{ - hom}$
    \item $\forall x \in X: \exists n\ge 0, U \text{ - open in } X, \varphi:U \to \varphi(U) \subset \R^n \text{ - hom}$
\end{enumerate}
\end{mdframed}
Доведення 1->2 очевидне. Доводжу 2->1.
\begin{proof}
    Фіксуємо 
    $x\in X, n\ge0, U \text{ - open in } X, \varphi:U \to \varphi(U) \subset \R^n \text{ - hom}$.\\
    Хочемо довести існуваня гомеомеорфізму з (можливо іншого) околу точки $x$ на весь $\R^n$. 
    Число $n$ залишаємо те саме через теорему про інваріантність області.
\begin{itemize}
    \item 
    Нехай $\varphi(x) = y, \varphi(U) = Y$.\\
    Оскільки $\varphi^{-1}$ - неперервне, з відкритості $U$ випливає відкритість $\varphi(U)$.
    З визначення відкритої множини в $\R^n$ можемо вписати в $Y$ відкриту кулю навколо $y$:
    \[\exists r>0: B_r(y) \subset Y\]
    Де $B_r(y) = \{z\in \R^n : |z-y| < r\}$.
    
    Нехай $B = B_r(y), V = \varphi^{-1}(B)$. Зауважу, що $\varphi$ - неперервна бієкція, а тому $V$ відкрита та містить $x$.
    Тобто, $V$ - дійсно відкритий окіл $x$ в просторі $X$.

    \item
    Розглянемо 
    \[\psi(v) = \frac{\frac{v-y}{r}}{1 - \frac{|v-y|^2}{r^2}} : B_r(y) \to \R^n\]
    Відомо, що $\psi$ - гладкий гомеомеорфізм.
    Тоді 
    \[\alpha = \psi\circ\varphi : V \to \R^n, \quad \alpha \text{ - hom}\] 
\end{itemize}
Отже, знайшли $V$ - окіл $x$ та $\alpha : V \to \R^n$ - hom для довільного $x \in X$.
\end{proof}

\section*{Ex 2.2}
\begin{mdframed}
    Перевірити наступні властивості топологічного многовиду $X$.
\end{mdframed}

\subsection*{Ex 2.2.1}
\begin{mdframed}
    $X$ є локально лінійно зв'язним, тобто у кожної точки є база, що складається з лінійно зв'язних околів.
\end{mdframed}

\begin{mdframed}[backgroundcolor=blue!10]
    Нехай $(X, \tau)$ - топ простір. $V(x) = \{V\in\tau : x\in V\}$ - всі околи $x$.\\
    База $B = \{B_\alpha\}$ околу точки $x$ топ. простору $X$:
    \[
    \begin{cases}
        B \subset V(x)\\
        \forall V\in V(x): \exists B_\alpha: B_\alpha \subset V
    \end{cases}    
    \]
\end{mdframed}

Локальна лінійна зв'язність вимагає щоб кожна множина бази була лінійно зв'язна.

% Тут багато необов'язкових пунктів доведення, але вони корисні мені для розуміння 
% топологічної природи многовидів над $\R$. Основне доведення в останньому пункті.

% \begin{proof}
%     \begin{itemize}
%         \item 
%         По-перше, фіксуємо точку $x \in X$ та деяку карту $(U, \phi)$, $\phi : U \to \R^n$ - hom.
        
%         Нехай $y = \phi(x)$.
%         Розглянемо систему околів $N_{\R^n}(y) = \{V \in \R^n : V {- open}, y \in V\}$.
%         Відомо, що у такої системи є зліченна база $B(y)$ 
%         (взагалі, весь $\R^n$ має зліченну базу).

%         \item
%         Нехай $M = \{\phi^{-1}(V) : V \in B(y)\}$. 
%         Доведу, що це база системи околів точки $x$ у просторі $X$.
%         \begin{proof}
%             Позначу систему околів точки $x$ як 
%             $N(x) = \{V\cap U \in X : V \text{ - open}, x\in V\}$.\\
%             Зрозуміло, що також $N(x) = \{\phi^{-1}(V) : V \in N_{\R^n}(y)\}$, оскільки $\phi$ - hom.

%             Візьмемо довільний $W \in N(x)$.
%             Для $\phi(W)$ знайдеться $B_1 \in B(y)$ такий, що $B_1 \subset \phi(W)$.
%             З властивостей гомеомеорфізму, $\phi^{-1}(B_1) \subset \phi^{-1}(\phi(W)) = W$.\\
%             З іншого боку, $\phi^{-1}(B_1) \in M$.

%             Отже, 
%             \[\forall W\in N(x): \exists B_\alpha \in M: B_\alpha \subset W\]
%             Тобто, $M$ - база точки $x$ в $X$.
%         \end{proof}
%         Позначу базу точки $x$ як $B(x) = \{B_n\}$.

%         \item
%         Переходимо до лінійної зв'язності.
%         Нехай $B(x)$ - база точки $x\in X$. Нехай $(U,\phi)$ - карта навколо $x$.\\
%         Відомо, що $\R^n$ - лінійно зв'язний простір.
%         Тобто
%         \[\forall u,v \in \R^n : \exists w:[0,1]\to \R^n, w \in C^0: w(0)=u \land w(1)=v\] 
        
%         Фіксуємо довільні $B_n\in B(x), u,v\in B_n$.
%         Нехай $u' = \phi(u), v' = \phi(v)$. 
%         Існує неперервна $w':[0,1]\to \R^n$, $w'(0) = u', w'(1)=v'$.
%         Оскільки $\phi$ - hom, то $w = \left(\phi^{-1} \circ w'\right), w : [0,1] \to U$ - також неперервна.

%         $w$ - шукана неперервна крива що з'єднує довільні точки в будь якому елементі базису будь якої точки
%     \end{itemize}
% \end{proof}

\begin{proof}
    Нехай $B(x)=\{B_\alpha\}$ - база точки $x\in X$. 
    Фіксуємо довільні $B_1\in B(x), u,v\in B_1$.

    Припустимо, що існує 
    Візьмемо довільну карту $(U, \phi)$ навколо точки $x$.
    $\exists B_1$

    Відомо, що $\R^n$ - лінійно зв'язний простір.
    Тобто
    \[\forall u,v \in \R^n : \exists w:[0,1]\to \R^n, w \in C^0: w(0)=u \land w(1)=v\] 
    
    Нехай $u' = \phi(u), v' = \phi(v)$. 
    Існує неперервна $w':[0,1]\to \R^n$, $w'(0) = u', w'(1)=v'$.
    Оскільки $\phi$ - hom, то $w = \left(\phi^{-1} \circ w'\right), w : [0,1] \to U$ - також неперервна.

\end{proof}

% todo: використати індуковану картою евклідову топологію 

\end{document}

