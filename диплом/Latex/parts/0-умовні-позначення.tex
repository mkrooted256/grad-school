
% \bigskip


\begin{tabular}{ll}

    $\R, \Z$ & Множини дійсних та цілих чисел, відповідно \\
    $\1(A)$ & Індикатор; випадкова величина, якщо $A$ -- подія\\
    DFT & Дискретне перетворення Фур'є\\
    $\# A$ & Кількість елементів множини $A$\\

% $\mathbb R^n$ & $n$-вимірний евклідів простір\\[1mm]
% $2^A$ & Булеан множини $A$, $2^A = \{B \mid B \subset A\}$ \\
% $\Omega, E$ & Простори елементарних подій \\
% $\mathcal F \subset 2^\Omega, \; \mathcal E \subset 2^E$ & $\sigma$-алгебри на просторі елементарних подій \\
% $A, B, C \in \mathcal F$ & Випадкові події, елементи $\sigma$-алгебри \\
% $1_A(x)$ & Індикатор події (вимірної множини) $A$ \\
% $\P(A)$ & Ймовірність події $A$ \\
% $\P(A\mid B)$ & Умовна ймовірність події $A$ за умови $B$ \\
% $X, Y, Z, \xi, \psi, \zeta$ & Випадкові величини (в.в.) \\
% $F, F_X$ & Функція розподілу в.в. \\
% $\E X, \D X$ & Математичне сподівання та дисперсія в.в. \\
% $\Theta \ni \theta$ & Простір параметрів. $\theta = (\theta_1, ..., \theta_n) \in \Theta \subset \mathbb R^n$ \\
% $\P_\theta(A),\;\E_\theta X,\;\D_\theta X$ & Позначення для $\P(A\mid \theta), \E(X \mid \theta), \D( X \mid \theta)$ \\
% $\text{iid}$ & незалежні в сукупності однаково розподілені в.в. \\
% $\text{a.s.}$ & майже скрізь; із ймовірністю 1 \\
% $X_1, ..., X_n \sim F \quad \text{iid}$ & Послідовність незалежних однаково розподілених \\
% ~ & випадкових величин із законом розподілу $F$ \\
% $\nu_a$ & Частота значення $a$ у послідовності iid; \\
% ~ & $\nu_a = \sum_{k=1}^n 1_{\{a\}}(X_k)$ \\

\end{tabular}